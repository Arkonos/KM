\section{Pharmakologie: Analgetika}
\subsection{Analgetika-Grundlagen}
	\begin{itemize}
		\item \textbf{Medikamente zur Schmerzbekämpfung}
		\item \textbf{Schmerz}
			\begin{itemize}
				\item \textbf{Symptom}
				\item \textbf{Warn- und Schutzfunktion}
			\end{itemize}
		\item \textbf{Einteilung}
			\begin{itemize}
				\item \textbf{Nicht-Opioid-Analgetika}
					\begin{itemize}
						\item \textbf{Unterdrückung der Schmerzübertragung v.a. im PNS}
					\end{itemize}
				\item \textbf{Opioid-Analgetika (="Opiate")}
					\begin{itemize}
						\item \textbf{Schmerzunterdrückung v.a. im ZNS (Bsp. Morphine)}
					\end{itemize}
			\end{itemize}
	\end{itemize}

\subsection{Physiologie des Schmerzes}
	\begin{itemize}
		\item \textbf{Schmerzentstehung}
			\begin{itemize}
				\item \textbf{Gewebeschädigung}
				\item \textbf{Freisetzung von Schmerz-Mediatoren}
				\item \textbf{Erregung der Schmerzrezeptoren}
				\item \textbf{Weiterleitung zum Gehirn}
				\item \textbf{bewusste Wahrnehmung}
			\end{itemize}
		\item \textbf{Schmerzqualitäten}
			\begin{itemize}
				\item \textbf{somatische Schmerzen}
					\begin{itemize}
						\item \textbf{Oberflächenschmerz}
						\item \textbf{Tiefenschmerz}
					\end{itemize}
				\item \textbf{viszerale Schmerzen}
					\begin{itemize}
						\item \textbf{Eingeweideschmerzen}
					\end{itemize}
			\end{itemize}
	\end{itemize}
						
\subsection{Analgesie - medikamentöse Schmerzlinderung}
	\begin{itemize}
		\item \textbf{Hemmung der peripheren Schmerzrezeptoren}
			\begin{itemize}
				\item \textbf{Hemmung der Synthese und Freisetzung von Prostaglandinen}
					\begin{itemize}
						\item \textbf{Prostaglandine sind Gewebshormone, die Entzündungs- und Schmerzreize vermitteln}
					\end{itemize}
				\item \textbf{Bsp.: Nicht-Opioide-Analgetika}
			\end{itemize}
		\item \textbf{Hemmung der Erregungsleitung}
			\begin{itemize}
				\item \textbf{Lokalanästhesie}
			\end{itemize}
		\item \textbf{Hemmung der zentralen Schmerzrezeptoren}
			\begin{itemize}
				\item \textbf{Opiate (Bsp. Morphin) besetzen die physiologischen Opiatrezeptoren im körpereigenen schmerzhemmenden System}
			\end{itemize}
		\item \textbf{Beeinflussung des Schmerzerlebnisses}
			\begin{itemize}
				\item \textbf{Psychopharmaka (Neuroleptika, Antidepressiva)}
				\item \textbf{Opiate}
			\end{itemize}
	\end{itemize}

\subsection{Anwendung der Analgetika}
	\begin{itemize}
		\item \textbf{Anwendung nicht-opioider Analgetika}
			\begin{itemize}
				\item \textbf{leichter bis mittelschwerer Schmerz}
					\begin{itemize}
						\item \textbf{entzündliche Schmerzen, Fieber}
						\item \textbf{Kopf- und Zahnschmerzen, Migräne}
						\item \textbf{Erkrankungen des rheumatischen Formenkreises}
					\end{itemize}
			\end{itemize}
		\item \textbf{Anwendung opioider Analgetika}
			\begin{itemize}
				\item \textbf{starker Schmerz}
					\begin{itemize}
						\item \textbf{traumatische Schmerzen}
						\item \textbf{Tumorschmerzen}
						\item \textbf{operative und postoperative Schmerzdämpfung}
					\end{itemize}
			\end{itemize}
	\end{itemize}

\subsection{Nicht-opioide-Analgetika}
	\begin{itemize}
		\item \textbf{Übersicht}
			\begin{itemize}
				\item \textbf{Acetylsalicylsäure}
				\item \textbf{Paracetamol}
				\item \textbf{Nichtsteroidale Antirheumatika}
				\item \textbf{COX-II-Hemmer (Cyclooxygenase-Hemmer)}
				\item \textbf{$\dots$}
			\end{itemize}
		\item \textbf{Wirkung}
			\begin{itemize}
				\item \textbf{analgetisch}
				\item \textbf{antiphlogistisch}
				\item \textbf{antipyretisch}
			\end{itemize}
	\end{itemize}
				
\subsubsection{Acetylsalicylsäure}
	\begin{itemize}
		\item \textbf{Wirkungen}
			\begin{itemize}
				\item \textbf{analgetisch}
				\item \textbf{schwach antiphlogistisch}
				\item \textbf{antipyretisch}
				\item \textbf{zusätzlich in niedriger Dosierung Thrombozytenaggregationshemmend (Infarkt-Prophylaxe)}
			\end{itemize}
		\item \textbf{Nebenwirkungen}
			\begin{itemize}
				\item \textbf{Blutungsneigung bei Langzeiteinnahme}
				\item \textbf{Sodbrennen, Magengeschwüre}
				\item \textbf{bei Dauergebrauch Verschlechterung der Nierenfunktion("Analgetika-Niere")}
			\end{itemize}
		\item \textbf{Kontraindikationen}
			\begin{itemize}
				\item \textbf{Magen-Duodenum-Ulcera}
				\item \textbf{cave: Asthma bronchiale}
				\item \textbf{geplante größere OP's (vorher Absetzen!)}
			\end{itemize}
		\item \textbf{Medikament}
			\begin{itemize}
				\item \textbf{Aspirin$^{\textregistered}$}
				\item \textbf{$\dots$}
			\end{itemize}
	\end{itemize}
		
\subsubsection{Paracetamol}
	\begin{itemize}
		\item \textbf{bei richtiger Dosierung gehört es zu den sichersten Analgetika} (Kinder, Schwangere, Stillende)
		\item \textbf{Medikamente}
			\begin{itemize}
				\item \textbf{Monopräparate: Mexalen$^{\textregistered}$, Perfalgan$^{\textregistered}$, …}
				\item \textbf{Kombinationspräp: Thomapyrin$^{\textregistered}$, $\dots$}
			\end{itemize}
		\item \textbf{Wirkung}
			\begin{itemize}
				\item \textbf{schmerzlindernd}
				\item \textbf{fiebersenkend}
				\item \textbf{(sehr schwach entzündungshemmend)}
			\end{itemize}
		\item \textbf{Nebenwirkung}
			\begin{itemize}
				\item \textbf{Leberzellschädigung bei vorgeschädigter Leber und Überdosierung}
					\begin{itemize}
						\item \textbf{(Leberversagen bei Dosis über 10 g, maximale Tagesdosis soll 4 g nicht überschreiten)}
					\end{itemize}
				\item \textbf{Nierenschädigung (Analgetika-Niere!)}
			\end{itemize}
		\item \textbf{Kontraindikationen}
			\begin{itemize}
				\item \textbf{Achtung bei Patienten mit vorgeschädigter Leber (Leberzirrhose) oder Niere}
			\end{itemize}
	\end{itemize}
				
\subsubsection{Nicht-steroidale Antirheumatika}
	\begin{itemize}
		\item \textbf{NSAR / NSAP / NSAID} (non-steroidal anti-inflammatory drug, Antiphlogistika)
		\item \textbf{Medikamente z.B.}
			\begin{itemize}
				\item \textbf{Diclofenac: Voltaren$^{\textregistered}$}
				\item \textbf{Naproxen: Proxen$^{\textregistered}$}
				\item \textbf{Ibuprofen: Brufen$^{\textregistered}$}
			\end{itemize}
		\item \textbf{Wirkung}
			\begin{itemize}
				\item \textbf{stark antiphlogistisch}
				\item \textbf{antipyretisch}
				\item \textbf{analgetisch}
			\end{itemize}
		\item \textbf{Indikationen}
			\begin{itemize}
				\item \textbf{rheumatische Erkrankungen}
				\item \textbf{degenerative Erkrankungen der Gelenke und Wirbelsäule}
				\item \textbf{Zahn-, Kopf-, Menstruationsschmerzen}
				\item \textbf{Fieber}
			\end{itemize}
		\item \textbf{Nebenwirkungen}
			\begin{itemize}
				\item \textbf{Magen-Darmstörungen (z.B. Ulcus ventriculi)}
				\item \textbf{zentralnervöse Symptome wie Schwindel, Kopfschmerzen}
				\item \textbf{pseudoallergische Reaktionen wie Bronchospasmen}
			\end{itemize}
		\item \textbf{Besonderes}
			\begin{itemize}
				\item \textbf{bei längerer Gabe von NSAR - Magenschutz!}
			\end{itemize}
	\end{itemize}
		
\subsection{zentral wirksame Analgetika}
\subsubsection{Opioide}
	\begin{itemize}
		\item \textbf{Wirkung an den zentralen Opioid-Rezeptoren des körpereigenen schmerzhemmenden Systems}
		\item \textbf{Unterdrückung der Weiterleitung und Verarbeitung von Schmerzreizen}
		\item \textbf{Wirkung aller Opioide gleich}
		\item \textbf{Unterschiede in der Wirkintensität}
		\item \textbf{zentrale Wirkung}
			\begin{itemize}
				\item \textbf{starke Schmerzdämpfung}
				\item \textbf{beruhigend}
				\item \textbf{Beseitigung von Angstgefühl, Verbesserung der Stimmungslage}
				\item \textbf{Atemdepression (Hemmung des Atemzentrums)}
				\item \textbf{Engstellung der Pupillen (Miosis)}
				\item \textbf{Toleranz- und Suchtauslösung bei wiederholter Gabe}
			\end{itemize}
		\item \textbf{periphere Wirkung}
			\begin{itemize}
				\item \textbf{Herabsetzung Magen-Darmtätigkeit (Obstipation)}
				\item \textbf{Miktionsstörungen, Harnverhalt}
				\item \textbf{Kontraktion der Gallenwege}
				\item \textbf{Blutdrucksenkung}
			\end{itemize}
	\end{itemize}
				
\subsubsection{(sehr) starke Opioide}
	\begin{itemize}
		\item \textbf{in Verwendung sind}
			\begin{itemize}
				\item \textbf{Morphin = natürliches Opiumalkaloid}
				\item \textbf{halbsynthetische Morphinderivate}
				\item \textbf{synthetische Opioide}
			\end{itemize}
		\item \textbf{Medikamente zB.}
			\begin{itemize}
				\item \textbf{Morphin: MST 10-100$^{\textregistered}$}
				\item \textbf{Fentanyl:  Durogesic$^{\textregistered}$}
				\item \textbf{Piritramid: Dipidolor$^{\textregistered}$}
				\item \textbf{Buprenorphin: Temgesic$^{\textregistered}$}
			\end{itemize}
		\item \textbf{Indikationen}
			\begin{itemize}
				\item \textbf{stärkste Schmerzzustände (zB. postop., Infarkt, …)}
				\item \textbf{schwere Tumorschmerzen}
			\end{itemize}
		\item \textbf{Kontraindikationen}
			\begin{itemize}
				\item \textbf{Lungenemphysem, Asthma bronchiale}
				\item \textbf{Gallenkolik}
				\item \textbf{whd. der Geburt, Stillende}
			\end{itemize}
		\item \textbf{zentrale Nebenwirkungen}
			\begin{itemize}
				\item \textbf{Atemdepression}
				\item \textbf{Erbrechen (klingt nach wiederholter Gabe ab)}
				\item \textbf{Euphorie und Suchtauslösung}
			\end{itemize}
	\pagebreak
		\item \textbf{periphere Nebenwirkungen}
			\begin{itemize}
				\item \textbf{Bradykardie, Blutdrucksenkung}
				\item \textbf{Harnverhalt, Obstipation}
			\end{itemize}
		\item \textbf{Vergiftungssymptome}
			\begin{itemize}
				\item \textbf{Miosis}
				\item \textbf{Atemdepression}
				\item \textbf{Koma}
			\end{itemize}
	\end{itemize}
		
\subsubsection{schwache Opioide}
	\begin{itemize}
		\item \textbf{bei mittelstarken Schmerzen}
		\item \textbf{häufig in Kombination mit Nicht Opioid Analgetika}
		\item \textbf{Medikamente z.B.}
			\begin{itemize}
				\item \textbf{Pethidin:  Dolantin$^{\textregistered}$}
				\item \textbf{Codein:  Codiopt$^{\textregistered}$}
				\item \textbf{Tramadol: Tramal$^{\textregistered}$}
				\item \textbf{Tilidin + Naloxon:  Valoron$^{\textregistered}$ N}
			\end{itemize}
		\item \textbf{Wirkung}
			\begin{itemize}
				\item \textbf{wie starke Opioide, nur weniger ausgeprägt}
			\end{itemize}
		\item \textbf{Nebenwirkungen}
			\begin{itemize}
				\item \textbf{wie starke Opioide, aber schwächer ausgeprägt}
			\end{itemize}
		\item \textbf{Besonderes}
			\begin{itemize}
				\item \textbf{Codein und verwandte Substanzen werden als Antitussiva eingesetzt}
				\item \textbf{unterdrücken den Hustenreiz}
			\end{itemize}
	\end{itemize}