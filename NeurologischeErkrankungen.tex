\section{Neurologische Erkrankungen}
	\subsection{Übersicht}
		\begin{itemize}
			\item \textbf{Bewusstseinsstörungen (Übersicht)}
			\item \textbf{Epilepsie}
			\item \textbf{Entzündungen, MS}
			\item \textbf{Morbus Parkinson}
			\item \textbf{cerebrovaskuläre Erkrankungen}
			\item \textbf{Lähmung (Übersicht)}
			\item \textbf{Hirndruck}
			\item \textbf{Demenzen (Übersicht)}
			\item \textbf{Transmissible Spongiforme Enzephalopathie (TSE)}
			\item \textbf{Tumoren}
			\item \textbf{Poyneuropathien}
		\end{itemize}
	\subsection{Bewusstseinsstörungen}
		\begin{itemize}
			\item \textbf{Benommenheit}
			\item \textbf{Somnolenz}
				\begin{itemize}
					\item \textbf{schläfrig, apathisch, aber weckbar, bedingt kooperativ}
				\end{itemize}
			\item \textbf{Sopor}
				\begin{itemize}
					\item \textbf{ähnlich dem Tiefschlaf, nur durch starke Reize (Schmerz) weckbar, gerichtete Abwehr}
				\end{itemize}
			\item \textbf{Koma}
				\begin{itemize}
					\item \textbf{nicht weckbar, Augen geschlossen, mit Intaktheit der vegetativen Funktionen vereinbar; vier Schweregrade}
				\end{itemize}
		\end{itemize}
		
	\subsection{Epilepsie}
		\begin{itemize}
			\item \textbf{Episoden chaotischer elektrischer Entladungen im Gehirn}
				\begin{itemize}
					\item \textbf{können das gesamte Gehirn oder einen umschriebenen Teil betreffen $\rightarrow$ Unteschiede in der Form des Anfalls}
						\begin{itemize}
							\item \textbf{Grand mal Anfälle: tonisch-klonische Krämpfe}
							\item \textbf{Absencen: Patient wirkt "geistig" abweisend}
							\item \textbf{Anfälle mit unkontrollierten Bewegungen einzelner Gliedmaßen, der Patient hat keinerlei Bewusstseinsbeeinträchtigung}
						\end{itemize}
				\end{itemize}
			\item \textbf{Ursachen}
				\begin{itemize}
					\item \textbf{Gehirnerkrankungen (z.B. Entzündungen, Vergiftungen, Tumore, Kopfverletzungen, Schlaganfall, $\dots$)}
				\end{itemize}
			\item \textbf{Häufigkeit: ca. 1\% der Bevölkerung}
			\item \textbf{Diagnose: mittels EEG}
		\end{itemize}
		
	\subsection{Entzündungen}
		\begin{itemize}
			\item \textbf{Einteilung}
				\begin{itemize}
					\item \textbf{Meningitis}
						\begin{itemize}
							\item \textbf{akute bakterielle Meningitis}
							\item \textbf{akute lymphozytäre Meningitis (=viral)}
							\item \textbf{chronisch lymphozytäre Meningitis}
						\end{itemize}
					\item \textbf{Encephalitis}
					\item \textbf{Hirnabszess}
					\item \textbf{Borreliose}
					\item \textbf{multible Sklerose}
				\end{itemize}
		\end{itemize}
	\subsection{Multible Sklerose}
		\begin{itemize}
			\item \textbf{Enzephalitis disseminata}
			\item \textbf{chronisch-entzündliche ZNS-Entmarkungs-KH}
				\begin{itemize}
					\item \textbf{Zerfall der isolierenden Markscheiden im Rahmen einer Entzündung, herdförmiger Myelinverlust an verschiedenen Stellen des Gehirns und des Rückenmarks}
					\item \textbf{Narbenbildung nach Entzündungsrückgang}
				\end{itemize}
			\item \textbf{Ätiologie - ? autoimmun?, slow-virus?}
			\item \textbf{Beginn häufig zw. 20 - 40 a, mehr Freuen, genetische Disposition}
			\item \textbf{Verlauf: sehr variabel (schwierige Prognose)}
				\begin{itemize}
					\item \textbf{schubförmig}
					\item \textbf{chronisch progredient}
				\end{itemize}
			\item \textbf{Symptome}
				\begin{itemize}
					\item \textbf{Sehstörungen: Sehnervenentzündungen, Doppeltsehen}
					\item \textbf{Sensibilitätsstörungen, Lähmungen, Blasen- und Mastdarmstörungen}
					\item \textbf{Kleinhirnsymptome (Sprachstörungen, Zittern, Koordinationsstörungen)}
					\item \textbf{psychische Veränderungen (Depressionen)}
				\end{itemize}
			\item \textbf{Diagnostik}
				\begin{itemize}
					\item \textbf{klinischer Verlauf}
					\item \textbf{Liquor}
					\item \textbf{Evozierte Potentiale}
					\item \textbf{MRT}
				\end{itemize}
			\item \textbf{Therapie}
				\begin{itemize}
					\item \textbf{Glukokortikoide im Schub}
					\item \textbf{Immunsuppression (Interferone, Azathioprin, $\dots$)}
					\item \textbf{symptomatische Th bei Spastik, Blasenstörungen, $\dots$}
				\end{itemize}
		\end{itemize}
	\subsection{Morbus Parkinson}
		\begin{itemize}
			\item \textbf{degenerative Erkrankung mit Zerstörung von Dopamin-produzierenden Strukturen im Gehirn}
			\item \textbf{Folge: Dopaminmangel}
			\item \textbf{Häufigkeit}
				\begin{itemize}
					\item \textbf{etwa 1\% der über 60-Jährigen, mehr Männer}
				\end{itemize}
			\item \textbf{Ursache}
				\begin{itemize}
					\item \textbf{?}
				\end{itemize}
			\item \textbf{Symptomen-Trias}
				\begin{itemize}
					\item \textbf{Rigor (Muskelsteifigkeit)}
					\item \textbf{Tremor (Ruhetremor)}
					\item \textbf{Akinese (Bewegungsarmut)}
				\end{itemize}
			\item \textbf{Therapie}
				\begin{itemize}
					\item \textbf{Dopamin-Ersatz}
				\end{itemize}
		\end{itemize}
	\subsection{zerebrovaskuläre Erkrankungen}
		\begin{itemize}
			\item \textbf{Mangeldurchblutung (Ischämie) des Gehirns}
				\begin{itemize}
					\item \textbf{Hirninfarkt:=ischämischer Insult}
						\begin{itemize}
							\item \textbf{Ursache: Gefäßverschluss durch}
								\begin{itemize}
									\item \textbf{Thrombose einer Zerebralaterie}
									\item \textbf{Embolie (aus A.carotis oder aus dem Herzen: Vorhofflimmern, Klappenerkrankung)}
									\item \textbf{Arteriosklerose}
								\end{itemize}
						\end{itemize}
					\item \textbf{Hirnblutung = intrazerebrales Hämatom (Gefäßruptur)}
						\begin{itemize}
							\item \textbf{bei älteren Menschen: Arteriosklerose, Hypertonie}
							\item \textbf{bei jungen Menschen: Gefäßdefekte (z.B. Aneurysma)}
							\item \textbf{altersunabhängig: Trauma}
						\end{itemize}
				\end{itemize}
		\end{itemize}
	\subsection{Schlaganfall}
		\begin{itemize}
			\item \textbf{Folge: Schlaganfall (syn.: Hirnschlag, Hirninfarkt, Apoplex, Apoplexie, apoplektischer Insult, ischämischer Insult, zerebrovaskulärer Insult)}
				\begin{itemize}
					\item \textbf{akute, zerebrovaskuläre Störung $\rightarrow$ Minderversorgung der Nervenzellen mit Sauerstoff und Nährstoffen $\rightarrow$ Funktionsausfall: Beeinträchtigung der Hirnleistung (motorisch, sensibel, kognitiv)}
					\item \textbf{plötzliches Einsetzen eines neurologischen oder neuropsychologischen Defizits, je nach Lokalisation}
						\begin{itemize}
							\item \textbf{Bewusstseinsstörungen, Gedächtnisverlust, Sprachstörungen, Sensibilitätsausfälle}
							\item \textbf{Hemiparese (Halbseitenlähmung): motorisch, sensorisch oder beides}
						\end{itemize}
				\end{itemize}
			\item \textbf{TIA}
				\begin{itemize}
					\item \textbf{transiente ischämische Attacke = „Streifung“ = akutes zerebrovaskuläres Ereignis mit vorübergehender Hirnleistungsstörung (Dauer: Sekunden bis max. $\dots$)}
				\end{itemize}
			\item \textbf{PRIND}
				\begin{itemize}
					\item \textbf{prolongiertes reversibles ischämisches neurologisches Defizit = akutes zerebrovaskuläres Ereignis, dessen Beeinträchtigung sich innerhalb von $\dots$ vollständig zurückbildet}
				\end{itemize}
	\pagebreak
			\item \textbf{Hauptrisikofaktoren}
				\begin{itemize}
					\item \textbf{Hypertonie}
					\item \textbf{Herzrhythmusstörungen}
					\item \textbf{Arteriosklerose}
				\end{itemize}
			\item \textbf{Diagnostik}
			\item \textbf{Therapie}
			\item \textbf{Rezidivprophylaxe}
		\end{itemize}
	\subsection{Lähmungen}
		\begin{itemize}
			\item \textbf{Lähmun einzelner Nerven}
			\item \textbf{Hemiplegie}
				\begin{itemize}
					\item \textbf{Halbseitenlähmung}
					\item \textbf{Schädigung der motorischen Zentren auf der "gegenüberliegenden" Gehirnhälfte}
					\item \textbf{Schädigung der sensiblen Zentren auf der gleichen Gehirnhälfte}
				\end{itemize}
			\item \textbf{Paraplegie}
				\begin{itemize}
					\item \textbf{Lähmung beider Beine und Teile des Rumpfes durch eine Schädigung des Rückenmarks}
				\end{itemize}
			\item \textbf{Tetraplegie}
				\begin{itemize}
					\item \textbf{Lähmung aller vier Gliedmaßen durch Schädigung des Rückenmarks}
					\item \textbf{je weiter oben die RM Schädigung ist, desto mehr Körperanteile sind betroffen}
				\end{itemize}
		\end{itemize}
	\subsection{Hirndruck}
		\begin{itemize}
			\item \textbf{Volumenzunahme im Schädel führt zu einem intrakraniellen Druckanstieg}
			\item \textbf{steigt der Druck weiter an, kommt es zu einer Verlagerung von Gehirnanteilen nach unten in Richtung Hinterhauptsloch, da dies die einzige Ausweichmöglichkeit ist („Einklemmung“)}
			\item \textit{\textbf{Symptome}}
				\begin{itemize}
					\item \textbf{Kopfschmerzen}
					\item \textbf{Erbrechen}
					\item \textbf{Bewusstseinstrübung}
					\item \textbf{Koma}
					\item \textbf{lebensbedrohlicher Zustand mit Ausfall von Atmung und Kreislauf}
				\end{itemize}
		\end{itemize}
	\subsection{Demenz}
		\begin{itemize}
			\item \textbf{psychopathologisches Symptomenbild mit}
			\item \textbf{Einbußen von Gedächtnisleistungen}
			\item \textbf{Einschränkungen intellektueller Fähigkeiten}
			\item \textbf{Auftreten emotionaler Störungen}
			\item \textbf{Persönlichkeitsveränderungen}
			\item \textbf{nachlassende körperliche Fähigkeiten und körperlicher Abbau}
			\item \textbf{ohne ausgeprägte Bewusstseinstrübung}
		\end{itemize}
		\subsubsection{Einteilung}
			\begin{itemize}
				\item \textbf{primäre Demenzen}
					\begin{itemize}
						\item \textbf{Grunderkrankung im Gehirn, z.B. Alzheimer-Demenz}
					\end{itemize}
				\item \textbf{sekundäre Demenzen}
					\begin{itemize}
						\item \textbf{Gehirn ist im Rahmen einer anderen Grunderkrankung mitbeteiligt}
							\begin{itemize}
								\item \textbf{Herz-Kreislauf-Erkrankungen: vaskuläre Demenzen (Hypertonus!)}
								\item \textbf{akuter Sauerstoffmangel}
								\item \textbf{Stoffwechselerkrankungen}
								\item \textbf{Missbrauch von Medikamenten, Alkohol, Drogen}
								\item \textbf{Schädel-Hirntraumen}
								\item \textbf{etc.}
							\end{itemize}
					\end{itemize}
			\end{itemize}
		\subsubsection{Symptome}
			\begin{itemize}
				\item \textbf{Beginn schleichend, kaum bemerkt, bis verstärkt Auffälligkeiten sichtbar werden}
				\item \textbf{Merkfähigkeitsstörungen}
				\item \textbf{Gedächtnisausfälle, verlangsamte Denkabläufe}
				\item \textbf{Auffassungs-und Konzentrationsstörungen}
				\item \textbf{Reduzierung von Kritik-und Urteilsvermögen, erschwerte Entscheidungsfindung}
				\item \textbf{allgemeine Verlangsamung}
				\item \textbf{Störungen im affektiven Bereich}
				\item \textbf{Distanzlosigkeit, Abstumpfung, Enthemmung}
				\item \textbf{Konfabulation und Perseveration}
				\item \textbf{Depression}
				\item \textbf{Harn-und Stuhlinkontinenz}
				\item \textbf{bei fortgeschrittener Erkrankung: stereotype Bewegungen und  Lautbildungen}
			\end{itemize}
	\subsection{Hirntumore}
		\subsubsection{Übersicht}
			\begin{itemize}
				\item \textbf{primäre Hirntumore}
					\begin{itemize}
						\item \textbf{neuroepithelialem Gewebe}
						\item \textbf{umgebenden Strukturen}
						\item \textbf{embryologische versprengten Zellen}
					\end{itemize}
				\item \textbf{Gefäßtumore}
				\item \textbf{Metastasen}
			\end{itemize}
		\subsubsection{Symptome}
			\begin{itemize}
				\item \textbf{je nach Lokalisation und Wachstumsgeschwindigkeit}
				\item \textbf{psychopathologische Veränderungen}
				\item \textbf{Kopfschmerzen}
				\item \textbf{erhöhter Hirndruck}
				\item \textbf{epileptische Anfälle}
			\end{itemize}
		\subsubsection{Diagnostik}
			\begin{itemize}
				\item \textbf{Differentialdiagnose: Raumforderung}
					\begin{itemize}
						\item \textbf{intrakranielle Blutung}
						\item \textbf{Entzündungen}
						\item \textbf{Tumoren des Schädelknochens und der Weichteile} 
					\end{itemize}
				\item \textbf{Diagnostik}
					\begin{itemize}
						\item \textbf{Bildgebung}
						\item \textbf{Liquorbefund}
						\item \textbf{ev. Biopsie}
						\item \textbf{Angiographie}
					\end{itemize}
			\end{itemize}
		\subsubsection{Therapie}
			\begin{itemize}
				\item \textbf{Totalresektion}
				\item \textbf{Teilresektion}
				\item \textbf{postoperative Chemo- oder Strahlentherapie}
				\item \textbf{Kortikosteroide}
				\item \textbf{Anlage eines Shunts}
			\end{itemize}
	\subsection{Hirnmetastasen}
		\begin{itemize}
			\item \textbf{meist hämatogen}
			\item \textbf{Primätumor}
				\begin{itemize}
					\item \textbf{bei Männern: Bronchial-Ca}
					\item \textbf{bei Frauen: Mamma-Ca}
					\item \textbf{Hypernephrom}
					\item \textbf{Malignes Melanom}
					\item \textbf{Ca im Gastrointestinaltrakt}
					\item \textbf{maligne Lymphome}
				\end{itemize}
		\end{itemize}
	\subsection{einzelne Hirntumore}
		\subsubsection{Übersicht}
			\begin{itemize}
				\item \textbf{Astrozytom}
					\begin{itemize}
						\item \textbf{pilozytisches Astrozytom}
						\item \textbf{Astrozytom WHO-Grad II}
					\end{itemize}					
				\item \textbf{Glioblastom (Astrozytom Grad 4)}
				\item \textbf{Oligodendrogliom}
				\item \textbf{Hypophysenadenom}
				\item \textbf{Meningeom}
				\item \textbf{Neurinom}
			\end{itemize}
	\subsection{Polyneuropathien}
		\begin{itemize}
			\item \textbf{Erkrankung peripherer Nerven ohne Trauma-Ursache}
			\item \textbf{Einteilung nach der Ursache}
				\begin{itemize}
					\item \textbf{genetische P.}
					\item \textbf{P. bei Stoffwechselstörungen (Diab. mell.)}
					\item \textbf{P. bei Mangel- und Fehlernährung}
					\item \textbf{P. bei Infektionskrankheiten}
					\item \textbf{P. durch Gifte (Alkohol!), u.a.}
				\end{itemize}				
			\item \textbf{Sym}
				\begin{itemize}
					\item \textbf{distal beginnende Sensibilitätsstörungen (Socken- und Handschuhförmig), fehlende Reflexe}
					\item \textbf{trophische Störungen (Muskelatrophie, geringe Schweißsekretion, trockene, glatte Haut, Ulcera)}
					\item \textbf{später auch motorische Ausfälle}
				\end{itemize}
		\end{itemize}