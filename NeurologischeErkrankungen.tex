\section*{Neurologische Erkrankungen}
	\subsection*{Übersicht}
		\begin{itemize}
			\item \textbf{Bewusstseinsstörungen (Übersicht)}
			\item \textbf{Epilepsie}
			\item \textbf{Entzündungen, MS}
			\item \textbf{Morbus Parkinson}
			\item \textbf{cerebrovaskuläre Erkrankungen}
			\item \textbf{Lähmung (Übersicht)}
			\item \textbf{Hirndruck}
			\item \textbf{Demenzen (Übersicht)}
			\item \textbf{Transmissible Spongiforme Enzephalopathie (TSE)}
			\item \textbf{Tumoren}
			\item \textbf{Poyneuropathien}
		\end{itemize}
	\subsection*{Bewusstseinsstörungen}
		\subsubsection*{Benommenheit}
		\subsubsection*{Somnolenz}
			\begin{itemize}
				\item \textbf{schläfrig, apathisch, aber weckbar, bedingt kooperativ}
			\end{itemize}
		\subsubsection*{Sopor}
			\begin{itemize}
				\item \textbf{ähnlich dem Tiefschlaf, nur durch starke Reize (Schmerz) weckbar, gerichtete Abwehr}
			\end{itemize}
		\subsubsection*{Koma}
			\begin{itemize}
				\item \textbf{nicht weckbar, Augen geschlossen, mit Intaktheit der vegetativen Funktionen vereinbar; vier Schweregrade}
			\end{itemize}
	\subsection*{Epilepsie}
		\subsubsection*{Episoden chaotischer elektrischer Entladungen im Gehirn}
			\begin{itemize}
				\item \textbf{können das gesamte Gehirn oder einen umschriebenen Teil betreffen $\rightarrow$ Unteschiede in der Form des Anfalls}
					\begin{itemize}
						\item \textbf{Grand mal Anfälle: tonisch-klonische Krämpfe}
						\item \textbf{Absencen: Patient wirkt "geistig" abweisend}
						\item \textbf{Anfälle mit unkontrollierten Bewegungen einzelner Gliedmaßen, der Patient hat keinerlei Bewusstseinsbeeinträchtigung}
					\end{itemize}
			\end{itemize}
		\subsubsection*{Ursachen}
			\begin{itemize}
				\item \textbf{Gehirnerkrankungen (z.B. Entzündungen, Vergiftungen, Tumore, Kopfverletzungen, Schlaganfall, $\dots$)}
				\item \textbf{Häufigfkeit: ca. 1\% der Bevölkerung}
				\item \textbf{Diagnose mittels EEG}
			\end{itemize}