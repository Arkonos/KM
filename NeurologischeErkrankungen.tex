\section{Neurologische Erkrankungen}
	\subsection{Übersicht}
		\begin{itemize}
			\item \textbf{Bewusstseinsstörungen (Übersicht)}
			\item \textbf{Epilepsie}
			\item \textbf{Entzündungen, MS}
			\item \textbf{Morbus Parkinson}
			\item \textbf{cerebrovaskuläre Erkrankungen}
			\item \textbf{Lähmung (Übersicht)}
			\item \textbf{Hirndruck}
			\item \textbf{Demenzen (Übersicht)}
			\item \textbf{Transmissible Spongiforme Enzephalopathie (TSE)}
			\item \textbf{Tumoren}
			\item \textbf{Poyneuropathien}
		\end{itemize}
	\subsection{Bewusstseinsstörungen}
		\begin{itemize}
			\item \textbf{Benommenheit}
			\item \textbf{Somnolenz}
				\begin{itemize}
					\item \textbf{schläfrig, apathisch, aber weckbar, bedingt kooperativ}
				\end{itemize}
			\item \textbf{Sopor}
				\begin{itemize}
					\item \textbf{ähnlich dem Tiefschlaf, nur durch starke Reize (Schmerz) weckbar, gerichtete Abwehr}
				\end{itemize}
			\item \textbf{Koma}
				\begin{itemize}
					\item \textbf{nicht weckbar, Augen geschlossen, mit Intaktheit der vegetativen Funktionen vereinbar; vier Schweregrade}
				\end{itemize}
		\end{itemize}
		
	\subsection{Epilepsie}
		\begin{itemize}
			\item \textbf{Episoden chaotischer elektrischer Entladungen im Gehirn}
				\begin{itemize}
					\item \textbf{können das gesamte Gehirn oder einen umschriebenen Teil betreffen $\rightarrow$ Unteschiede in der Form des Anfalls}
						\begin{itemize}
							\item \textbf{Grand mal Anfälle: tonisch-klonische Krämpfe}
							\item \textbf{Absencen: Patient wirkt "geistig" abweisend}
							\item \textbf{Anfälle mit unkontrollierten Bewegungen einzelner Gliedmaßen, der Patient hat keinerlei Bewusstseinsbeeinträchtigung}
						\end{itemize}
				\end{itemize}
			\item \textbf{Ursachen}
				\begin{itemize}
					\item \textbf{Gehirnerkrankungen (z.B. Entzündungen, Vergiftungen, Tumore, Kopfverletzungen, Schlaganfall, $\dots$)}
				\end{itemize}
			\item \textbf{Häufigkeit: ca. 1\% der Bevölkerung}
			\item \textbf{Diagnose: mittels EEG}
		\end{itemize}
		
	\subsection{Entzündungen}
		\begin{itemize}
			\item \textbf{Einteilung}
				\begin{itemize}
					\item \textbf{Meningitis}
						\begin{itemize}
							\item \textbf{akute bakterielle Meningitis}
							\item \textbf{akute lymphozytäre Meningitis (=viral)}
							\item \textbf{chronisch lymphozytäre Meningitis}
						\end{itemize}
					\item \textbf{Encephalitis}
					\item \textbf{Hirnabszess}
					\item \textbf{Borreliose}
					\item \textbf{multible Sklerose}
				\end{itemize}
		\end{itemize}
	\subsection{Multible Sklerose}
		\begin{itemize}
			\item \textbf{Enzephalitis disseminata}
			\item \textbf{chronisch-entzündliche ZNS-Entmarkungs-KH}
				\begin{itemize}
					\item \textbf{Zerfall der isolierenden Markscheiden im Rahmen einer Entzündung, herdförmiger Myelinverlust an verschiedenen Stellen des Gehirns und des Rückenmarks}
					\item \textbf{Narbenbildung nach Entzündungsrückgang}
				\end{itemize}
			\item \textbf{Ätiologie - ? autoimmun?, slow-virus?}
			\item \textbf{Beginn häufig zw. 20 - 40 a, mehr Freuen, genetische Disposition}
			\item \textbf{Verlauf: sehr variabel (schwierige Prognose)}
				\begin{itemize}
					\item \textbf{schubförmig}
					\item \textbf{chronisch progredient}
				\end{itemize}
			\item \textbf{Symptome}
				\begin{itemize}
					\item \textbf{Sehstörungen: Sehnervenentzündungen, Doppeltsehen}
					\item \textbf{Sensibilitätsstörungen, Lähmungen, Blasen- und Mastdarmstörungen}
					\item \textbf{Kleinhirnsymptome (Sprachstörungen, Zittern, Koordinationsstörungen)}
					\item \textbf{psychische Veränderungen (Depressionen)}
				\end{itemize}
			\item \textbf{Diagnostik}
				\begin{itemize}
					\item \textbf{klinischer Verlauf}
					\item \textbf{Liquor}
					\item \textbf{Evozierte Potentiale}
					\item \textbf{MRT}
				\end{itemize}
			\item \textbf{Therapie}
				\begin{itemize}
					\item \textbf{Glukokortikoide im Schub}
					\item \textbf{Immunsuppression (Interferone, Azathioprin, $\dots$)}
					\item \textbf{symptomatische Th bei Spastik, Blasenstörungen, $\dots$}
				\end{itemize}
		\end{itemize}
	\subsection{Morbus Parkinson}