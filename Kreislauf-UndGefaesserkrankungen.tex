%\section*{Kreislauf- und Gefäßerkrankungen}
%	\subsection{Kreislauf- und Gefäßerkrankungen}
%		\subsubsection{Übersicht}
%			\begin{itemize}
%				\item \textbf{Ödem}
%				\item \textbf{Thrombose}
%				\item \textbf{Embolie}
%				\item \textbf{Pahtologie der Arterien}
%					\begin{itemize}
%						\item \textbf{Arteriosklerose}
%						\item \textbf{Aneurysma}
%						\item \textbf{pAVK}
%						\item \textbf{akuter Arterienverschluss}
%					\end{itemize}
%				\item \textbf{Pahtologie der Venen}
%					\begin{itemize}
%						\item \textbf{Varizen}
%						\item \textbf{Thrombophlebitis, Phlebothrombose/TVT}
%					\end{itemize}
%				\item \textbf{arterielle Hypertonie}
%				\item \textbf{Schock}
%			\end{itemize}
%	\subsection{Ödem}
%		\begin{itemize}
%			\item \textbf{Definition}
%				\begin{itemize}
%					\item \textbf{Flüssigkeitsansammlung in einem Gewebe}
%				\end{itemize}
%			\item \textbf{Einteilung (Übersicht)}
%				\begin{itemize}
%					\item \textbf{Lymphstauungsödem}
%					\item \textbf{Blutstauungsödem}
%					\item \textbf{Proteinmangelödem (=onkotische Ödeme)}
%					\item \textbf{renale Ödeme}
%					\item \textbf{kapillartoxische Ödeme}
%				\end{itemize}
%			\item \textbf{Lymphstauungsödem}
%				\begin{itemize}
%					\item \textbf{Blockade größerer Lymphgefäße bzw. Lymphknoten}
%					\item \textbf{Ursachen: Tumorerkrankungen und –behandlung, Infektionen}, Entzündung (durch Filarien $\rightarrow$ Elephantiasis)\\
%						früher radikale Lyhmphknotenentfernen zB. bei Mammacarzinom
%				\end{itemize}
%			\item \textbf{Blutstauungsödem}
%				\begin{itemize}
%					\item \textbf{Ursachen}
%						\begin{itemize}
%							\item \textbf{örtliche Behinderung des Blutabflusses}\\
%								' \textbf{venös (Beinvenen)}
%							\item \textbf{kardial bedingte Abflussbehinderung: Herzinsuffizienz (siehe Herzerkrankungen)}
%						\end{itemize}
%				\end{itemize}
%			\item \textbf{Eiweißmangelödeme}
%				\begin{itemize}
%					\item \textbf{Ursachen}
%						\begin{itemize}
%							\item Proteinmangel $\rightarrow$ Aszites (Abgemagert, aber dicker Bauch)
%							\item \textbf{Hunger, Fehlernährung, Eiweißverlust (renal), Eiweißsynthesestörung}
%						\end{itemize}
%				\end{itemize}
%			\item \textbf{renale Ödeme} $\rightarrow$ Augenliedödem
%			\item \textbf{Ödeme durch Schädigung der Kapillarwand}\\
%				zB.: Insektengift
%		\end{itemize}
%	\subsection{Thrombose}
%		\begin{itemize}
%			\item \textbf{Definition}
%				\begin{itemize}
%					\item \textbf{Bildung eines Blutgerinnsels (Thrombus) in einer Vene oder Arterie}, Gerinnungskaskade
%					\item solten sich nach Heilung wieder auflösen
%					\item \textbf{intravitale, intravasale Blutgerinnung}
%					\item \textbf{Folge: teilweise oder vollständige Unterbrechung des Blutflusses}
%				\end{itemize}
%			\item \textbf{Entstehung}
%					\begin{itemize}
%						\item \textbf{Virchow'sche Trias}
%							\begin{itemize}
%								\item \textbf{Gefäßwandfaktor}, Form der Gefäßwand
%								\item \textbf{Zirkulationsfaktor}, zu langsame Zirkulation
%								\item veränderte Blutzusammensetzung, \textbf{Humoralfaktor}, zu viel Zellen, od. Flüssigkeit
%							\end{itemize}
%					\end{itemize}
%		\end{itemize}
%	\subsection{Embolie}
%		\begin{itemize}
%			\item \textbf{Definition}
%				\begin{itemize}
%					\item \textbf{Verschleppung von geformten Elementen (= Embolus, zB } (Thrombus, selten Luft)\textbf{ auf dem Blut - oder Lymphweg}
%				\end{itemize}
%			\item \textbf{Folge}
%					\begin{itemize}
%						\item \textbf{Steckenbleiben in einem Gefäß mit engerer Gefäßlichtung}
%						\item \textbf{Gefäßverschluss}
%					\end{itemize}
%			\item \textbf{Einteilung}
%				\begin{itemize}
%					\item \textbf{nach der Wegrichtung des Embolus in der Strombahn}
%					\item \textbf{nach der benützten Gefäßstrecke (arteriell, venös)}
%					\item \textbf{nach der Art des verschleppten Materials}
%				\end{itemize}
%			\item \textbf{Einteilung nach der benützten Gefäßstrecke}
%				\begin{itemize}
%					\item \textbf{arterielle Embolie in den Körperkreislauf:}
%						\begin{itemize}
%							\item \textbf{Quellen der Embolie: Lungenvenen, linker Vorhof, Mitralklappe, linker Ventrikel, Aortenklappe, Aorta}
%							\item \textbf{häufigste Zielorgane der Embolie: Gehirnarterien, Bauchraumarterien, Arterien der unteren Extremität}
%						\end{itemize}
%					\item \textbf{venöse Embolie in den Lungenkreislauf:}
%						\begin{itemize}
%							\item \textbf{Quellen der Embolie: tiefe Venen der unteren Extremität, Venen des kleinen Beckens, Vena cava inferior, rechter Vorhof}
%							\item \textbf{Zielorgan: Lunge}
%						\end{itemize}
%				\end{itemize}
%		\end{itemize}
%	\subsection{Arterioskelrose}
%		\begin{itemize}
%			\item \textbf{WHO-Definition}
%				\begin{itemize}
%					\item  \textbf{chronisch fortschreitende Arterienerkrankung mit \emph{Wandverhärtung} („Sklerose“) und Einengung der Arterienlichtung durch herdförmige Anhäufung von Fettsubstanzen, Kohlehydraten, Blutbestandteilen, Bindegewebe und Calcium (Plaque)}
%				\end{itemize}
%			\item Ursachen
%				\begin{itemize}
%					\item Cholesterin\\
%					' \textbf{H}igh \textbf{D}ensity \textbf{L}iporotein, schützender Effekt\\
%					' \textbf{L}ow \textbf{D}ensity \textbf{L}ipoprotein, schlechte Cholesterin
%				\end{itemize}
%			\item \textbf{keine Rückbildung, Beginn oft schon in früher Jugend}
%			\item \textbf{Lokalisation}
%				\begin{itemize}
%					\item \textbf{größere elastische und muskuläre Arterien (Aorta, A.carotis, A.iliaca, Hirnarterien, Koronararterien)}
%					\item[$\rightarrow$] Beurteilung der Koronaraterien durch Karotis Ultraschall gibt gute Auskunft über Wandzustand
%				\end{itemize}
%			\item \textbf{Folgen der Atherosklerose}
%				\begin{itemize}
%						\item \textbf{chron. Lichtungseinengung = chron. Stenose} $\rightarrow$ Thrombosbildung
%							\begin{itemize}
%								\item \textbf{Ruhedurchblutung ausreichend, bei Mehrforderung $\rightarrow$ Mangeldurchblutung}
%							\end{itemize}
%						\item \textbf{akute Lichtungseinengung = akute Stenose $\rightarrow$ Infarkt}
%						\item \textbf{Lichtungsverschluss durch Thrombose oder Embolie $\rightarrow$ Infarkt}
%						\item \textbf{Wandschwäche $\rightarrow$ Ausweitung = "Aneurysma"}
%					\end{itemize}
%			\item \textbf{häufige Lokalisation}
%				\begin{itemize}
%					\item \textbf{Aorta (v.a. Bauchaorta)}
%					\item \textbf{Gehirn}
%						\begin{itemize}
%							\item \textbf{Einengung $\rightarrow$ Durchblutungsstörung = "vaskuläre Demenz"}
%							\item \textbf{Verschluss = Infarkt =} Apoplex = \textbf{$\dots$}
%						\end{itemize}
%					\item \textbf{Herz: KHK}
%						\begin{itemize}
%							\item \textbf{Einengung = Angina pectoris}
%							\item \textbf{Verschluss = Myokardinfarkt}
%						\end{itemize}
%					\item \textbf{Niere}
%						\begin{itemize}
%							\item \textbf{Einengung = Durchblutungsstörung $\rightarrow$ Schrumpfniere}
%							\item \textbf{Verschluss = Niereninfarkt}
%						\end{itemize}
%					\item \textbf{Beine}
%						\begin{itemize}
%							\item \textbf{Einengung = Durchblutungsstörung $\rightarrow$ "Schaufensterkrankheit" = pAVK}
%							\item \textbf{Verschluss = Infarkt}
%						\end{itemize}
%				\end{itemize}
%			\item \textbf{Risikofaktoren}
%				\begin{itemize}
%					\item \textbf{Klasse 1}	
%						\begin{itemize}
%							\item \textbf{Hyperlipidämie (Cholesterin – LDL, Triglyceride)}
%							\item \textbf{Hypertonie} schädigt Gefäße $\Leftrightarrow$ steigert Hypertonie
%							\item \textbf{Diabetes mellitus}
%							\item \textbf{Zigarettenkonsum}
%						\end{itemize}
%					\item \textbf{Klasse 2}
%						\begin{itemize}
%							\item \textbf{Adipositas}
%							\item \textbf{Bewegungsmangel}
%							\item \textbf{Stress}\\
%								' Eustress, befähigender Stress\\
%								' Dysstress, schädigender Dauerstress
%						\end{itemize}
%					\item \textbf{unbeeinflussbare Faktoren}
%						\begin{itemize}
%							\item \textbf{Lebensalter}, erste Welt immer jünger
%							\item \textbf{Geschlecht (Östrogenschutz!)}
%							\item \textbf{familiäre Häufung, genetische Faktoren}
%						\end{itemize}
%				\end{itemize}
%			\end{itemize}
	\subsection{Aneurysma}
		\begin{itemize}
			\item \textbf{Definition}
				\begin{itemize}
					\item \textbf{lokalisierte Ausweitung einer Arterie durch}	
						\begin{itemize}
							\item \textbf{angeborene Wandschwäche}\\
								\textbf{Gefäß hält dem RR nicht Stand $\rightarrow$ Aussackung (z.B. Hirnbasisgefäße) $\rightarrow$ Ruptur, letale Blutung}
							\item \textbf{erworbene (atheriosklerotische) Wandschwäche}\\
								\textbf{durch schwere arteriosklerotische Wandschädigung, meist in der Bauchaorta}
						\end{itemize}
				\end{itemize}
			\item \textbf{Folgen eines Aneurysmas}
				\begin{itemize}
					\item \textbf{Thrombose}
						\begin{itemize}
							\item \textbf{Durchblutungsstörung}
							\item \textbf{Emboliegefahr}
						\end{itemize}
					\item \textbf{Kompression}
						\begin{itemize}
							\item \textbf{Druckatrophie von Nachbarorganen}
						\end{itemize}
					\item \textbf{Perforation = Ruptur}
						\begin{itemize}
							\item \textbf{ev. tödliche Blutung}
						\end{itemize}
				\end{itemize}
		\end{itemize}
	\subsection{pAVK (periphäre arterielle Verschlusskrankheit}
		\begin{itemize}
			\item \textbf{Definition}
				\begin{itemize}
					\item \textbf{Einengung der Extremitätenarterien (meist Beine)}
				\end{itemize}
			\item \textbf{Ätiologie}
				\begin{itemize}
					\item \textbf{Arteriosklerose (Risikofaktoren!)}
			\end{itemize}
			\item \textbf{Folge}
				\begin{itemize}
					\item \textbf{Durchblutungsstörung der Extremitäten}
				\end{itemize}
			\item \textbf{Einteilung in Schweregrade}
				\begin{itemize}
					\item zunehmend kürzer werdende \textbf{schmerzfreie Gehstrecke}
				\end{itemize}
			\item Schaufensterkrankheit: Schmerz zwingt zu Pausen
			\item Raucherbein
			\item ultimativ: Amputation
		\end{itemize}
	\subsection{akuter Arterienverschluss}
		\begin{itemize}
			\item \textbf{Definition}
				\begin{itemize}
					\item \textbf{plötzlich auftretender arterieller Durchbltgs.-Stop}
					\item \textbf{80\% Beine betroffen}
				\end{itemize}
			\item \textbf{Ätiologie}
				\begin{itemize}
					\item \textbf{80\% Thrombo-Embolien, davon 90\% kardial}
					\item \textbf{lokale Thrombose (pAVK)}
			\end{itemize}
			\item \textbf{Symptome}
				\begin{itemize}
					\item \textbf{Schmerz, Blässe, Pulslosigkeit, Lähmung, Schwäche, Kältegefühl, ev. Schock}
				\end{itemize}
			\item \textbf{Diagnostik}
				\begin{itemize}
					\item \textbf{klinisches Bild}
					\item \textbf{Gefäßdarstellung}
					\item Ultraschall, Dopplerschall
					\item kontrastmittel
					\item (Fuß-) Pulse
				\end{itemize}
			\item \textbf{Therapie}
				\begin{itemize}
					\item \textbf{Thrombolyse}\\
						Blutgerinnungsmittel bei frischen Thromben
					\item \textbf{Rekanalisation}
						\begin{itemize}
							\item \textbf{Thrombo-/Embolektomie}
						\end{itemize}
					\item \textbf{ultima ratio: Amputation}
					\item \textbf{Rezidiv-Prophylaxe durch Antikoagulation}
				\end{itemize}
		\end{itemize}
	\subsection{Pathologie der Venen Varizen}
		\begin{itemize}
			\item \textbf{Varicosis}
				\begin{itemize}
					\item \textbf{Ausbuchtungen einer geschädigten Venenwand}
				\end{itemize}
			\item \textbf{Ursache}
				\begin{itemize}
					\item \textbf{Missverhältnis zwischen Wandstärke und intravenösem Druck}
					\item \textbf{Wandschwäche}
						\begin{itemize}
							\item \textbf{angeboren-konstitutionell}
							\item \textbf{erworben}
						\end{itemize}
					\item \textbf{Blutstauung und Druckerhöhung}
						\begin{itemize}
							\item \textbf{kardial bedingte venöse Stauung}
							\item \textbf{langes Stitzen btw. Stehen}
							\item \textbf{Adipositas}
							\item \textbf{Abflussbehinderungen}
						\end{itemize}
			\end{itemize}
		\end{itemize}
	\subsection{Varizen}
		\begin{itemize}
			\item \textbf{allgemeine Folgen der Varizen}
				\begin{itemize}
					\item \textbf{Durchblutungsstörung infolge langsamer Blutströmung}
					\item \textbf{Thrombose und Embolie}
					\item \textbf{Thrombophlebitis / Phlebothrombose}
					\item \textbf{Ruptur mit Blutung (Ösophagus!)}
				\end{itemize}
			\item \textbf{mögliche Spätfolgen an den Beinen}
				\begin{itemize}
					\item \textbf{Ulcus cruris}
					\item \textbf{postthrombotisches Syndrom (chronisch-venöse Insuffizienz)}
			\end{itemize}
		\end{itemize}
	\subsection{entzündliche venöse Gefäßerkrankungen}
		\begin{itemize}
			\item \textbf{Thrombophlebitis / Phlebothrombose (TVT)}
				\begin{itemize}
					\item \textbf{Thromben $\rightarrow$ Entzündung der Venenwand}
					\item \textbf{Venenwandentzündung $\rightarrow$ Thrombusbildung}
					\item \textbf{Lokalisation. v.a. untere Extremität}
					\item \textbf{hohes Embolierisiko bei TVT!!}
					\item \textbf{Risikofaktoren Phlebothrombose}
						\begin{itemize}
							\item \textbf{Strömungsverlangsamung}
							\item \textbf{Endothelschäden}
							\item \textbf{Hyperkoagulobilität}
						\end{itemize}
				\end{itemize}
			\item \textbf{Thrombophlebitis}
				\begin{itemize}
					\item \textbf{oberflächliche Venen betroffen}
					\item \textbf{Therapie: lokale Maßnahmen, Bewegung}
				\end{itemize}
			\item \textbf{Phlebothrombose}
				\begin{itemize}
					\item \textbf{tiefe Venen betroffen}
					\item \textbf{Therapie: Bettruhe, Antikoagulation, Thrombolyse oder Thrombektomie}
				\end{itemize}
			\item \textbf{Diagnostik}
				\begin{itemize}
					\item \textbf{Druckschmerzpunkte}
					\item \textbf{Gefäßdarstellung}
						\begin{itemize}
							\item \textbf{Doppler-Sonographie}
							\item \textbf{Phlebographie}
						\end{itemize}
				\end{itemize}
		\end{itemize}
	\subsection{Hypertonie}
		\begin{itemize}
			\item \textbf{RR-Erhöhung über den Normwert im}
				\begin{itemize}
					\item \textbf{großen Kreislauf (Körperkreislauf) = arterielle Hypertonie}
					\item \textbf{kleinen Kreislauf (Lungenkreislauf) = pulmonale Hypertonie}
				\end{itemize}
			\item \textbf{arterielle Hypertonie – Epidemiologie}
				\begin{itemize}
					\item \textbf{gehört zu den häufigsten Erkrankungen}
					\item \textbf{Risikofaktor erster Ordnung für Atherosklerose und ihre Folgeschäden (Gehirn, Herz, Nieren)}
				\end{itemize}
		\end{itemize}
	\subsection{arterielle Hypertonie}
		\begin{itemize}
			\item \textbf{physiologische / pathologische Werte}
				\begin{itemize}
					\item \textbf{WHO: über 140/90mmHg$\dots$}
					\item \textbf{Klassifikation nach dt. Hochdruckliga}
						\begin{itemize}
							\item \textbf{optimal}
							\item \textbf{normal}
							\item \textbf{hochnormal (Grenzwerthypertonie)}
						\end{itemize}
					\item \textbf{pathologische Werte (Hypertonie) ab}
						\begin{itemize}
							\item \textbf{Stufe 1 (leicht)}
							\item \textbf{Stufe 2 (mittelschwer)}
							\item \textbf{Stufe 3 (schwer)}
						\end{itemize}
					\item \textbf{Einteilung nach der Ätiologie in}
						\begin{itemize}
							\item \textbf{prümare (="essentielle") Hypertonie}\\
								' \textbf{90\%-95\%}\\
								' \textbf{Enstehung weitgehend ungelärt}\\
								' \textbf{multifaktoriell, "Wohlstandserkrankung"}\\
								$\mbox{}\qquad$ \textbf{erhöhter peripherer Gefäßwiderstand, erhöhtes HMV, Kochsalzkonsum,}\linebreak
								$\mbox{}\qquad\:$ \textbf{Sympathikus, RAAS, renale Faktoren, vaskuläre Faktoren, Umweltfaktoren,}\linebreak
								$\mbox{}\qquad\:$ \textbf{Adipositas,$\dots$}
							\item \textbf{sekundäre (= organgebundene) Hypertonie}\\
								' \textbf{renale Hypertonie, endokrine Hypertonie, kardiovaskuläre Hypertonie,$\dots$}
						\end{itemize}
					\item \textbf{Folgen der chronischen Hypertonie}
						\begin{itemize}
							\item \textbf{kardiale Schäden}
								\begin{itemize}
									\item \textbf{Linksherzhypertrophie, LinksherzinsuffizienzLinksherzhypertrophie, Linksherzinsuffizienz}
								\end{itemize}
							\item \textbf{frühzeitige Arteriosklerose}
								\begin{itemize}
									\item \textbf{Koronargefäße: KHK}
									\item \textbf{Arterien: Elastizitätsverlust, pAVK, Aortenaneurysma}
									\item \textbf{Gehirn: zerebrale Ischämie, Infarkt, Gefäßruptur, SAB}
									\item \textbf{Nieren: Nephrosklerose, Niereninsuffizienz, Urämie}
								\end{itemize}
						\end{itemize}
					\item \textbf{Symptome}
						\begin{itemize}
							\item \textbf{wenig}
							\item \textbf{Kopfschmerz, Kopfdruck, Ohrensausen, Schwindel, ev. Nasenbluten}
						\end{itemize}
					\item \textbf{Therapie}
						\begin{itemize}
							\item \textbf{Antihypertonika, Ziel: RR $<$ 140/90 mm Hg, altersangepasst}
						\end{itemize}
				\end{itemize}
		\end{itemize}
	\subsection{Hypertonie-TH}
		\begin{itemize}
			\item \textbf{Diuretika}
			\item \textbf{$\beta$-Blocker}
			\item \textbf{Kalzium-Antagonisten}
			\item \textbf{ACE-Hemmer}
			\item \textbf{Sympathikolytika}
			\item \textbf{Angiotensin II-Rezeptorantagonisten}
			\item \textbf{arterioläre Vasodilatatoren}
		\end{itemize}
	\subsection{Schock}
		\begin{itemize}
			\item \textbf{Definition}
				\begin{itemize}
					\item \textbf{akute Minderdurchblutung lebenswichtiger Organe (O$_2$-Mangel)}
				\end{itemize}
			\item \textbf{Ursachen}
				\begin{itemize}
					\item \textbf{peripher: ungenügender venöser Rückstrom zum Herzen}
						\begin{itemize}
							\item \textbf{Blutverlust: nach außen oder nach innen}
							\item \textbf{Blut versackt in erweiterten Kapillaren und Venolen}
							\item \textbf{Flüssigkeitsverlust nach außen oder nach innen (Plasma)}
						\end{itemize}
					\item \textbf{kardial: ungenügendes Auswurfvolumen des Herzens}
				\end{itemize}
			\item \textbf{Stadium 1: Zentralisation}
				\begin{itemize}
					\item \textbf{Kontraktion der peripheren Arteriolen (zB. Haut) als Reaktion auf das verminderte zirkulierende Blutvolumen $\rightarrow$ Blutdruck wird aufrechterhalten $\rightarrow$ Versorgung lebenswichtiger Organe}
				\end{itemize}
			\item \textbf{Stadium 2: Dezentralisation}
				\begin{itemize}
					\item \textbf{Weitstellen der Gefäße in der Peripherie, Blutdruckabfall mit Mangelversorgung lebenswichtiger Organe, zunehmende Sauerstoffnot}
				\end{itemize}
			\item \textbf{Stadium 3: irreversibler Schock}
				\begin{itemize}
					\item \textbf{schwere Organschäden an Gehirn, Herz, Lungen, Leber, Niere}
				\end{itemize}
			\item \textbf{Schockformen nach klinischen Ursachen:}
				\begin{itemize}
					\item \textbf{kardiogener Schock}
					\item \textbf{Blutungsschock (hypovolämischer Schock)}
					\item \textbf{allergischer (anaphylaktischer) Schock}
					\item \textbf{traumatischer Schock}
					\item \textbf{Verbrennungsschock}
					\item \textbf{septischer Schock}
					\item \textbf{$\dots$}
				\end{itemize}
		\end{itemize}
		
\section{Herzerkrankungen}
	\subsection{Übersicht}
		\begin{itemize}
			\item \textbf{kardiale Überlastung: Herzhypertrophie}
			\item \textbf{Herzinsuffizienz}
			\item \textbf{Erkrankungen des Reizleitungssystems: Rhythmusstörungen}
			\item \textbf{entzündliche Herzerkrankungen: Endokarditis, Myokarditis, Perikarditis}
			\item \textbf{koronare Herzkrankheiten: KHK}
				\begin{itemize}
					\item \textbf{Angina pectoris}
					\item \textbf{Myokardinfarkt}
				\end{itemize}
			\item \textbf{Klappenvitien}
		\end{itemize}
	\subsection{Herzerkrankungen}
		\subsubsection{Grundformen der kardialen Überlastung}
			\begin{itemize}
				\item \textbf{chronische Druckbelastung}
				\item \textbf{chronische Volumenbelastung}
				\item \textbf{Folge: Adaptation der Ventrikel  $\rightarrow$  Hypertrophie, ab kritischem Herzgewicht: Hyperplasie  $\rightarrow$ Ventrikeldilatation  $\rightarrow$  enddiastolisches Volumen $\uparrow$ $\rightarrow$  zunehmende Herzinsuffizienz und Koronarinsuffizienz (durch Missverhältnis O$_2$-Bedarf und  O$_2$-Angebot) }
			\end{itemize}
	\subsection{Herzinsuffizienz}
		\subsubsection{Def}
			\begin{itemize}
				\item \textbf{durch unzureichendes syst. Auswurfvolumen oder mangelhafte ventrikuläre Füllung $\rightarrow$}
				\item \textbf{Missverhältnis zwischen Pumpleistung (geförderter Auswurfmenge) des Herzens und Blutbedarf der Körpergewebe}
			\end{itemize}
		\subsubsection{Einteilung}
			\begin{itemize}
				\item \textbf{akut oder chronisch}
				\item \textbf{den li, den re, oder beide Ventrikel betreffend}
				\item \textbf{in klinische Schweregrade nach der NYHA}
			\end{itemize}
		\subsubsection{Ätiologie}
			\begin{itemize}
				\item \textbf{Hypertonie}
				\item \textbf{Herzerkrankungen}
					\begin{itemize}
						\item \textbf{KHK, Klappenfehler, Rhythmusstörungen, $\dots$}
					\end{itemize}
			\end{itemize}
		\subsubsection{Klinik}
			\begin{itemize}
				\item \textbf{"Rückwertsversagen": Blutstauung vor der infuffizienten Herzhälfte}
				\item \textbf{"Vorwärtsversagen":Ö nachlassende Pumpfunktion $\rightarrow$ Unterversorgung der Organe mit O$_2$ und Nährstoffen}
			\end{itemize}
		\subsubsection{Leitsymptome Linksherzinsuffizienz}
			\begin{itemize}
				\item \textbf{Rückwärtsversagen}
					\begin{itemize}
						\item \textbf{Lungenstauung, Dyspnoe, Stauungsbronchitis, Lungenödem, feuchte Rasselgeräusche über der Lunge, Zyanose}
						\item \textbf{chronisch: Rechtsherzüberlastung mit Hypertrophie, "Corpulmonale"}
					\end{itemize}
				\item \textbf{akutes Vorwärtsversagen: kardiogener Schock}
				\item \textbf{morphologisch: Linksherdilitation mit runbogiger Herzspitze}
			\end{itemize}
		\subsubsection{Leitsymptome Rechtsherzinsuffizienz}
			\textbf{$\rightarrow$ Rückstau des Blutes im gesamten Venensystem des großen Kreislaufs:}
			\begin{itemize}
				\item \textbf{gestaute Halvene}
				\item \textbf{Stauung im Bauchraum, Aszites, Hepatomegalie}
				\item \textbf{Knöchelödeme}
				\item \textbf{Gewichtszunahme}
			\end{itemize}
		\subsubsection{Begleitsymptome}
			\begin{itemize}
				\item \textbf{Schwäche, Müdigkeit, Leistungsabfall}
				\item \textbf{Nykturie}
				\item \textbf{tachykarde Herztythmusstörungen (Vorhofflimmern)}
			\end{itemize}
		\subsubsection{"Globalinsuffizienz"}		
		\subsubsection{Diagnostik}
			\begin{itemize}
				\item \textbf{Anamnese}
				\item \textbf{EKG, Herz-Ultraschall (Echokardiographie)}
				\item \textbf{bildgebende Diagnostik: MRT, CT, Thorax-Röntgen}
			\end{itemize}
		\subsubsection{pharmakologische Therapie}
			\begin{itemize}
				\item \textbf{Herz-Belastung senken: z.B. RR-Senkung}
				\item \textbf{Steigerung der Herzkraft und Auswurfleistung}
			\end{itemize}
	\subsection{Herzrhythmusstörungen}
		\subsubsection{Definition}
			\begin{itemize}
				\item \textbf{Störung der Herzfrequenz/der Rhythmik}
			\end{itemize}
		\subsubsection{Einteilung}
			\begin{itemize}
				\item \textbf{Reizbildungsstörung}
				\item \textbf{Reizleitungsstörung}
				\item \textbf{nach der Frequenz}
					\begin{itemize}
						\item \textbf{bradykarde Rhythmusstörungen (<60/min)}
						\item \textbf{tachykarde Rhythmusstörungen (>100/min)}
							\begin{itemize}
								\item \textbf{SA-Block}
								\item \textbf{AV-Block (I. - III.Grades}
								\item \textbf{Extrasystolen}
								\item \textbf{Vorhofflattern, Vorhofflimmern}
								\item \textbf{Kammerflattern, Kammerflimmern}								
							\end{itemize}
					\end{itemize}
			\end{itemize}
		\subsubsection{Ätiologie}
			\begin{itemize}
				\item \textbf{kardial}
				\item \textbf{extrakardial}
			\end{itemize}
		\subsubsection{Symptome}
			\begin{itemize}
				\item \textbf{Beeinträchtigung der Auswufleistung}
				\item \textbf{Herzklopfen, Herzstolpern}
				\item \textbf{RR-Abfall, Schwindel}
				\item \textbf{Kurzatmigkeit, Schweißausbruch, Beklemmungsgefühle, Angst}
			\end{itemize}
		\subsubsection{Diagnostik}
			\begin{itemize}
				\item \textbf{EKG}
			\end{itemize}
		\subsubsection{Therapie}
			\begin{itemize}
				\item \textbf{medik.: Antiarrhythmika, Schrittmacher}
			\end{itemize}
	\subsection{entzündliche Herzerkrankungen}
		\subsubsection{Einteilung nach der Ursache}
			\begin{itemize}
				\item \textbf{Endokarditis}
				\item \textbf{Myokarditis}
				\item \textbf{Perikarditis}
			\end{itemize}
		\subsubsection{Endokarditis}
			\begin{itemize}
				\item \textbf{Entzündung der Klappen}
				\item \textbf{Störung der hämodynamischen Klappenfunktion}
				\item \textbf{bevorzugt li-Herz Klappen}
				\item \textbf{nicht infektiös}
					\begin{itemize}
						\item \textbf{Endocarditis verrucosa rheumatica}
					\end{itemize}
				\item \textbf{infektiös}
					\begin{itemize}
						\item \textbf{akute infektiöse Endokarditis}
						\item \textbf{subakute infektiöse Endokarditis}
					\end{itemize}
				\item \textbf{Komplikationen}
					\begin{itemize}
						\item \textbf{Klappeninsuffizienz}
						\item \textbf{septischer Schock}
					\end{itemize}
			\end{itemize}
	\subsection{KHK}
		\subsubsection{Definition}
			\begin{itemize}
				\item \textbf{Verengung der Koronararterien (Stenose)}
				\item \textbf{dadurch: Missverhältnis zwischen O$_2$-Bedarf des Myokards und O$_2$-Angebot}
				\item \textbf{vier Koronaraterienäste}
					\begin{itemize}
						\item \textbf{RCA}
						\item \textbf{LCA}
						\item \textbf{RIVA}
						\item \textbf{RCX}
					\end{itemize}
			\end{itemize}
		\subsubsection{Ätiologie}
			\begin{itemize}
				\item \textbf{Arteriensklerose der Koronaraterien}
					\begin{itemize}
						\item \textbf{Risikofaktoren:$\dots$}
					\end{itemize}
			\end{itemize}
	\subsection{Angina pectoris}
		\subsubsection{Letisymtom}
			\begin{itemize}
				\item \textbf{retrosternaler oder linksthorakaler Schmerz/Druckgefühl}
				\item \textbf{Ausstrahlung in $\dots$}
			\end{itemize}
		\subsubsection{Einteilung}
			\begin{itemize}
				\item \textbf{stabile A.p.}
				\item \textbf{instabile A.p.}
			\end{itemize}
		\subsubsection{Diagnostik}
			\begin{itemize}
				\item \textbf{Anamnese}
				\item \textbf{Labor: herzspezifische Enzyme}
				\item \textbf{EKG}
				\item \textbf{Bildgebung}
				\item \textbf{Herzkatheteruntersuchung}
			\end{itemize}
		\subsubsection{Therapie}
			\begin{itemize}
				\item \textbf{medikamentös}
				\item \textbf{PTCA}
			\end{itemize}
	\subsection{Myokardinfarkt}
		\subsubsection{Definition}
			\begin{itemize}
				\item \textbf{akuter Koronararterienast-Verschluss}
				\item \textbf{Folge: Nekrose}
			\end{itemize}
		\subsubsection{Einteilung}
			\begin{itemize}
				\item \textbf{fast immer linke Herzhälfte betroffen}
				\item \textbf{nach Lokalisation}
					\begin{itemize}
						\item \textbf{Vorderwand, Seitenwand, Hinterwand}
					\end{itemize}
				\item \textbf{nach Infarkttiefe in der Kammerwand}
					\begin{itemize}
						\item \textbf{Innenschichtinfrarkt, transmuraler Infarkt}
					\end{itemize}
				\item \textbf{kaum Regeneration $\rightarrow$ Belastung des restlichen Gewebes $\rightarrow$ kompensatorische Hypertrophie}
			\end{itemize}
		\subsubsection{Symptome}
			\begin{itemize}
				\item \textbf{Leitsymptome (pektaginöser Schmerz)}
				\item \textbf{vegetative Begleitsymptome}
				\item \textbf{RR $\downarrow$, Herzfrequenz $\uparrow$($\rightarrow$ kardiogener Schock!)}
			\end{itemize}
		\subsubsection{Diagnostik}
			\begin{itemize}
				\item \textbf{Anamnese}
				\item \textbf{Diagnosekriterien (WHO)}
					\begin{itemize}
						\item \textbf{akuter Brustschmerz > 20 min}
						\item \textbf{typische EKG-Veränderungen (STEMI, NON-STEMI)}
						\item \textbf{erhöhte Serumwerte der Herzmarker-Enzyme}
					\end{itemize}
				\item \textbf{Echokardiographie}
				\item \textbf{Koronarangiographie}
			\end{itemize}
		\subsubsection{Therapie}
			\begin{itemize}
			\item \textbf{MONA: Morphium, O$_2$ Nitrate, ASS}
			\item \textbf{Blutverdünnung}
			\item \textbf{frühestmögliche Reperfusion = Blutfluss wiederherstellen}
				\begin{itemize}
					\item \textbf{Auflösen des Gerinnsels mittels (Thrombolyse)}
					\item \textbf{PTCA}
					\item \textbf{Bypass-OP}
				\end{itemize}
			\end{itemize}
		\subsubsection{mögliche Komplikationen}
			\begin{itemize}
				\item \textbf{kardiogener Schock}
				\item \textbf{Papillarmuskelabriss}
				\item \textbf{Herzwandaneurysma, Herzwandruptur}
				\item \textbf{Reinfarkt, $\dots$}
			\end{itemize}
	\subsection{Klappenventitien}
		\subsubsection{Einteilung}
			\begin{itemize}
				\item \textbf{angeboren oder erworben (Endokarditis!)}
				\item \textbf{Klaüüenstenose oder Klappeninsuffizienz}
					\begin{itemize}
						\item \textbf{Mitralklappenstenose}
						\item \textbf{Mitralklappeninsuffizienz}
						\item \textbf{Aortenklappenstenose}
						\item \textbf{Aortenklappeninsuffizienz} 
					\end{itemize}
			\end{itemize}