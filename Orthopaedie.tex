\section{Orthopädie}
\subsection{Inhaltsübersicht Orthopädie}
	\begin{itemize}
		\item \textbf{Gelenkserkrankungen}
			\begin{itemize}
				\item \textbf{degenerative Erkrankungen (Arthrose)}
				\item \textbf{entzündliche Erkrankungen (Arthritiden), "Rheuma"}
				\item \textbf{metabolische Arthropathien (Gicht, u.a.)}
			\end{itemize}
		\item \textbf{Knochenerkrankungen}
			\begin{itemize}
				\item \textbf{Knocheninfektionen, M.Paget, Osteoporose, Rachitis und Osteomalazie}
			\end{itemize}
		\item \textbf{orthopädische Onkologie}
		\item \textbf{Fehlbildungen und Entwicklungsstörungen}
		\item \textbf{Erkrankungen von Muskeln, Sehnen, Bändern, Menisken, Bursen} (Schleimbeutel)
		\item \textbf{spezielle Orthopädie der Wirbelsäule und der Extremitäten}
 	\end{itemize}

\subsection{Übersicht Gelenkerkrankungen}
	\begin{itemize}
		\item \textbf{degenerative Gelenkerkrankungen}
		\item \textbf{entzündlich-rheumatische Gelenkerkrankungen}
		\item \textbf{metabolische Gelenkerkrankungen}
 	\end{itemize}
\subsection{degenerative Gelenkerkrankungen}
	\subsubsection{Übersicht}
		\begin{itemize}
			\item \textbf{Knorpel-Physiologie}
			\item \textbf{Arthrose}
				\begin{itemize}
					\item \textbf{Coxarthrose}
					\item \textbf{Endoprothetik}
					\item \textbf{Knorpelchirurgie}
					\item \textbf{Osteochondrosis dissecans}
		 		\end{itemize}
		\end{itemize}
	\subsubsection{Knorpel - Physiologie}
		\begin{itemize}
			\item \textbf{Knorpelaufbau}
				\begin{itemize}
					\item \textbf{60-80\% Wasser, Chondrozyten, Kollagen, Proteoglykane, nicht kollagene Proteine}
				\end{itemize}
			\item \textbf{Chondrozyten entstehen aus mesenchymalen Stammzellen, zeigen nur während des Wachstums mitotische Aktivität}
			\item \textbf{Knorpelgewebe: keine Gefäß-, Nervenversorgung, alymphatisch, Ernährung über Diffusion}
			\item \textbf{meist keine Restitutio ad integrum im Knorpelgewebe möglich (nur Grad 1)}
			\item an gewichtstragenden Gelenken Abnutzungserscheinungen (Grad 1-4)
	 	\end{itemize}
	\subsubsection{Knorpeldefekte}
		\begin{itemize}
			\item \textbf{Einteilung von Knorpeldefekten}
				\begin{itemize}
					\item \textbf{Grad 1: Oberfläche unauffällig, Knorpel aufgeweicht}
					\item \textbf{Grad 2: oberflächliche Knorpeldefekte bzw. Ausfransungen}
					\item \textbf{Grad 3: Defekte bis zum darunterliegenden Knochen reichend}
					\item \textbf{Grad 4: Knochen liegt frei}
				\end{itemize}
	  	\end{itemize}
	\subsubsection{Arthrose}
		\begin{itemize}
			\item \textbf{Arthrosis deformans}
			\item \textbf{Definition}
				\begin{itemize}
					\item \textbf{primär nicht entzündliche, degenerative und irreversible Gelenkzerstörung}
					\item \textbf{verursacht durch ein Missverhältnis zwischen Belastung und Belastbarkeit}
				\end{itemize}
			\item \textbf{Einteilung}
				\begin{itemize}
					\item \textbf{primäre Arthrose, sekundäre Arthrose}
				\end{itemize}
			\item \textbf{Einteilung nach der Anzahl betroffener Gelenke}
				\begin{itemize}
					\item \textbf{Monarthrosen, Polyarthrosen}
				\end{itemize}
			\item \textbf{Bezeichnung nach betroffenem Gelenk}
				\begin{itemize}
					\item Cox- ()
					\item Gon- ()
					\item Om- ()
					\item Spndylathrose (Wirbelsäule)
				\end{itemize}
			\item \textbf{pathologischer Verlauf}
				\begin{itemize}
					\item \textbf{typisches Nebeneinander destruktiver und proliferativer Veränderungen}
						\begin{itemize}
							\item \textbf{zunächst auf den Gelenkknorpel beschränkt}
							\item \textbf{punktuelle Auffaserung}
							\item \textbf{im weiteren Verlauf flächenhafte Abtragung mit Freilegung des Knochens}
							\item \textbf{Bildung einer "Knochenglatze"}
							\item \textbf{Gelenkkapsel: Fibrosierung und Schrumpfung}
							\item \textbf{periartikuläres Muskelgewebe: Hartspann, Verkürzung, Atrophie}
						\end{itemize}
				\end{itemize}
			\item \textbf{Klinik}
				\begin{itemize}
					\item \textbf{meist jahrelanges symptomfreies Intervall}
					\item \textbf{Bildgebung lässt keine Rückschlüsse auf Beschwerdesymptomatik und klinischen Befund zu}
					\item \textbf{Symptomenbeginn}
						\begin{itemize}
							\item \textbf{Steifigkeit}
							\item \textbf{diffuse Gelenk- und Muskelschmerzen}
						\end{itemize}
					\item \textbf{im weiteren Verlauf}
						\begin{itemize}
							\item \textbf{Anlaufschmerz}
							\item \textbf{belastungs- und bewegungsabhängiger Schmerz}
							\item \textbf{keine Schmerzen in der Nacht}
							\item \textbf{Dauer-, Ruhe- und Nachtschmerz durch begleitende Synovitis} (Gelenkskapselentzündung)
							\item \textbf{Funktionsverlust, Bewegungseinschränkungen, Deformierungen}
						\end{itemize}
				\end{itemize}
			\item \textbf{Diagnostik}
				\begin{itemize}
					\item \textbf{Anamnese}
					\item \textbf{klinische Untersuchung}
					\item \textbf{Röntgenaufnahmen in zwei Ebenen}
					\item \textbf{typische radiologische Veränderungen, zB.:}
						\begin{itemize}
							\item \textbf{Gelenkspaltverschmälerung}
							\item \textbf{osteophytäre Anbauten} (Auswachsen der Randzacken aus Knochengewebe)
						\end{itemize}
				\end{itemize}
		\pagebreak
			\item \textbf{Therapie}
				\begin{itemize}
					\item \textbf{abhängig vom Leidensdruck des Patienten}
						\begin{itemize}
							\item \textbf{allgemeine Maßnahmen}
								\begin{itemize}
									\item \textbf{Vermeidung und Reduktion gelenkbelastender Tätigkeiten}
									\item \textbf{Gewichtsreduktion}
									\item \textbf{Gymnastik}
									\item Corteson ins Gelenk
								\end{itemize}
							\item \textbf{Physiotherapie}
								\begin{itemize}
									\item \textbf{Beseitigung und Vermeidung störender Kontrakturen und Spannungen}
									\item \textbf{Schmerzlinderung}
								\end{itemize}
							\item \textbf{medikamentöse Therapie}
								\begin{itemize}
									\item \textbf{Analgetika}
									\item \textbf{intraartikuläre Glukokortikoidinjektionen}
								\end{itemize}
							\item \textbf{operative Therapie}
								\begin{itemize}
									\item \textbf{Indikation bei erheblichem Leidensdruck und erfolglosen konservativen Maßnahmen}
									\item \textbf{gelenkerhaltende OP: nur jüngere Patienten}
									\item \textbf{gelenkersetzende OP: Hüfte und Knie}
					  			\end{itemize}
						\end{itemize}
	  			\end{itemize}
	  	\end{itemize}
	\subsubsection{Arthrose - Endoprothetik}
		\begin{itemize}
			\item \textbf{Hüftendoprothetik}
				\begin{itemize}
					\item \textbf{Indikationen}
						\begin{itemize}
							\item \textbf{primäre Coxarthrose}
							\item \textbf{Traumen} (zB.: nach Oberschenkelhalsfraktur)
						\end{itemize}
					\item \textbf{OP}
						\begin{itemize}
							\item \textbf{zementfrei / zementiert / kombiniert}
						\end{itemize}
					\item \textbf{Komplikationen}
						\begin{itemize}
							\item \textbf{Femurschaftfrakturen}
							\item \textbf{Absprengungen des Trochanter major}
							\item \textbf{Infektionen der Prothese}
							\item \textbf{Prothesenluxationen} (Kopf springt aus der Pfanne)
						\end{itemize}
				\end{itemize}
			\item \textbf{Knieendoprothetik}
				\begin{itemize}
					\item \textbf{Indikation}
						\begin{itemize}
							\item \textbf{eingeschränkte oder aufgehobene schmerzfreie Gehstrecke}
						\end{itemize}
					\item \textbf{Pangonarthrose $\rightarrow$ bikondylärer Oberflächenersatz}
					\item \textbf{mediale Gonarthrose $\rightarrow$ unikondylärer Oberflächenersatz}
					\item \textbf{zementfrei oder zementiert verankert}
				\end{itemize}
	 	\end{itemize}
		

\subsection{rheumatoide Arthritis}
	\begin{itemize}
		\item \textbf{synonym: chronische Polyarthritis}
		\item \textbf{chronisch-entzündlich-systemische Erkrankung}
		\item \textbf{Ätiologie und Pathogenese}
			\begin{itemize}
				\item \textbf{genetische Disposition plus virale- oder bakterielle Infektion plus $\dots$?}
					\begin{itemize}
						\item[$\rightarrow$] \textbf{chronisch / rezidivierende Immunantwort}
						\item[$\rightarrow$] \textbf{aggressive Synovialitis}
						\item[$\rightarrow$] \textbf{Knorpel- und vollständige Gelenkdestruktion}
					\end{itemize}
			\end{itemize}
		\item \textbf{Klinik}
			\begin{itemize}
				\item Ruheschmerz wegen Entzündung
				\item In Schüben
				\item Betroffen: Hand- Fingergelenke, aber auch andere
				\item kann auch andere Organe betreffen
			\end{itemize}
	\pagebreak
		\item \textbf{Diagnostik}
			\begin{itemize}
				\item \textbf{Rheumafaktoren}
					\begin{itemize}
						\item \textbf{antinukleäre Faktoren ANF}
						\item \textbf{Antikörper gegen mikrobielle Antigene}
						\item \textbf{HLA-Antigen B 27}
					\end{itemize}
				\item \textbf{BSG- und CRP-Erhöhung, Anämie}
				\item \textbf{Synoviaanalyse}
				\item \textbf{Röntgen}
			\end{itemize}
		\item \textbf{Therapie}
			\begin{itemize}
				\item Entzünungsbekämpfung
				\item Schmerztherapie
				\item Immunsupression
				\item Physiotherapie
				\item Operation
			\end{itemize}
		\item Rheuma Sammelbegriff
		\item Rheuma, gr - fließend
		\item oft fälschlich für Gelenksschmerzen benutzt
 	\end{itemize}

\subsection{Osteoporose}
	\begin{itemize}
		\item osteo - porös
		\item \textbf{systemische Skeletterkrankung mit pathologischem Knochenabbau}
			\begin{itemize}
				\item \textbf{abzugrenzen von normaler Altersatrophie}
				\item \textbf{Spongiosaverlust im Vordergrund}
			\end{itemize}
		\item \textbf{Ätiologie}
			\begin{itemize}
				\item \textbf{primäre Form = idiopathische O.}
					\begin{itemize}
						\item \textbf{Typ I: postmenopausal, Östrogenmangel, high-turnover}
						\item \textbf{Typ II: senile O., verminderte Knochenneubildung, Altersinvolution, low-turnover}
					\end{itemize}
				\item \textbf{sekundäre Form}
					\begin{itemize}
						\item \textbf{durch andere Grunderkrankung (Immobilisierung, Hyperthyreose, Cushing-Syndrom, $\dots$)}
					\end{itemize}
			\end{itemize}
		\item \textbf{Klinik}
			\begin{itemize}
				\item \textbf{chronischer Rückenschmerz}
					\begin{itemize}
						\item Keilwirbelbildung
						\item[$\rightarrow$] verstärkte Brustkyphose
						\item Brüchigkeit - schon bei leichten Berührungen Brüche
					\end{itemize}
				\item \textbf{Wirbelkörperdeformierungen}
				\item \textbf{Frakturen ohne entsprechendes Trauma}
			\end{itemize}
		\item \textbf{Diagnostik}
			\begin{itemize}
				\item \textbf{Rumpfveränderungen}
				\item \textbf{Radiologie}
				\item \textbf{Knochendensitometrie} (Knochendichtemessung)
			\end{itemize}
		\item \textbf{Therapie}
			\begin{itemize}
				\item \textbf{Ernährung (Ca, Vit. D), Bewegung}
				\item \textbf{Biphosphonate, Analgetika / Antiphlogistika} (Entzündungshemmer)\textbf{, ev. HRT} (Hormone-Replacement Therapy)
			\end{itemize}
 	\end{itemize}

\subsection{orthopädische Onkologie}
	\begin{itemize}
		\item \textbf{Allgemeines}
		\item \textbf{benigne primäre Knochentumoren}
		\item \textbf{maligne primäre Knochentumoren}
		\item \textbf{maligne sekundäre Knochentumoren = Metastasen}
		\item \textbf{(maligne Weichteiltumoren)}
	\pagebreak
		\item \textbf{Allgemeines}
			\begin{itemize}
				\item \textbf{Klassifikation: staging nach der Enneking-Klassifikation}
					\begin{itemize}
						\item \textbf{low-grade, high-grade}
						\item \textbf{intrakompartimental - Tu. noch auf Knochen beschränkt}
						\item \textbf{extrakompartimental - die Kortikalis durchbrochen}
						\item \textbf{Metastasen?}
					\end{itemize}
				\item \textbf{Klinik}
					\begin{itemize}
						\item \textbf{Schmerz}
						\item \textbf{Schwellung}
					\end{itemize}
			\end{itemize}
		\item \textbf{Diagnostik}
			\begin{itemize}
				\item \textbf{Bildgebung}
					\begin{itemize}
						\item \textbf{Röntgenbild, CT, MRT, Angiographie, Skelettszintigraphie}
					\end{itemize}
				\item \textbf{Biopsie}
			\end{itemize}
		\item \textbf{Therapie: 3 Säulen der Schulmedizin}
 	\end{itemize}

\subsection{maligne Knochentumoren}
	\begin{itemize}
		\item \textbf{Osteosarkom}
			\begin{itemize}
				\item am häufigsten
				\item 10-25 Jahre
				\item 5 Jahres Überlebensrate 70%
			\end{itemize}
		\item \textbf{Chondrosarkom}
			\begin{itemize}
				\item zweit häufigsten
				\item Knorpelbildung
				\item sechstes Lebensjahrzehnt
			\end{itemize}
		\item \textbf{Ewing-Sarkom}
			\begin{itemize}
				\item 12-17 Jahre
			\end{itemize}
 	\end{itemize}