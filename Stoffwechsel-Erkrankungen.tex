\section{Stoffwechsel-Erkrankungen}
\subsection{Definition}
	\begin{itemize}
		\item \textbf{Krankheiten, die verursacht werden durch eine Störung}
			\begin{itemize}
				\item \textbf{der Aufnahme}
				\item \textbf{der Verarbeitung}
				\item \textbf{des Abbaus eines Stoffes im Organismus}
			\end{itemize}
	\end{itemize}
	
\subsection{Einteilung - Übersicht}
	\begin{itemize}
		\item \textbf{Einteilung nach der Ursache}
			\begin{itemize}
				\item \textbf{exogen (Zufuhr eines Stoffes)}
				\item \textbf{endogen (Produktion oder Abbau eines Stoffes)}
			\end{itemize}
		\item \textbf{nach der Art der Störung}
			\begin{itemize}
				\item \textbf{Mangel oder Überschuss eines Stoffwechselproduktes}
			\end{itemize}
		\item \textbf{angeboren (genetisch) oder erworben}
			\begin{itemize}
				\item \textbf{genetisch bedingte Stoffwechselerkrankungen}
					\begin{itemize}
						\item \textbf{geringgradiger bis kein Umwelteinfluss}
						\item \textbf{Defekt oder Fehlen eines Enzyms $\rightarrow$ Abbau bestimmter Substanzen blockiert $\rightarrow$ Anhäufung $\rightarrow$ Schädigung}
					\end{itemize}
			\end{itemize}		
			\begin{itemize}
				\item \textbf{durch genetische Disposition und Umwelteinflüsse bedingte Stoffwechselerkrankungen}
					\begin{itemize}
						\item \textbf{Hyperlipidämien, Diabetes mellitus, Gicht, Mukoviszidose, $\dots$}
					\end{itemize}
				\item \textbf{erworbene Stoffwechselerkrankungen}
					\begin{itemize}
						\item \textbf{geringgradiger bis kein genetischer Einfluss}
					\end{itemize}
			\end{itemize}
		\item \textbf{Einteilung nach Stoffklassen}
			\begin{itemize}
				\item \textbf{Störungen des Fettstoffwechsels}
					\begin{itemize}
						\item 
					\end{itemize}
				\item \textbf{Störungen des Kohlenhydrat-Stoffwechsels}
					\begin{itemize}
						\item Bsp.: Diabetes
					\end{itemize}
				\item \textbf{Störungen des Eiweiß- und Aminosäurenstoffwechsels}
				\item \textbf{Störungen des Nukleinsäure-Stoffwechsels}
				\item \textbf{$\dots$}
			\end{itemize}
	\end{itemize}
\pagebreak
\subsection{Diabetes mellitus} (lat, Diabetese - Durchfluss, mellitus - honigsüß $\rightarrow$ honigsüßer Durchfluss)
	\subsubsection{Einleitung, Definition}
		\begin{itemize}
			\item \textbf{Def.: Sammelbegriff verschiedener Krankheiten-Stoffwechselstörungen, gekennzeichnet durch Mangel an biologisch wirksamem Insulin}
				\begin{itemize}
					\item Alle Diabetes Formen: Mangel an Insulinwirkung, entweder Mangel an Insulin, oder Beeinträchtigung der Wirksamkeit
					\item \textbf{absoluter Mangel (Insulinproduktion vermindert)}
					\item \textbf{relativer Mangel (nachlassende Insulinwirksamkeit oder Überwiegen der Insulin Antagonisten: Glukagon, Adrenalin und Kortison)}
					\item Kohlenhydrate, Eiweiße werden gespalten
					\item Langerhans'sche Inseln produzieren Insulin und Glucagon 
					\item Zucker schädigt Gefäße, daher schnellstmöglich in die Zellen
					\item Falls kein Energiebedarf wird Glukose als Glykogen gespeichert (Glykogen $\neq$ Glucagon)
					\item Insulin senkt Blutzuckerspiegel, Glucagon lässt ihn steigen
					\item Insulin nicht durch Anderes ersetzbar, Glucosemangel kann kompensiert werden
				\end{itemize}
			\item \textbf{Folgen des Insulinmangels (Leitsymptom)}
				\begin{itemize}
					\item \textbf{Hyperglykämie (nü, postprandial)}
					\item \textbf{ab Blutzuckerspiegel $>$ 150 mg \% $\rightarrow$ Glucosurie} (Glucose im Urin)
				\end{itemize}
		\end{itemize}
	\subsubsection{Einteilung des DM}
		\begin{itemize}
			\item \textbf{primärer DM}
				\begin{itemize}
					\item \textbf{Typ 1: DMT 1 (IDDM)} (insulin-dependent diabetes mellitus)
						\begin{itemize}
							\item \textbf{Insulinabhängig}
							\item \textbf{ca. 10\% aller DM}
							\item \textbf{früher: "juveniler DM"}
							\item \textbf{absoluter Insulinmangel aufgrund zerstörter B-Zellen}
							\item \textbf{Ätiologie}
								\begin{itemize}
									\item \textbf{Autoimmun-Erkrankung}
									\item \textbf{genetisch (Vater!)}
									\item \textbf{viral? Nahrung?}
								\end{itemize}
						\end{itemize}
					\item \textbf{Typ 2 DMT 2 (NIDDM)}
						\begin{itemize}
							\item \textbf{zunächst nicht insulinabhängig}
							\item \textbf{ca. 90\% aller DM}
							\item \textbf{früher "Altersdiabetes"} (Altersgrenzen verschwimmt aber)
							\item \textbf{Störung der B-Zellen (anfangs Hyperinsulinämie, Down-Regulation der Rezeptoren}(nachlassende Wirkung)\textbf{, Sekretionsstarre) und Insulinresistenz der Muskel-und Fettzellen}
							\item \textbf{Ätiologie}
								\begin{itemize}
									\item \textbf{genetische Disposition der Insulinresistenz (deutliche familiäre Häufung)}
									\item \textbf{erworbene Insulinresistenz - Wohlstandserkrankung (Lebensweise!), Fettstoffwechselstörung}
									\item \textbf{metabolisches Syndrom:$\dots$} (viele Stoffwechselsymptome)
								\end{itemize}
							\item Im Gegensatz zu IDDM auch durch orale anti-Diabetiker
						\end{itemize}
				\end{itemize}
\pagebreak
			\item \textbf{sekundärer DM}
				\begin{itemize}
					\item \textbf{Typ 3}
						\begin{itemize}
							\item \textbf{Pankreaserkrankungen, Pankreasektomie}
							\item \textbf{Überschuss kontrainsulinärer Hormone (Endokrinopathien)} (zB.: wegen Tumoren)
							\item \textbf{passagere Glukosetoleranzstörung (Stress, Med.)} (klinischer Dauerstress, Trauma-Stress)
						\end{itemize}
					\item \textbf{Typ 4}
						\begin{itemize}
							\item \textbf{Gestationsdiabetes} (Gestation - Schwangerschaft)
							\item nach Ende der Schwangerschaft wieder in Ordnung, aber erhöhte Chance auf Typ2
							\item Erhöhte Komplikationsrate vor und nach der Geburt
							\item Druch überhöhtes Zuckerangebot gibt es "giant babies" >5k
							\item Ungeborene produziert für Mutter Insulin, kommt aber nicht an
						\end{itemize}
				\end{itemize}
		\end{itemize}
	\subsubsection{Symptome}
		\begin{itemize}
			\item \textbf{DM Typ 1}
				\begin{itemize}
					\item \textbf{Polyurie}
					\item \textbf{Polydipsie}
					\item \textbf{Gewichtsabnahme}
					\item \textbf{Pruritus, trockene Haut, Furunkel etc.}
					\item \textbf{Müdigkeit, Leistungsschwäche}
					\item \textbf{Ketoazidose}
				\end{itemize}		
			\item \textbf{DM Typ 2}
				\begin{itemize}
					\item \textbf{wenig auffällig, Zufallsbefund}
					\item \textbf{Mykosen, Pruritus, Müdigkeit}
					\item \textbf{Diagnose-Parameter (Blut, Urin)}
					\item \textbf{bereits vorhandene Folgeerkrankungen}
				\end{itemize}
		\end{itemize}
	\pagebreak
	\subsubsection{Diagnostik}
			\begin{itemize}
				\item \textbf{BZ-Bestimmung (nüchtern, postprandial)}
				\item \textbf{oGTT} (Oraler Glukosetoleranztest)
				\item \textbf{Glucose im Urin bestimmen}
				\item \textbf{HBA-1c = glykosiliertes Hb (="Blutzuckergedächnis")}
					\begin{itemize}
						\item Gut für Verlaufskontrollen über Wochen
					\end{itemize}
			\end{itemize}
	\subsubsection{Therapie}
			\begin{itemize}
				\item \textbf{Typ 1}
					\begin{itemize}
						\item \textbf{Insulin} Injektionen
					\end{itemize}
				\item \textbf{Typ 2}
					\begin{itemize}
						\item \textbf{Lebensweise!}
						\item \textbf{orale Antidiabetika (Insulin unterstützende Medikamente)}
						\item \textbf{ev. Insulin}
					\end{itemize}
			\end{itemize}
	\subsubsection{Folgekomplikationen}
			\begin{itemize}
				\item \textbf{diabetische Mikroangiopathie: diabetische Retinopathie, diabetische Nephropathie}
					\begin{itemize}
						\item Retinopathie - kann bis zur Erblindung führen
						\item Nephropathie - kann zur Dialysepflicht führen
					\end{itemize}
				\item \textbf{unspezifische Makroangiopathie: frühe, beschleunigte arteriosklerotische Veränderungen (RisikoKHK, Hirninfarkt, pAVK, Beingangrän)}
				\item \textbf{Infektneigung}
				\item \textbf{diabetische Polyneuropathie}
				\item \textbf{diabetisches Fußsyndrom}
					\begin{itemize}
						\item v.a. in Beinen und Füßen
						\item tiefe Ulcera
						\item Fußpflege sehr wichtig bei fortgeschrittenem DM
						\item bis zur Amputation
					\end{itemize}
				\item \textbf{Fettleber}
				\item \textbf{Akutkomplikationen}
					\begin{itemize}
						\item \textbf{Hypoglykämie} (Unterzuckerung)
						\item \textbf{Hyperglykämie} (Überzuckerung)
						\item \textbf{diabetisches Koma} (durch Blutzuckerentgleisung)
						\item \textbf{ketoazidotisch, hyperosmolar)}
					\end{itemize}
			\end{itemize}
			
\subsection{Gicht = Arthritis urica}
	\begin{itemize}
		\item \textbf{Störung des Purin- und Harnsäurestoffwechsels}
			\begin{itemize}
				\item \textbf{Purine (Adenin und Guanin): Bestandteile der Nuleinsäuren RNA und DNA}
				\item \textbf{Harnsäure = physiologisches Endprodukt des Purinabbaues, zu 90\% über Nieren ausgeschieden}
				\item auch durch Nahrung aufgenommen
				\item meist Männer
			\end{itemize}
\pagebreak
		\item \textbf{Folgen}
			\begin{itemize}
				\item \textbf{Hyperurikämie (erhöhter Harnsäurespiegel im Blut)} (Harnsäure $\neq$ Harnstoff)
				\item \textbf{Ablagerung von Uratkristallen in Gelenken, gelenknahen Weichteilen (z.B. Sehnenscheiden), Knorpel (z.B. Ohrmuschel) und Niere}
			\end{itemize}
		\item \textbf{Einteilung}
			\begin{itemize}
				\item \textbf{primäre und sekundäre Gicht}
			\end{itemize}
		\item \textbf{primäre Gicht}
			\begin{itemize}
				\item \textbf{Störung des Purinstoffwechsels (genetische Disposition)}
				\item \textbf{Ablauf in vier Stadien}
					\begin{itemize}
						\item \textbf{asymptomatische Hyperurikämiegesteigerte Purinsynthese in der Leber, verminderte renale Ausscheidung von Harnsäure}
						\item \textbf{akuter Gichtanfall = Arthritis urica}
							\begin{itemize}
								\item \textbf{exogene Auslöser (purinreiche Kost, Alkohol, etc.)}
								\item \textbf{Podagra} (Großzehengrundgelenk)
								\item \textbf{Gonagra} (Kniegelenk)
								\item \textbf{Chiragra} (Hände)
								\item \textbf{Omagra} Schultergelenkt
							\end{itemize}
						\item \textbf{beschwerdefreie Intervalle}
						\item \textbf{chronische Gicht}
							\begin{itemize}
								\item \textbf{extraartikuläre Uratablagerungen = Gichttophi an Prädilektionsstellen} (zB.:Ohr)\textbf{, Gelenkdeformierungen, Gichtnephropathie} (Nierenerkrankung)
							\end{itemize}
					\end{itemize}								
			\end{itemize}
		\item \textbf{sekundäre Gicht}
			\begin{itemize}
				\item \textbf{Hyperurikämie durch}
					\begin{itemize}
						\item \textbf{verminderte Harnsäureausscheidung}
							\begin{itemize}
								\item \textbf{Niereninsuffizienz}
							\end{itemize}
						\item \textbf{vermehrten Harnsäureanfall durch erhöhten Zellzerfall oder Blockade der Zellneubildung}
							\begin{itemize}
								\item \textbf{maligne Tumore und deren Therapien}
							\end{itemize}
						\item \textbf{Nebenwirkung von Medikamenten}
					\end{itemize}
				\item Nebenwirkung von Tumorbehandlung
			\end{itemize}
	\end{itemize}
	
\subsection{Mukoviszidose = zystische Fibrose}
	\begin{itemize}
		\item (lat, mucus - Schleim, viscidus - klebrig) 
		\item \textbf{autosomal rezessiv vererbte Stoffwechselstörung}
			\begin{itemize}
				\item \textbf{Defekt am Chromosom 7}
				\item \textbf{Störung des Chlorid-Transportes in exokrinen Drüsenzellen $\rightarrow$ erhöhte Viskosität des Sekretes $\rightarrow$ Sekretrückstau $\rightarrow$ Keimbesiedelung $\rightarrow$ Infektion $\rightarrow$ Organschädigungen}
			\end{itemize}
		\item \textbf{Symptome}
			\begin{itemize}
				\item \textbf{Lungen: chronische Bronchitiden, Pneumonien}
				\item \textbf{Pankreas: Pankreasinsuffizienz (Untergewicht, Kleinwuchs, Fettstühle)}
					\begin{itemize}
						\item Es fehlt an spaltenden Verdauungsenzymen
						\item Nahrung kann nicht so weit aufgespalten werden, dass sie absorbiert wird
						\item ähnlich mit Fett
					\end{itemize}
				\item \textbf{Schweißdrüsen: stark salziger Schweiß}
				\item \textbf{Speicheldrüsen, Gallenwege, Dünndarm, $\dots$}
				\item Bei starker Ausprägung: sehr eingeschränkte Lebensqualität, Lungentherapie/Transplantation notwendig
			\end{itemize}
	\end{itemize}
	
\subsection{erworbene Stoffwechselerkrankungen}
	\begin{itemize}
		\item \textbf{Überernährung}
			\begin{itemize}
				\item \textbf{mehr Energieaufnahme als Verbrauch $\rightarrow$ Speicherung}
				\item \textbf{Einteilung nach BMI in Adipositas Grad I - III}
			\end{itemize}
		\item \textbf{Unterernährung}
			\begin{itemize}
				\item \textbf{langfristig zu geringe Kalorienzufuhr}
					\begin{itemize}
						\item \textbf{Marasmus}
						\item \textbf{Kwashiorkor}
						\item \textbf{Kachexie}
						\item \textbf{Anorexie}
					\end{itemize}
			\end{itemize}
		\item \textbf{Vitaminmangel}
			\begin{itemize}
				\item \textbf{Rachitis}
				\item \textbf{Skorbut}
			\end{itemize}
	\end{itemize}