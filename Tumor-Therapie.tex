 \section*{Tumor-Therapie}
	\begin{itemize}
		\item \textbf{3 Säulen der Schulmedizin}
			\begin{itemize}
				\item \textbf{Operation}
				\item \textbf{Radiotherapie:}
					\begin{itemize}
						\item \textbf{Zelltod duch ionisierende Stahlung, präoperative und/oder postoperative Bestrahlung}
						\item auch pre-OP, verkleinert den Tumor, zerstört besonders aktive Zellen, verringert OP-bedingtes Streuungsrisiko
					\end{itemize}					
				\item \textbf{Chemotherapie mit Zytostatika}
			\end{itemize}
		\item \textbf{Nebenwirkung}
			\begin{itemize}
				\item auch gesunde Zellen in Teilung werden vorübergehend zerstört. v.a. 
				\item Haut- und Schleimhautzellen, Haare $\rightarrow$ Haarausfall
				\item Blutzellen:
					\begin{itemize}
						\item[$\rightarrow$] Erythrozytenmangel $\rightarrow$ Anämie (Schwäche, depressive Verstimmung, …)
						\item[$\rightarrow$] Leukozytenmangel $\rightarrow$ Schwächung d. Immunsystems $\rightarrow$ mangelnde Abwehr, Infektanfälligkeit
					\end{itemize}
				\item Strahlentherapie bzw. Chemotherapie ist je nach Dosis kanzerogen $\rightarrow$ Risisko Zweittumor?
			\end{itemize}
		\item \textbf{neuere Methoden}
			\begin{itemize}
				\item \textbf{monoklonale Antikörper}\\
					weist d. Tumor Antigenkörper auf, kann man Antikörper geben, die Tumorzellen zerstören sollen
				\item \textbf{dendritische Zelltherapie}
				\item \textbf{Hyperthermie,$\dots$}
			\end{itemize}
	\end{itemize}
\pagebreak
	\subsection{einzelne Tumorbeispiele}
		\begin{itemize}
			\item \textbf{Malignes Melanom}\\
				Abgrenzung zum benignen Naevus (Muttermal)\\
				ABCD(E)-Regel:
				\begin{itemize}
					\item Asymmetrie
					\item Begrenzung
					\item Colour
					\item Durchmesser
					\item (Erhaben)
				\end{itemize}
			\item \textbf{Basaliom}\\
				„semimaligne“ = borderline, lokal malignes Wachstum aber keine Metastasierung!
			\item \textbf{Leukämien}\\
				Einteilung:
				\begin{itemize}					
					\item akute Leukämie
						\begin{itemize}
							\item 90\% Leukämien im Kindesalter
							\item myeloische ($\rightarrow$ akute meloische Leukämie)
							\item lymphatische ($\rightarrow$ akute lymphatische Leukämie
						\end{itemize}
					\item chronische Leukämie
						\begin{itemize}
							\item myeloische ($\rightarrow$ chronische myeloische Leukämie)
							\item lymphatische ($\rightarrow$ chronische lymphatische Leukämie)
						\end{itemize}
					\item[$\rightarrow$] Mangel an Erythrozyten = Anämie
					\item[$\rightarrow$] Mangel an Trombozyten $\rightarrow$ Blutgerinnungsproblem, Spontanblutungen
					\item[$\rightarrow$] Mangel an Leukozyten $\rightarrow$ Abwehrschwäche, Infektanfälligkeit
				\end{itemize}
			\item \textbf{maligne Lymphome}
				\begin{itemize}
					\item M(orbus)-Hodgkin-Lymphom
						\begin{itemize}
							\item geht von B-Lympozyten aus
							\item Symptome
								\begin{itemize}
									\item Nachtschweiß, Gewichtsverlust, evtl. Fieber
									\item erhöhte BSG, Blutsenkungsgeschwindigkeit (später mehr)
									\item manchmal Schmerzen/Juckreiz nach Alkoholkonsum
								\end{itemize}
							\item Staging:
								\begin{itemize}
									\item Eine, zwei, mehrere Knoten befallen
									\item Behandlung: Strahlen \& Chemo
								\end{itemize}
							\item Prognose:
								\begin{itemize}
									\item bei Früherkennung 70\% Überlebensrate
								\end{itemize}
						\end{itemize}
					\item Non-Hodgkin-Lymphom
				\end{itemize}		
			\item \textbf{Hodencarcinom}
				\begin{itemize}
					\item Altersgipfel: 20-30
					\item überwiegend v. Keimzellen ausgehend $\rightarrow$ Keimzellentumore (häufigster maligner Tumor bei jungen Männern)
					\item Ätiologie: risikoerhöhend: Hoden zum Zeitpunkt der Geburt nicht im Skrotum (noch in Bauchhöhle)
				\end{itemize}
			\item \textbf{Prostatacarcinom}
				\begin{itemize}
					\item überwiegend ältere ältere Männer
					\item durch Abfall v. Testosteron relativer Anstieg von Östrogen $\rightarrow$ Wachstumsstimulus für Prostata
					\item Therapie:
						\begin{itemize}
							\item OP (möglichst Nerven-schonend! aber: höheres Risiko, nicht alle Carcinom-Anteile zu entfernen!)
							\item Hormontherapie: anti-androgen (Nebenwirkung: „Verweiblichung“ $\rightarrow$ z.B. Brustdrüsenwachstum)
						\end{itemize}
				\end{itemize}
			\pagebreak
			\item \textbf{Mammacarcinom}
				\begin{itemize}
					\item Insidenz nimmt stetig zu, zzT. jede 8. Frau
					\item Lymphknoten in der Achsel wird kaum noch durchgeführt
					\item Lokalisation meist obere Hälfte
					\item Risikofaktoren
						\begin{itemize}
							\item genetische Veranlagung
							\item Östrogene
								\begin{itemize}
									\item frühe Menarche (erste Regelblutung)
									\item späte Monopause
									\item Östrogentherapie i.d. Menopause
									\item Keine Schwangerschaften (Schwangerschaft+Stillzeit unterbricht Zyklus)
									\item Adipositas
								\end{itemize}
						\end{itemize}
					\item mit dem Alter deutlich Ansteigend nach 50
					\item gute Prognose
					\item Behandlung
						\begin{itemize}
							\item Operation
							\item kosmetische Restauration
						\end{itemize}
				\end{itemize}
			\item \textbf{Cervixcarcinom}
				\begin{itemize}
					\item 
				\end{itemize}
			\item \textbf{Coloncarcinom}
				\begin{itemize}
					\item Insidenz nimmt stetig zu, vermutlich auf Grund von Lebensweise
					\item möglicherweise Ernärung und Genetik Ursachen\\
						(cancerogene Lebensmittel bleiben länger im Colon durch balaststoffarme Ernärung)
					\item 90\% entwickeln sich aus malignen Polypen\\
						(Vorsorgliche Spiegelung im höheren Alter, Entfernung und Analyse der Polypen)
					\item Therapie
						\begin{itemize}
							\item Chemo Therapie mit Operation
						\end{itemize}
					\item Metastasen $\rightarrow$ Leber $\rightarrow$ Lunge 
					\item 90\% Überlebensrate bei rechtzeitiger Behandlung
				\end{itemize}
				
		\end{itemize}
		\subsection{Einschub: Erethrozyten}
		Blut
			\begin{itemize}
				\item Flüssigkeit = Plasma (Serum: ohne Gerinnung)
				\item Zellen
					\begin{itemize}
						\item Erythrozyten (Hämoglobin, O2-Transport, ABO-System, Rh-System
						\item Thrombozyten: Gerinnselbildung (Thrombus) zur Gefäßwandabdichtung
						\item Leukozyten
							\begin{itemize}
								\item Granulozyten
								\item Monozyten	
								\item Lymphozyten
									\begin{itemize}
										\item T(hymus)-Lymphozyten
										\item B(one marrow)-Lymphozyten
										\item NK-Zellen
									\end{itemize}
							\end{itemize}
								
					\end{itemize}
			\end{itemize}