\documentclass[10pt,a4paper,fleqn]{article}
\usepackage[utf8]{inputenc}
\usepackage{amsmath}
\usepackage{amsfonts}
\usepackage{amssymb}
\usepackage{graphicx}
\usepackage[left=2cm,right=2cm,top=2cm,bottom=2cm]{geometry}
\usepackage[hidelinks]{hyperref}
\usepackage[naustrian]{babel}
\usepackage{csquotes}
\usepackage{pbox}
\MakeOuterQuote{"}
\begin{document}
\begin{enumerate}
\section{Allgemeine Krankheiten}
	\item \textbf{Was versteht man unter: Ätiologie, Morbidität, Mortalität, Letalität, Inzidenz}
		\begin{itemize}
			\item \textbf{Ätiologie}
				\begin{itemize}
					\item \textbf{Lehre von den Krankheitsursachen}
				\end{itemize}
			\item \textbf{Morbidität} (Mobrid, lat - krank)
				\begin{itemize}
					\item \textbf{Häufigkeit einer bestimmten Krankheit in einer Bevölkerungsgruppe}
					\item \textbf{Verhältnis der Zahl der Erkrankungen zur Zahl der Gesamtbevölkerung in einem bestimmten Zeitraum}, meist pro Jahr
				\end{itemize}
			\item \textbf{Mortalität ("Sterblichkeit")} (mortal, lat - tödlich)
				\begin{itemize}
					\item \textbf{Häufigkeit einer bestimmten Krankheit als Todesursache in einer Bevölkerungsgruppe}
					\item \textbf{Verhältnis der Zahl der Todesfälle an bestimmter Erkrankung zur Zahl der Gesamtbevölkerung in einem bestimmten Zeitraum, in der Regel 1 Jahr, pro 10k Einwohner}
				\end{itemize}
			\item \textbf{Letalität ("Tödlichkeit")}
				\begin{itemize}
					\item \textbf{Zahl der Todesfälle bezogen auf die Zahl der Erkrankten}
					\item \textbf{Verhältnis der Zahl der Todesfälle zur Zahl der an einer bestimmten Krankheit Erkrankten ("Mortalität in \%")}
				\end{itemize}
			\item \textbf{Inzidenz (Erkrankungshäufigkeit)} (Inzidere, lat - Neubeginn)
				\begin{itemize}
					\item \textbf{Zahl von Neuerkrankungen an einer bestimmten KH innerhalb eines bestimmten Zeitraumes}
					\item \textbf{Anzahl der Personen, die im Verlauf eines bestimmten Zeitraumes (i.d.R. 1 Jahr) an einer bestimmten Krankheit erstmals erkranken}
					\item Nimmt das in einer bestimmten Region zu? $\rightarrow$ der Ursache nachgehen
				\end{itemize}
		\end{itemize}
	\item \textbf{Zytologische und histologische Diagnostik: was wird gemacht? Zweck?}
		\begin{itemize}
			\item \textbf{zytologische Untersuchungsmethoden}
				\begin{itemize}
					\item \textbf{Analyse von Einzelzellen}
					\item \textbf{Gewinnung der Zellen}
						\begin{itemize}
							\item \textbf{von Schleimhautoberflächen, Sekreten, Spülflüssigkeiten}
								\begin{itemize}
									\item Abstrichbeispiel: optimal Papanicolaou: Pap - Diagnose im Gebärmutterhals zur Diagnose von Krebs. Nicht sonderlich invasiv, kostengünstig, sehr aussagekräftig.
								\end{itemize}
							\item \textbf{durch Punktion von Flüssigkeiten}
								\begin{itemize}
									\item Entnahme von Sekreten. Einspritzen und Ausspülen und anschließende Analyse der Flüssigkeit in Organen
								\end{itemize}
							\item \textbf{durch Feinnadelpunktion von Organen}
								\begin{itemize}
									\item "pflücken" von Zellen mit einer Feinnadel
								\end{itemize}
						\end{itemize}
					\item \textbf{Zweck / häufige Fragestellungen}
						\begin{itemize}
							\item \textbf{infektiöse Erkrankungen (Erregernachweis) und deren Folgen}
							\item \textbf{Entzündungsdiagnostik}
							\item \textbf{Tumorzellnachweis}
							\item \textbf$\dots$}
						\end{itemize}
				\end{itemize}
			\item \textbf{histologische Untersuchungsmethoden}
				\begin{itemize}
					\item \textbf{Analyse von Gewebeschnitten von chirurgisch oder bioptisch gewonnenen Gewebestücken}
					\item \textbf{Gefrierschnellschnitt, Paraffinschnitt}
						\begin{itemize}
							\item Parafinschnitte sehr haltbar und aussagekräftiger als Gefrierschnitt
							\item Parafinschnitt 3-4 Tage, Gerierschnitt 15 Minuten
						\end{itemize}
					\item \textbf{Analysetechniken}
						\begin{itemize}
							\item \textbf{Lichtmikroskopie}
								\begin{itemize}
									\item \textbf{Anfertigung von Paraffinschnitten oder Gefrierschnitten}
								\end{itemize}
							\item \textbf{Immunfluoreszenz}
							\item \textbf{Elektronenmikroskopie}
						\end{itemize}
					\item \textbf{Zweck / häufige Fragestellungen}
						\begin{itemize}
							\item \textbf{Zellbeurteilung im Gewebeverband, Tumordiagnostik, Kontrolle des chirurgischen Eingriffs (Resektionsränder), etc.}
						\end{itemize}
					\end{itemize}
		\end{itemize}
	\item \textbf{Nennen Sie 3 Möglichkeiten Krankheitserreger nachzuweisen}
		\begin{itemize}
			\item u.a. Gerichtsmedizin
			\item \textbf{intravitale Diagnostik} auch am lebenden Gewebe
				\begin{itemize}
					\item \textbf{zytologische Untersuchungsmethoden}
					\item \textbf{histologische Untersuchungsmethoden}
				\end{itemize}
			\item \textbf{postmortale Diagnostik}
				\begin{itemize}
					\item \textbf{sanitätspolizeiliche Obduktion}
					\item \textbf{gerichtsmedizinische Obduktion}
					\item \textbf{klinische Obduktion}
				\end{itemize}
		\end{itemize}
	\item \textbf{Nennen Sie je 2 innere und äußere Krankheitsursachen}
	\item \textbf{Unterschied akuter / chronischer Krankheitsverlauf }
	\item \textbf{Nennen Sie die 3 Möglichkeiten des Krankheitsausgangs}
	\item \textbf{Definition: Rezidiv, Remission}
	\item \textbf{Nennen Sie je 2 unsichere und sichere Todeszeichen}
	\item \textbf{Definition (grob): Nekrose, Apoptose, Ulkus, Dekubitus}
	\item \textbf{Was ist eine Atrophie mit je 1 Bsp.: physiolog.A., lokale A., general. A.}
	\item \textbf{Was ist eine Hypertrophie / Hyperplasie mit je 1 Bsp.}
	\item \textbf{Was ist eine Metaplasie (Dysplasie, Anaplasie)?}
	\item \textbf{Nennen Sie je 4 Kennzeichen benigner / maligner Tumoren}
	\item \textbf{Metastasen: Definition, einzelne Metastasierungsarten}
	\item \textbf{Definition: Präkanzerose = ? obligate, fakultative = ? je ein Beispiel}
	\item \textbf{Tumornomenklatur (grob): Carcinom = ? Sarkom = ? Adeno-, Fibro-, Lipo-, Myo-, Chondro-, Osteo- = ?}
	\item \textbf{Tumor-staging: TNM-System = ?}
	\item \textbf{Malignitätsgrading: Grade und Kriterien grob}
	\item \textbf{Ätiologie maligner Tumoren: nennen Sie je ein Beispiel hormoneller, chemischer, physikalischer, infektiöser und Ernährungs-Faktoren}
	\item \textbf{Was sind Tumormarker, was können sie / was nicht,  plus 2 Bsp.}
	\item \textbf{Entzündung: Definition, Zweck?}
	\item \textbf{Nennen Sie 2 Bsp. von Entzündungsmediatoren}
	\item \textbf{Nennen Sie die 5 lokalen Entzündungszeichen ("Kardinalsymptome")}
	\item \textbf{Welche Blutwerte sind bei Entzündung erhöht?}
	\item \textbf{Was ist ein Ödem und nennen Sie wesentlichen die Ödemarten}
	\item \textbf{Thrombose = ? 3 Entstehungsfaktoren (Virchow'sche Trias)}
	\item \textbf{Embolie: Definition, Folge, Einteilung: arteriell, venös}
	\item \textbf{Arteriosklerose: Was passiert an der Arterienwand? Folgeerkrankungen (Lokalisationen), 3 Risikofaktoren}
	\item \textbf{Aneurysmen: Definition, Folgen}
	\item \textbf{pAVK =  ? Ätiologie, Symptome}
	\item \textbf{Varizen = ? Phlebothrombose = ? Gefahr?}
	\item \textbf{arterielle Hypertonie – ab welchen Werten patholog.? Folgeerkrankungen}
\section{Herzerkrankungen}
	\item \textbf{Was ist das Zeichen einer kardialen Überlastung (Herzhypertrophie)}
	\item \textbf{Herzinsuffizienz = ? Ätiologie? Je 2 Symptome Li- u Re-Herzinsuff.}
	\item \textbf{Bradykardie = ? Tachykardie = ?}
	\item \textbf{koronare Herzkrankheiten: KHK}
		\begin{itemize}
			\item \textbf{Angina pectoris: Leitsymptome, Einteilung}
			\item \textbf{Myokardinfarkt = ? Symptome, Diagnostik, Therapie}
		\end{itemize}
	\item \textbf{Klappenvitien: Stenose = ? Insuffizienz = ?}
\section{Erkrankungen des Atemsystems}
	\item \textbf{Labordiagnostik bei Atemwegserkrankungen: welche Blutwerte werden bestimmt?}
	\item \textbf{COPD}
		\begin{itemize}
			\item \textbf{Lungenemphysem = ? Ätiologie? Symptome? Risikofaktoren?}
		\end{itemize}
	\item \textbf{Pneumonie: Diagnostik?}
	\item \textbf{Lungenembolie = ? Folge? Woher stammt der Thrombus meist? Symptome? Diagnostik? Therapie - medikamentös?}
\section{Pathologie des Stoffwechsels}
	\item \textbf{Diabetes mellitus}
		\begin{itemize}
			\item \textbf{Einteilung: Typ1 und Typ2, Folgeerkrankungen, Diagnostik}
		\end{itemize}
	\item \textbf{Mukoviszidose: Ursache genau, Lungensymptomatik}
	\item \textbf{Gicht: welche Stoffwechselstörung? Welche Substanz kristallisiert aus? Symptome}
\section{Neurologie}
	\item \textbf{Multiple Sklerose: Pathogenese? Verlaufsformen?}
	\item \textbf{Morbus Parkinson: Pathogenese? Symptomen-Trias?}
	\item \textbf{Schlaganfall: Ätiologie? Diagnostik? Therapie? TIA = ?}
\section{Orthopädie}
	\item \textbf{Arthrose = ? Welche Gelenke? Symptome? Therapie?}
	\item \textbf{Osteoporose = ? Ätiologie? Wirbel-  und Wirbelsäulenveränderungen?}
\end{enumerate}
\end{document}