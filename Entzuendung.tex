\section{Entzündung}
	\begin{itemize}
		\item \textbf{Definition}
			\begin{itemize}
				\item \textbf{Entzündung ist Reaktion des Gewebes auf einen schädigenden Reiz}
				\item Abwehrreaktion - es geht ihr eine Gefahr voraus. Es entstehen auch Kollateralschäden 
				\item Danach ist Reperation nötig
			\end{itemize}
		\item \textbf{Bezeichnung}
			\begin{itemize}
				\item \textbf{"-itis" (mit Ausnahme)}
			\end{itemize}
		\item \textbf{Zweck der Entzündung}
			\begin{itemize}
				\item \textbf{Ausschalten des ursprünglichen Entzündungsreizes}
				\item \textbf{Reparation, d.h. Ersatz des zugrundegegangenen Gewebes}
			\end{itemize}
		\item \textbf{Ursachen (Entzündungsreize)}
			\begin{itemize}
				\item \textbf{lebende Organismen}
				\item \textbf{mechanische, chemische, physikalische Einwirkung, u.a.}
			\end{itemize}
		\item \textbf{Faktoren, die Art und Ablauf einer Entzündung beeinflussen}
			\begin{itemize}
				\item \textbf{Beschaffenheit des Gewebes}
					\begin{itemize}
						\item lockeres Gewebe entzündlicher, als festes Bindegewebe
					\end{itemize}
				\item \textbf{Durchblutung}
					\begin{itemize}
						\item schlecht durchblutetes Gewebe $\rightarrow$ Nekroses, Gutdurchblutetes wird repariert
					\end{itemize}
				\item \textbf{Alter, Ernärungszustand, konsumierende Erkrankungen}
				\item \textbf{Störung der Imunabwehr}
				\item \textbf{bei Infektion: Virulenz des Erregers}
			\end{itemize}
		\item \textbf{an einer Entzündung sind beteiligt}
			\begin{itemize}
				\item \textbf{Abwehrzellen (Granulozyten, Lymphozyten, Monozyten)}
				\item \textbf{Thrombozyten, Erythrozyten}
			\end{itemize}
		\item \textbf{Entzündungsmediatoren}
			\begin{itemize}
				\item \textbf{chemische Faktoren, die den Entzündungsprozess steuern} (Histamin)
				\item Werden während der Entzündung freigesetzt. Medikamente wirken häufig dem entgegen.
			\end{itemize}
		\item \textbf{Beispiele}
			\begin{itemize}
				\item \textbf{Prostaglandine und Leukotriene}
				\item \textbf{Histamin}
				\item \textbf{Serotonin}
				\item \textbf{Kallikrein-Kinin-System}
				\item \textbf{Zytokine, Inferone und Interleukine}
			\end{itemize}
		\item \textbf{Wirkung}
			\begin{itemize}
				\item \textbf{Vasodilitation $\rightarrow$ Permeabilitätssteigerung}
					\begin{itemize}
						\item Flüssigkeit im Gewebe $\rightarrow$ Schwellung
					\end{itemize}
				\item \textbf{Erregung der Schmerzrezeptoren (ev. Juckreiz} (Histamin)
				\item \textbf{Aktivierung der Phagozyten}
					\begin{itemize}
						\item Fresszellen fressen Erreger, dabei enstehen Zerfallsprodukte
					\end{itemize}
				\item \textbf{Fieber}
				\item \textbf{$\dots$}
			\end{itemize}
\pagebreak
		\item \textbf{lokales Entzündungsgeschehen}
			\begin{itemize}
				\item \textbf{Stärkung der Miktozirkulation $\rightarrow$ Rötung und Erwärmung}
				\item \textbf{Steigerung der Gefäßpermeabilität $\rightarrow$ Schwellung, Schmerz, eingeschränkte Funktion}
				\item \textbf{Reparation $\rightarrow$ Deckung des entstandenen Gewebsdefektes mit Granulationsgewebe anschließend Umwandlung in Narbengewebe}
			\end{itemize}
		\item \textbf{lokale Entzündungszeichen = "Kardinalsymptome"}
			\begin{itemize}
				\item \textbf{Rötung}
				\item \textbf{Schwellung}
				\item \textbf{Überwärmung} (Erhöhte lokale Erwärmung durch verbesserte Durchblutung)
				\item \textbf{Schmerz}
				\item \textbf{eingeschränkte Funktion}
			\end{itemize}
		\item \textbf{allgemeine Entzündungszeichen}
			\begin{itemize}
				\item \textbf{erhöhte Temperatur}
					\begin{itemize}
						\item Wenn eine lokale Bekämpfung nicht möglich ist, wird diese im gesamten Körper ausgetragen $\rightarrow$ Fieber + deutliche Vermehrung der Abwehrzellen (Leukozytosen), Temperaturerhöhung.
					\end{itemize}
				\item \textbf{Leukozytose (Welche sind erhöht? Hilft bei Diagnose)}
				\item \textbf{erhöhte BSG} (Blutsenkungsgeschwindigkeit) \textbf{und CRP} (C-Reaktives Protein)\textbf{, erhöhte Immuglobuline} (wieder Einteilung in Klassen zur Diagnose)
				\item \textbf{(Krankheitsgefühl)}
			\end{itemize}
		\item \textbf{Ausbreitungsmöglichkeiten einer Entzündung}
			\begin{itemize}
				\item \textbf{hämatogene Streuung}
				\item \textbf{lymphogene Streuung}
				\item \textbf{kontinuierliche Ausbreitung}
				\item \textbf{kanalikuläre Ausbreitung (in Organen mit Gangsystem)}
				\item Ähnlich wie Tumorausbreitung
			\end{itemize}
		\item \textbf{Eintelung nach Dauer und Verlauf}
			\begin{itemize}
				\item \textbf{perakut} (unmittelbar Lebensbedrohlich)
				\item \textbf{akut}
				\item \textbf{subakut}
				\item \textbf{chronisch}
				\item \textbf{rezidivierend}
			\end{itemize}
		\item \textbf{Einteilung nach der Art des vorherrschenden Entzündungsgeschehens}
			\begin{itemize}
				\item \textbf{exsudativ}
					\begin{itemize}
						\item \textbf{Austreten von flüssigen und zellulären Blutbestandteilen in das umliegende Gewebe (serös, fibrinös, eitrig, hämorrhagisch,$\dots$)}
					\end{itemize}
				\item \textbf{alterierend/nekrotisierend}
					\begin{itemize}
						\item \textbf{Schädigung des betroffenen Gewebes von Dystrophie bis Nekrose}
					\end{itemize}
				\item \textbf{proliferativ}
					\begin{itemize}
						\item \textbf{entzündungsbedingte, lokale Vermehrung von Granulationsgewebe}
							\begin{itemize}
								\item Ersatzgewebe - neu eingewachsene Kapillaren sehen im Querschnitt aus wie Körnchen
								\item Narbengewebe - einfachstes Bindegewebe
							\end{itemize}
					\end{itemize}
			\end{itemize}
		\end{itemize}
	\subsection{Entzündungsbeispiele}
		\begin{itemize}
			\item \textbf{Rhinitis, Sinuitis, Otitis media, Pharyngitis, Laryngitis, Tracheitis}
				\begin{itemize}
					\item Rhinitis (Nase) $\rightarrow$ Sinuitis (Nasennebenhöhlen)
						\begin{itemize}
							\item kann Eitrig werden
							\item bei komplexen Verlauf operative Entleerung
							\item bei bakteriellem Verlauf Antibiotika
						\end{itemize}
					\item[$\rightarrow$] Otitis media (Mittelohr)
						\begin{itemize}
						 	\item Wölbung des Trommelfells, starker Schmerz, oft eitrig, kann Trommelfell aufreisen $\rightarrow$ Vernarbung, Einschränkung des Hören
						 \end{itemize}
					\item absteigen der Viren $\rightarrow$ Pharyngitis (Mund-Racheninfektion)
						\begin{itemize}
							\item Meist nur Behandlung der Symptome nötig, bakteriell können sich Streptokokken ansammeln die mit Antibiotika zu therapieren sind, ansonsten Wochen später irrtümliche Auto-immun Reaktion, nach Streptokokken Erkrankung an Herz und Nieren
						\end{itemize}
					\item[$\rightarrow$] Laryngitis (Kehlkopf) $\rightarrow $Heiserkeit, Beschwerden beim Schlucken
					\item Tracheitis (selten allein) $\rightarrow$ Broncheitis
				\end{itemize}
			\item \textbf{Bronchitis}
			\item \textbf{Pneumonie}
				\begin{itemize}
					\item Pleuritis
				\end{itemize}
			\item \textbf{Endocarditis, Myocarditis, Pericarditis} (Entzündungen am Herzen)
%	\pagebreak
			\item \textbf{Appendicitis}
				\begin{itemize}
					\item nicht der gesamte Blinddarm, nur Wrumvortsatz
					\item Symptome:
						\begin{itemize}
						 	\item Schmerz meist im rechten Unterbauch
						 	\item aber auch hinten oder links unten
						 	\item Spannungsschmerz $\rightarrow$ verkrümmte Haltung
						 \end{itemize}
					\item Diagnose
						\begin{itemize}
							\item Loslasschmerz and Druckschmerzpunkte
							\item Blutanalyse $\rightarrow$ sämtliche oben genannte Indikatoren
							\item Bildgebend: Ultraschall
						\end{itemize}
					\item Operation = Appendectomie
					\item Komplikationen
					\begin{itemize}
						\item Durchbruch $\rightarrow$ Ausweitung auf Bauchfell (Peritonitis) $\rightarrow$ Bauchhöle 
						\item Schockgeschehen, wird Lebensbedrohlich
						\item Sepsis, Streuung über Blutweg in den ganzen Körper ("Blutvergiftung")
					\end{itemize}
				\end{itemize}
\pagebreak
			\item \textbf{Gastritis}
				\begin{itemize}
					\item Ursachen
						\begin{itemize}
							\item \textbf{A}utoimmun
							\item \textbf{B}akteriell: Helicobacter pylori, hohe Druchseuchtungsfaktor, nur selten Komplikationen
							\item \textbf{C}hemisch, aggressive Nahrungsinhaltsstoff: Nikotin, Alkohol, zu heiß/kalt, zu scharf
						\end{itemize}
					\item Symptome
						\begin{itemize}
							\item Rötung
							\item Schwellung
							\item kein Fieber, Blutwerte
						\end{itemize}
					\item Behandlung
						\begin{itemize}
							\item Diätnahrung
						\end{itemize}
					\item chronische Gastritis
					\item Helicobacter pylori $\rightarrow$ Ulcus im Magen, Antibiotische Therapie
					\item Diagnose
						\begin{itemize}
							\item Endoskopie
						\end{itemize}
				\end{itemize}
			\item \textbf{Enterokolitis}
				\begin{itemize}
					\item Dünn/Dickdarm Entzündung
					\item Vieren, Häufung bei heißen, unhygienischer Umgebung
					\item Durchfall, Erbrechen
					\item Flüssigkeitsersatz (v.a. junge und alte Menschen)
					\item Salmonellen
					\item Entzünungszeichen im Stuhl, Antigene im Blut
				\end{itemize}
			\item \textbf{Cholecystitis}
				\begin{itemize}
					\item Gallenblasenentzündung
					\item Entzündung + Steinleiden meist kombiniert
					\item Risikofaktoren
						\begin{itemize}
							\item 5 F
								\begin{itemize}
									\item Female
									\item 40
									\item fertile
									\item fat
									\item fair haired
									\item (family)
								\end{itemize}
						\end{itemize}
					\item Ausdehnung auf Bauchspeicheldrüse $\rightarrow$ Pankreatitis
				\end{itemize}
	\pagebreak
			\item \textbf{Pankreatitis}
				\begin{itemize}
					\item Bauchspeicheldrüse
					\item Blutzuckerregulierende Hormone
					\item chronisch und akut
					\item Auslöser
						\begin{itemize}
							\item Alkoholexzess, auch in jungen Jahren
						\end{itemize}
					\item Mitbeteiligung mit Gallenerkrankung
				\end{itemize}
			\item \textbf{Hepatitis}
				\begin{itemize}
					\item Hep. A: komplikationsloses Erbrechen/Durchfall, fäkal-oral Übertragen
					\item Hep. B: kann in Leberzerose enden, relativ komplikationslos, STD
					\item Hep. C: komplikationsreich $\rightarrow$ Leberzerose
				\end{itemize}
			\item \textbf{Urocystitis}
				\begin{itemize}
					\item Harnblasenentzündung
					\item überwiegend bakteriell (warm, feucht, dunkel)\\
						$\rightarrow$ hauptsächlich Frauen betroffen
					\item häufig rezidivierende Harnwegsinfekte
					\item Symptome
						\begin{itemize}
							\item Schmerzen
							\item blutiger Harn
						\end{itemize}
					\item Ursachen
						\begin{itemize}
							\item gehäuftes Auftreten bei jungen Frauen, bei häufigem Auftreten Ursachenforschung
							\item Geschlechtsverkehr (urinieren nach Geschlechtsverkehr)
							\item im Alter ist Restharn Auslöser
							\item Belastung bei Schwangerschaft
							\item Verengung d. Prostata
						\end{itemize}
					\item Komplikationen
						\begin{itemize}
							\item Aufsteigen über Harnleiter $\rightarrow$ \textbf{Pyelonepthritis}
							\item \textbf{Glomerulonephritis}
						\end{itemize}
				\end{itemize}
			\item \textbf{Arthritis}
			\item \textbf{Neuritis}
			\item \textbf{Meningitis, Encephalitis} (Gehirnentzündung)
			\item \textbf{Salpingitis, Orchitis} (Eileiter, Hoden)
			\item \textbf{$\dots$}
	\end{itemize}