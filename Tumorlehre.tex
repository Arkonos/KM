\section{Tumorlehre}
	\subsection{Grundlagen}
		\begin{itemize}
			\item \textbf{Tumor} (lat) \textbf{("onkos")} (gr) \textbf{= Schwellung = Volumenzunahme}
			\item \textbf{synonym: Neoplasma} (Neoplasie) \textbf{("Neubildung"), Blastom}
				\begin{itemize}
					\item \textbf{Neubildung körpereigenen Gewebes mit autonomer Wachstumstendenz, die jene eines normalen Gewebes weit übersteigt}
					\item \textbf{Wachstum ist unkontrolliert und überschießend, auf Kosten der gesunden Zellen (Nährstoffentzug!)} Geht weiter, auch wenn es keine Ursache mehr dafür gibt.
					\item \textbf{Wachstum wird auch nach Wegfall der auslösenden Ursache nicht eingestellt}
						\begin{itemize}
							\item \textbf{vgl. Hypertrophie/-plasie: Wachstum vom auslösenden Reiz abhängig, reversibel}
						\end{itemize}
				\end{itemize}
			\item \textbf{Dignität: es gibt gutartige (benigne) und bösartige (maligne) Tumore}
				\begin{center}
					\begin{tabular}{|rl|}
						\hline
						\textbf{benigne Tumore} & \textbf{maligne Tumore} \\
						\hline
						\multicolumn{2}{|c|}{\textbf{Wachstum}} \\ 
						langsam & rasch \\ 
						scharf begrenzt (Kapsel) & unscharf begrenzt \\ 
						expansiv, verdrängend & infiltrativ, eindringend \\
						komprimierend & destruierend \\
						eher verschieblich & nicht verschieblich \\
						\multicolumn{2}{|c|}{\textbf{Zellen}} \\
						keine Zellatypien & Zellatypien (Analplasie) \\
						differenziert & undifferenziert \\
						reif & \pbox{20cm}{unreif (nicht höher ausgebildet,\\ keine Feinheiten in der Ausbildung)} \\
						\multicolumn{2}{|c|}{\textbf{Metastasen}} \\
						keine Metastasen & \pbox{20cm}{bildet Metastasen: Streuung,\\Aussiedeln an anderen Stellen im Körper} \\
						\multicolumn{2}{|c|}{\textbf{Verlauf}} \\
						geringe Allgemeinstörung & starke Allgemeinstörung \\
						\pbox{20cm}{wenig Rezidive: kein Erneutes\\ Wiederauftreten nach Heilung} & oft Rezidive \\
						meist keine direkte Lebensgefahr & meist hohe Lebensgefahr\\
						\hline
					\end{tabular}
				\end{center}
			\item Es ist kenie klare Abgrenzung und Unterscheidung möglich, der Übergang ist fließend.
		\end{itemize}
	\subsection{borderline lesions (=semimaligne Tumore)}
		\begin{itemize}
			\item \textbf{wachsen bösartig (lokal infiltrativ und destruierend)}
			\item \textbf{hochgradige Rezidivneigung}
			\item \textbf{metastasierend, jedoch sehr selten und sehr spät}
		\end{itemize}
	\subsection{Präkanzerosen}
		\begin{itemize}
			\item "vor dem Krebs"
			\item \textbf{Gewebsveränderungen mit erhöhtem Risiko der malignen Entartung}
			\item \textbf{fakultative Präkanzerose: Entartungsrisiko $<$ 20\%, Dauer $>$ 5 Jahre} (Colotis ulcerosa - Darm)
			\item \textbf{obligate Präkanzerose: Entartungsrisiko $>$ 20\%, Dauer $<$ 5 Jahre}
				\begin{itemize}
					\item hohes Risiko
					\item Durch Gendefekt in Familien
					\item starkes Polypenwachstum (gestielt - Adenom) im Dickdarm mit großem Entartungsrisiko $\rightarrow$ Entfernung des Dickdarmstückes
					\item[$\rightarrow$] Stuhl nicht mehr eingedickt, künstlicher Darmausgang, hoher Leidensdruck
					\item familiäre Ademomatose
				\end{itemize}
		\end{itemize}
\pagebreak
	\subsection{Metastasen}
		\begin{itemize}
			\item \textbf{Absiedlungen (Tochtergeschwülste) vom Primärtumor (Muttergeschwulst) über}
				\begin{itemize}
					\item \textbf{den Lymphweg (lymphogen)}
						\begin{itemize}
							\item \textbf{regionale Lymphknoten - weitere Lymphknotengruppen - über den Ductus thoracicus in das Blutgefäßsystem}
								\begin{itemize}
									\item Normalerwise bekämpft das Immunsystem alleinschwimmende Zellen
									\item Sentinallymphnode - es wird der Fluss der Lymphe analysiert und die jeweiligen Lymphknoten werden dann ausgeräumt und von Tumorzellen befreit.
									\item Tumorzelle tarnt sich - Lymphe gefüllt mit Abwehrzellen
								\end{itemize}
						\end{itemize}
					\item \textbf{den Blutweg (hämatogen)}
						\begin{itemize}
							\item \textbf{arterieller Typ}
								\begin{itemize}
									\item Primärtumor Lunge, Verschleppung Ateriell im Körper
									\item Tumorzellen nicht gern in stark durchströmten Gefäßen $\rightarrow$ Kapillargewebe + Ansiedelung (Lunge) - (Kein Tumor im Herz)
								\end{itemize}
							\item \textbf{Holvenen Typ}
							\item \textbf{Pfortadertyp}
								\begin{itemize}
									\item Dickdarmkrebs $\rightarrow$ Leber (primär metastasierend) $\rightarrow$ Lunge, hämatogener Weg ist anatomisch vorgegeben
								\end{itemize}
							\item \textbf{vertebraler Typ}
								\begin{itemize}
									\item Tumorzellen aus Prostata in Lendenwirbel. (Ebenfalls Rückschluss von der Metastase auf den Primärtumor möglich)
								\end{itemize}
						\end{itemize}
					\item \textbf{innerhalb der Körperhölen (Absiedelung an Pleura, Peritoneum}
						\begin{itemize}
							\item im Torax - Ausbreitung flächig über die Pleura
						\end{itemize}
					\item \textbf{kanalikulär}
						\begin{itemize}
							\item Nutzung der Tumorzelle von bestehenden Kanälen (Magen/Darm)
						\end{itemize}
					\item \textbf{Knochenmetastasen (indifferent, osteoblastisch, osteoklastisch)}
						\begin{itemize}
							\item Metastasen als Fremdgewebe identifizierbar $\rightarrow$ Ursprung des Tumors diagnostizierbar
							\item osteoblastisch - Verdichten der Knochen - lässt Rückschlüsse auf Primärtumor zu.
						\end{itemize}
				\end{itemize}
			\item Zellen passen jeweils auf den Verband auf. Sobald sich eine daraus löst, wird das Immunsystem aktiviert. Auch werden defekte und fehlerhafte Zellen normalerweise ausgesondert. Beim Tumor erkennt das Immunsystem diesen Fehlerfall nicht (Möglichkeit der Markierung von Tumorzellen)
			\item Bewegung erleichtert das Lösen von Metastasen vom Primärtumor, ebenfalls ist eine iatrogene Manipulation möglich, Operation im Tumorgebiet.
			\item Terrainfaktor: bestimmte Tumore metastasieren in bestimmte Organe - die Tumorzellen werden dort nicht als fremd erkannt.
			\item Frühmetastasen / Spätmetastasen (nach zB. 20 Jahren vermehrt sich die Tumorzelle im Zielgebiet)
			\item Faktoren welche die Metastasierung begünstigen:
				\begin{itemize}
					\item starkes Primärtumorwachstum
					\item viele Nekrosen im Tumorgebiet
					\item Gefäßreichtum eines Tumors
					\item geringer Zusammenhalt des Tumorgewebes
				\end{itemize}
			\item Auch wenn sie sich in Fremdgewebe ansiedeln, müsste das Immunsystem anspringen. Die Tumorzellen adaptieren sich, oder sind autonom um die Kommunikation (Energiehaushalt) zu Nachbarzellen zu verhindern. Tumorentwicklung hängt mit Abwehrschwäche zusammen.
		\end{itemize}
	\subsection{Tumorrezidiv}
		\begin{itemize}
			\item gleicher Tumor an gleicher Stelle
			\item \textbf{entsteht aus liegen gebliebenen Zellen eines unvollständig entfernten Primärtumors}
			\item Tumorzellen schlummern in Lymphspalten (kann Jahre dauern), die Wahrscheinlichkeit ist äquivalent zum Alter des Ersttumors
		\end{itemize}
	\subsection{5-Jahres-Heilungsrate / 5-Jahres-Überlebensrate}
		\begin{itemize}
			\item als Durchschnittswert zu verwenden
			\item Statistisch 5-Jahres-Überlebensrate nur sehr gering, weil es sich hier um schwere Tumore handelt (Lungenkrebs), zu späte Diagnose häufig
			\item \textbf{fünf Jahre nach der Behandlung eines malignen Tumors weder ein Rezidiv noch Metastasen nachweisbar (= Behandlungserfolg) / überlebt}
			\item Nur begrenzt aussagekräftig, auch andere Krankheitsfälle die nichts mit dem Tumor zu tun haben, werden aufgenommen.
		\end{itemize}
	\subsection{Tumorbeurteilung}
		\begin{itemize}
			\item \textbf{typing}
			\item \textbf{staging}
			\item \textbf{grading}
		\end{itemize}
	\subsection{typing (Tumornomenklatur)}
		\begin{itemize}
			\item \textbf{Benennung der Tumore: Endung  "-om"}
			\item \textbf{Bezeichnung nach der Bauart bzw. dem Muttergewebe} (Mutter und Tochtergeschwulst)
				\begin{itemize}
					\item \textbf{Drüsengewebe} (Adeno-om)
					\item \textbf{Bindegewebe} (Fibro-om)
					\item \textbf{Fettgewebe} (Lipom)
					\item \textbf{Muskelgewebe} (Myom)
					\item \textbf{Knorpelgewebe} (Chondrom)
					\item \textbf{Knochengewebe} (Osteom)
					\item \textbf{Blutgefäße} (Ham Angion)
				\end{itemize}
			\item \textbf{Mischtumore: Tumoren mit sowohl epithelialen als auch mesenchymalen Anteilen (z.B. Fibroadenom)} (Deckgewebe, Haut Schleimhaut, Drüsengewebe)
			\item \textbf{maligner epithelialer Tumor: Carcinom} (organspezifisches Gewebe (Muskeln, Knorpel, Blutgefäße, $\dots$)
				\begin{itemize}
					\item Großteil der Tumore sind carcinome Tumore, Epithelgewebe natürlich teilungsfreudig, mechanische Einflüsse, große Flächen
				\end{itemize}
			\item \textbf{maligner mesenchymaler Tumor: Sarkom}
			\item \textbf{Tumore des lymphatischen Systems:}
				\begin{itemize}
					\item \textbf{maligne Lymphome}
						\begin{itemize}
							\item weiße Blutzellen (maligne Leukosen / Entartung $\rightarrow$ Leukämie)
						\end{itemize}
				\end{itemize}
			\item \textbf{Tumore des blutbildenden Systems bzw.Knochenmark:}
				\begin{itemize}
					\item \textbf{maligne Leukosen bzw. Leukämie}
				\end{itemize}
			\item \textbf{Tumore des Nervensystems:}
				\begin{itemize}
					\item \textbf{Gehirnzwischensubstanz (Gliazellen)}
					\item \textbf{Hirnhaut (Meningen)}
					\item \textbf{periphere Nerven (Schwann'sche Zellen)}
				\end{itemize}
			\item \textbf{Tumore des pigmentbildenden Systems}
				\begin{itemize}
					\item \textbf{Nävus} (gutartig, Muttermal)
					\item \textbf{malignes Melanom} (bösartig)
				\end{itemize}
		\end{itemize}
	\subsection{staging: Tumorstadien - Klassifizierung nach dem TNM-System:}
		\begin{itemize}
			\item \textbf{Feststellung der Ausbreitung des Tumorgewebes}
						\begin{itemize}
							\item \textbf{am primären Entstehungsort (Primärtumor = T)}(1-4)
							\item \textbf{Befall der Lymphknoten (Nodulus = N)}(1-3)
							\item \textbf{entferntere Organe (Metastasen = M)}(0-1)
						\end{itemize}
			\item \textbf{wichtig für Therapiewahl und Prognose!}
			\item \textbf{pTNM-Klassifizierung = postoperativ = aussagekräftiger!} (Post-OP Parafinschnitt genauer!)
		\end{itemize}
	\subsection{grading: Beurteilung der Malignität}
		\begin{itemize}
			\item Grad der Naplase, Bsp.: hohe Mitose
			\item \textbf{Grundlage für weitere Therapie und Prognose}
				\begin{itemize}
					\item \textbf{3 (4) Malignitätsgrade}
						\begin{itemize}
							\item \textbf{G1, niedrigste Malignität}
							\item \textbf{G3, oder 4,  höchste Malignität}
						\end{itemize}
					\item \textbf{Kriterien}
						\begin{itemize}
							\item \textbf{gewebliche Entdifferenzierung}
							\item \textbf{Grad der Anaplasie}
							\item \textbf{Wachstumstendenz}
							\item \textbf{(Verhalten zum umliegenden Gewebe)}
						\end{itemize}
				\end{itemize}
		\end{itemize}
	\subsection{Tumorhäufigkeit}
		\begin{itemize}
			\item \textbf{Frauen}
				\begin{itemize}
					\item \textbf{Inzidenz} (=Neuauftreten)
						\begin{itemize}
							\item Brist
							\item Darm
							\item Gebärmutter
							\item Eierstöcke
							\item Magen
							\item Lunge
						\end{itemize}
					\item \textbf{Mortalität}
						\begin{itemize}
							\item Brust
							\item Darm
							\item Lungen
						\end{itemize}
				\end{itemize}
			\item \textbf{Männer}
				\begin{itemize}
					\item \textbf{Inzidenz}
						\begin{itemize}
							\item Prostata
							\item Darm
							\item Lunge
						\end{itemize}
					\item \textbf{Mortalität}
						\begin{itemize}
							\item Lungenkrebs
							\item Darm
							\item Prostata
						\end{itemize}
				\end{itemize}
			\item Durchbruch noch nicht gelungen. Man versteht die Ursachen noch lange nicht.
		\end{itemize}
	\subsection{Folgen maligner Neoplasmen:}
		\begin{itemize}
			\item \textbf{lokal}
				\begin{itemize}
					\item \textbf{Organfunktionsstörungen}
					\item \textbf{Stenosen oder Verschluss von Hohlorganen}
						\begin{itemize}
							\item Stenose - Einengung von Öffnungen
							\item Nekrose - Zerstörung von Gewebe
							\item brauchen Platz und Energie, sind parasitär
						\end{itemize}
					\item \textbf{Tumornekrosen}
				\end{itemize}
			\item \textbf{allgemein:}
				\begin{itemize}
					\item \textbf{Tumorkachexie}
					\item \textbf{Fieber}
					\item \textbf{Tumoranämie}
						\begin{itemize}
							\item Zerstörung von Blutzellen (weiß, rot) $\rightarrow$ Depression, Infektanfälligkeit
						\end{itemize}
					\item \textbf{Infektanfälligkeit und herabgesetzte Immunabwehr}
					\item \textbf{endokrine Effekte bei endokrin-aktiven Tumoren}
						\begin{itemize}
							\item hormonelle Aktivität, Lungenkrebs erhöhen Kalziumspiegel im Blut
						\end{itemize}
					\item \textbf{paraneoplastische Syndrome}
				\end{itemize}
		\end{itemize}
	\subsection{Todesursachen bei malignen Tumoren}
		\begin{itemize}
			\item \textbf{Zerstörung lebenswichtiger Organe}
			\item \textbf{akute oder chronische Blutungen}
			\item \textbf{Verschluss wichtiger Hohlorgane}
			\item \textbf{Infektion}
			\item \textbf{Metastasierung in lebenswichtige Organe}
			\item \textbf{Herzversagen}
				\begin{itemize}
					\item zusätzliches Gewebe benötigt zusätzliche Durchblutung
				\end{itemize}
			\item \textbf{Tumorkachexie}
				\begin{itemize}
					\item auf Kosten des gesunden Gewebes $\rightarrow$ Ausgezehrtheit
				\end{itemize}
		\end{itemize}
\pagebreak
	\subsection{Ätiologie maligner Neoplasmen}
		\begin{itemize}
			\item \textbf{Krebsentstehung: Zusammenwirken verschiedener krebserregender Faktoren}
			\item \textbf{endogene Ursachen}
				\begin{itemize}
					\item \textbf{genetische Faktoren}
						\begin{itemize}
							\item ~5\% 
							\item Bsp1.: familiäre Dickdarm-Adenomatose = fam. Polyposis
								\begin{itemize}
									\item Polypen: gutartige Tumore, aus denen mit d. Zeit bösartige entstehen können, treten im Alter einzeln auf, werden häufig Kontrolliert und ggf. entfernt
									\item bei Polyposis: hunderte mit hohem Entartungsrisiko, in kurzer Zeit, auch in jungen Jahren! Engmaschige Kontrollen, ggf. operative Entfernung der betroffenen Dickdarm-Teile (muss zu viel entfernt werden, kann nicht mehr ausreichend eingedickt werden $\rightarrow$ künstlicher Ausgang = Stoma) 
									\item Fehlen eines Tumor-Suppressor Gens (erstes entdecktes Tumor-Suppressor-Gen: p53Gen)
								\end{itemize}
							\item Bsp2.: Gendefekt-verursachtes Mamma-Karzinom (sehr selten): sehr hohes Risiko, Angebot der präventiven Brust-Amputation
						\end{itemize}
					\item  \textbf{hormonelle Faktoren} \\
						zB.: Prostata Carcinom (Details folgen)						
					\item	\textbf{chronische Gewebereizung} \\
						Chronisch gereiztes Gewebe hat höheres Karzinom Risiko\\
						zB.: chronische Entzündung, schlecht sitzende Implantate 
				\end{itemize}
			\item \textbf{exogene Ursachen}
				\begin{itemize}
					\item \textbf{chemische Faktoren}
						\begin{itemize}
							\item häufigste Ursache
							\item bei geringer Dosis kann es durchaus lange dauern bis Auftreten, aber: Dosisakkumulation!
							\item Beispiele für chemische Verbindungen
								\begin{itemize}
									\item Benzidin, Anilin $\rightarrow$ Harnblasencarcinom
									\item Benzpyren, polyzyklische Wasserstoffe $\rightarrow$ Hautcarcinom
									\item versch. Substanzen $\rightarrow$ Lebercarcinom
										(zB Schimmelpilz im Getreide $\rightarrow$ Aflatoxin) 
									\item Arsen/Chrom Verbindungen
									\item Asbest, Nickel \& Holzstaub $\rightarrow$ Lungen und Nasennebenhöhlen
									\item Asbest $\rightarrow$ Pleuramesotheliom
									\item Nitrosamine, in gepökeltem/verbranntem Fleisch $\rightarrow$ Magen\\
										(daher in Tirol \& Vorarlberg höher wegen Speck, Japan durch geräucherten gepökeltem Fisch)
									\item Tabak $\rightarrow$ Mundhöhle, Lunge, Kehlkopf, Speiseröhre (meist Alkohol+Nikotin), Harnblase, Lippencarcinom (betrifft auch Zigarrenraucher - ohne Inhalation)
									\item Hormone:\\
										\mbox{}\quad ' Androgene: doping – Leber\\
										\mbox{}\quad ' Pille - geringe Erhöhung gutartiger Lebertumore, aber deutliche Senkung d. \\
										\mbox{}\quad\:  Ovarialencarzinome
								\end{itemize}
						\end{itemize}
					\item \textbf{physikalische Faktoren}
						\begin{itemize}
							\item Radioaktive Strahlung
								\begin{itemize}
									\item[$\rightarrow$] Plattenepidelkarz. an Händen durch ungeschützten, direkten Kontakt (z.B. erste Radiologie-Forscher, Hiroshima, Nagasaki, Tschernobyl: DNA-Schädigung $\rightarrow$ Leukämien, Schildrüsencarcinom)
								\end{itemize}
							\item UV-Strahlung: DNA-Schädigung
								\begin{itemize}
									\item[$\rightarrow$] Plattenepidelcarzinom, Melanom (maligner Hauttumor), Basaliom (Haut "merkt" sich Schädigung, muss nach UV-Einstrahlung Reparaturmaßnahmen durchführen. $\rightarrow$ bei zu viel UV-Einwirkung überfordert)
									\item Melanom: genetische Veranlagung, eventuell Viren u.a. unbekannte Einflüsse. Auch bei jungen Erwachsenen möglich
								\end{itemize}		  
						\end{itemize}
\pagebreak	
					\item \textbf{infektiöse Faktoren} (onkogene Viren, selten Alleinauslösende Faktoren)
						\begin{itemize}
							\item Humanes Papillomavirus: Warzen an Haut u. Genitalien, deutlich erhöhtes 
								\begin{itemize}
									\item Cervixkarzinomrisiko (Impfung gegen die häufigsten Arten, kostspielig!)
									\item STD! durch oralen Verkehr: Larynxkarzinomrisiko $\dagger$
								\end{itemize}
							\item Herpes-Simplex-Virus (HSV) \emph{Typ2}: genitaler Herpes $\rightarrow$ Cervixkarzinomrisiko $\dagger$
							\item Epstein-Barr-Virus: Preiffer'sches Drüsenfieber = Mononukleose
								\begin{itemize}
									\item engl. umgs. kissing disease
									\item (sichtbare) Schwellung der Hals-Lymphknoten
									\item meist komplikationslose Erkrankung i.d. Pubertät, aber: erhöhtes Risiko für maligne Lymphome
								\end{itemize}
						\end{itemize}
					\item \textbf{Ernährung}
						\begin{itemize}
							\item Nitrosamine, Ballaststoffe, tierische Fette? $\dots$
						\end{itemize}
				\end{itemize}
		\end{itemize}
	\subsection{Onkogenese (Karziogenese)}
		\begin{itemize}
			\item \textbf{immunologische Reaktion des Wirtsorganismus}
				\begin{itemize}
					\item Immun-Überwachungs-Theorie: fehlende immunologische Reaktion des Wirtsorganismus auf entartete Tumorzellen
				\end{itemize}
			\item \textbf{Tumorwachstum: Zellkommunikationsstörung}
				\begin{itemize}
					\item Zellkommunikationsstörung $\rightarrow$ Tumorwachstum
				\end{itemize}
			\item \textbf{Tumor-Angiogenese-Faktor: ausreichende Blutversorgung ist für das Tumorwachstum essentiell}
			\item \textbf{Invasion und Metastatsierung: verminderter interzellurärer Zusammenhalt}\\
				zB. Tarnung als Thrombus
		\end{itemize}
	\subsection{Diagnostik: Tumormarker}
		\begin{itemize}
			\item \textbf{im Blut messbare Substanzen, die mit malignem
				Tumorgewebe korrelieren \emph{können}}
			\item aber: \textbf{nicht tumorspezifisch, nicht organspezifisch}
			\item \textbf{Nachweis teilweise bis zu Grenzwert normal}
			\item \textbf{daher v.a. für postoperative Verlaufskontrolle}\\
				Vergleich mit pre-OP Wert
			\item \textbf{Beispiele:}
				\begin{itemize}
					\item \textbf{AFP} Alpha Feto Protein, \textbf{CEA} Carcino Enbryonales Antigen:
						\begin{itemize}
							\item Bei Embryos vorhanden, gehen m.d.Z. verloren, bilden sich bei Erkrankung neu
							\item Bsp: Dickdarmcarzinom
						\end{itemize}
					\item \textbf{HCG} Humanes Choriongonadotropin (von Tumorzellen erzeugte Hormone)
						\begin{itemize}
							\item wird auch an Beginn der Schwangerschaft gebildet (Schwangerschaftstest!)
							\item gut verwertbar beim Mann $\rightarrow$ Hodentumor
						\end{itemize}
					\item \textbf{Calcitonin}: Kann mit Schilddrüsenkarzinom korrelieren
					\item Enzyme: \textbf{PSA} Enzyme: Prostata Spezifisches Antigen, \textbf{PAP} Prostatic Acid Phosphatase (Indikator erst ab physiologischem Schwellwert)
				\end{itemize}
		\end{itemize}
	\subsection{Behandlung}
		\begin{itemize}
			\item \textbf{Operation}
			\item \textbf{Radiotherapie}
				\begin{itemize}
					\item \textbf{Zelltod duch ionisierende Stahlung, präoperative und/oder postoperative Bestrahlung}
					\item auch pre-OP, verkleinert den Tumor, zerstört besonders aktive Zellen, verringert OP-bedingtes Streuungsrisiko
					\item Stahlen und Chemotherapie zerstört alles teilungsfreudige Gewebe $\rightarrow$ enorme Nebenwirkungen
					\item keine neuen Blutzellen, Schleimhäute, Haarausfall, Hautschäden, Zerfall vieler Zellen
					\item verursacht Gewebeschäden - gesundes Gewebe wird verletzt
				\end{itemize}					
			\item \textbf{Chemotherapie mit Zytostatika}
				\begin{itemize}
					\item Gewebetoxisch - Zerstören die Wände von kleinen Venen
				\end{itemize}
		\end{itemize}
	\subsection{neuere Methoden}
		\begin{itemize}
			\item \textbf{monoklonale Antikörper}
				\begin{itemize}
				\item weist der Tumor Antigenkörper auf, kann man Antikörper geben, die Tumorzellen zerstören sollen
				\end{itemize}
			\item \textbf{dendritische Zelltherapie}
			\item \textbf{Hyperthermie}
			\item \textbf{Neutronenstrahlung}
			\item \textbf{$\dots$}
		\end{itemize}
	\subsection{Nebenwirkungen}
			\begin{itemize}
				\item auch gesunde Zellen in Teilung werden vorübergehend zerstört. Vor allem
					\begin{itemize}
						\item Haut- und Schleimhautzellen, Haare $\rightarrow$ Haarausfall
						\item Blutzellen
							\begin{itemize}
								\item[$\rightarrow$] Erythrozytenmangel $\rightarrow$ Anämie (Schwäche, depressive Verstimmung, …)
								\item[$\rightarrow$] Leukozytenmangel $\rightarrow$ Schwächung d. Immunsystems $\rightarrow$ mangelnde Abwehr, Infektanfälligkeit
							\end{itemize}
					\end{itemize} 
				\item \textbf{sekundäre Neoplasien}
					\begin{itemize}
						\item ionisierende Strahlung ist cancerogen (Bsp. 10-20 Jahre später)
					\end{itemize}
				\item \textbf{Knochenmarkschädigung}
				\item \textbf{gastrointestinale Nebenwirkungen}
				\item \textbf{Haarausfall (Alopezie)}
				\item \textbf{Hyperpigmentierung der Haut}
				\item \textbf{Fieber, Schüttelfrost, depressive Verstimmung}
				\item \textbf{Organschäden (Leber, Niere, Lunge, Herz, Muskulatur, Nerven)}
				\item \textbf{lokale Gewebstoxizität}			
			\end{itemize}
	
\pagebreak
\subsection{einzelne Tumorbeispiele}
	\begin{itemize}
		\item \textbf{Malignes Melanom}
			\begin{itemize}
			\item Abgrenzung zum benignen Naevus (Muttermal)\\
				ABCD(E)-Regel
					\begin{itemize}
						\item Asymmetrie
						\item Begrenzung
						\item Colour
						\item Durchmesser
						\item (Erhaben)
					\end{itemize}
			\end{itemize}
		\item \textbf{Basaliom}
			\begin{itemize}
				\item "semimaligne" = borderline, lokal malignes Wachstum aber keine Metastasierung!
			\end{itemize}
		\item \textbf{Leukämien}
			\begin{itemize}
				\item Einteilung
					\begin{itemize}					
						\item akute Leukämie
							\begin{itemize}
								\item 90\% Leukämien im Kindesalter
								\item myeloische ($\rightarrow$ akute meloische Leukämie)
								\item lymphatische ($\rightarrow$ akute lymphatische Leukämie)
							\end{itemize}
						\item chronische Leukämie
							\begin{itemize}
								\item myeloische ($\rightarrow$ chronische myeloische Leukämie)
								\item lymphatische ($\rightarrow$ chronische lymphatische Leukämie)
							\end{itemize}
						\item[$\rightarrow$] Mangel an Erythrozyten = Anämie
						\item[$\rightarrow$] Mangel an Trombozyten $\rightarrow$ Blutgerinnungsproblem, Spontanblutungen
						\item[$\rightarrow$] Mangel an Leukozyten $\rightarrow$ Abwehrschwäche, Infektanfälligkeit
					\end{itemize}
			\end{itemize}
		\item \textbf{maligne Lymphome}
			\begin{itemize}
				\item M(orbus)-Hodgkin-Lymphom
					\begin{itemize}
						\item geht von B-Lympozyten aus
						\item Symptome
							\begin{itemize}
								\item Nachtschweiß, Gewichtsverlust, evtl. Fieber
								\item erhöhte BSG, Blutsenkungsgeschwindigkeit (später mehr)
								\item manchmal Schmerzen/Juckreiz nach Alkoholkonsum
							\end{itemize}
						\item Staging:
							\begin{itemize}
								\item Eine, zwei, mehrere Knoten befallen
								\item Behandlung: Strahlen \& Chemo
							\end{itemize}
						\item Prognose:
							\begin{itemize}
								\item bei Früherkennung 70\% Überlebensrate
							\end{itemize}
					\end{itemize}
				\item Non-Hodgkin-Lymphom
			\end{itemize}		
		\item \textbf{Hodencarcinom}
			\begin{itemize}
				\item Altersgipfel: 20-30
				\item überwiegend von Keimzellen ausgehend $\rightarrow$ Keimzellentumore (häufigster maligner Tumor bei jungen Männern)
				\item Ätiologie: risikoerhöhend: Hoden zum Zeitpunkt der Geburt nicht im Skrotum (noch in Bauchhöhle)
			\end{itemize}
		\item \textbf{Prostatacarcinom}
			\begin{itemize}
				\item überwiegend ältere ältere Männer
				\item durch Abfall von Testosteron relativer Anstieg von Östrogen $\rightarrow$ Wachstumsstimulus für Prostata
				\item Therapie:
					\begin{itemize}
						\item OP (möglichst Nerven-schonend! aber: höheres Risiko, nicht alle Carcinom-Anteile zu entfernen!)
						\item Hormontherapie: anti-androgen (Nebenwirkung: „Verweiblichung“ $\rightarrow$ z.B. Brustdrüsenwachstum)
					\end{itemize}
			\end{itemize}
		\pagebreak
		\item \textbf{Mammacarcinom}
			\begin{itemize}
				\item Insidenz nimmt stetig zu, zZt. jede 8. Frau
				\item Lokalisation meist obere Hälfte
				\item Risikofaktoren
					\begin{itemize}
						\item genetische Veranlagung
						\item Östrogene
							\begin{itemize}
								\item frühe Menarche (erste Regelblutung)
								\item späte Monopause
								\item Östrogentherapie i.d. Menopause
								\item Keine Schwangerschaften (Schwangerschaft+Stillzeit unterbricht Zyklus)
								\item Adipositas
							\end{itemize}
					\end{itemize}
				\item mit dem Alter deutlich Ansteigend nach 50
				\item gute Prognose
				\item Behandlung
					\begin{itemize}
						\item Operation
							\begin{itemize}
								\item Entfernung von Lymphknoten in der Achsel, wird kaum noch durchgeführt
							\end{itemize}
						\item kosmetische Restauration
					\end{itemize}
			\end{itemize}
		\item \textbf{Cervixcarcinom}
		\item \textbf{Coloncarcinom}
			\begin{itemize}
				\item Insidenz nimmt stetig zu, vermutlich auf Grund von Lebensweise
				\item möglicherweise Ernärung und Genetik Ursachen\\
					(cancerogene Lebensmittel bleiben länger im Colon durch balaststoffarme Ernärung)
				\item 90\% entwickeln sich aus malignen Polypen\\
					  Vorsorgliche Spiegelung im höheren Alter, Entfernung und Analyse der Polypen (Poloskopie)
				\item Therapie
					\begin{itemize}
						\item Im Krankheitsfall Chemotherapie und Operation, nicht aber Bestrahlung (Beschädigung der umliegenden Organe)
					\end{itemize}
				\item Metastasen $\rightarrow$ Leber $\rightarrow$ Lunge 
				\item 90\% Überlebensrate bei rechtzeitiger Behandlung
			\end{itemize}
	\end{itemize}
\subsection{Einschub: Erethrozyten}
	Blut
		\begin{itemize}
			\item Flüssigkeit = Plasma (Serum: ohne Gerinnung)
			\item Zellen
				\begin{itemize}
					\item Erythrozyten (Hämoglobin, O2-Transport, ABO-System, Rh-System
					\item Thrombozyten: Gerinnselbildung (Thrombus) zur Gefäßwandabdichtung
					\item Leukozyten
						\begin{itemize}
							\item Granulozyten
							\item Monozyten	
							\item Lymphozyten
								\begin{itemize}
									\item T(hymus)-Lymphozyten
									\item B(one marrow)-Lymphozyten
									\item NK-Zellen
								\end{itemize}
						\end{itemize}	
				\end{itemize}
		\end{itemize}