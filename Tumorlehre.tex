\section{Tumorlehre}
	\subsection{Grundlagen}
		\begin{itemize}
			\item \textbf{Tumor ("onkos") = Schwellung = Volumenzunahme}
			\item \textbf{synonym: Neoplasma ("Neubildung"), Blastom}
				\begin{itemize}
					\item \textbf{Neubildung körpereigenen Gewebes mit autonomer Wachstumstendenz, die jene eines normalen Gewebes weit übersteigt}
					\item \textbf{Wachstum ist unkontrolliert und überschießend, auf Kosten der gesunden Zellen (Nährstoffentzug!)}
					\item \textbf{Wachstum wird auch nach Wegfall der auslösenden Ursache nicht eingestellt}
						\begin{itemize}
							\item \textbf{vgl.Hypertrophie/-plasie: Wachstum vom auslösenden Reiz abhängig, reversibel}
						\end{itemize}
				\end{itemize}
			\item \textbf{Dignität: es gibt gutartige (benigne) und bösartige (maligne Tumore:)}
				\begin{center}
					\begin{tabular}{|ll|}
						\hline
						\textbf{benigne Tumore} & \textbf{maligne Tumore} \\
						\hline
						\multicolumn{2}{|c|}{\textbf{Wachstum}} \\ 
						langsam & rasch \\ 
						scharf begrenzt (Kapsel) & unscharf begrenzt \\ 
						expansiv, verdrängend & infiltrativ, eindringend \\
						komprimierend & destruierend \\
						eher verschieblich & nicht verschieblich \\
						\multicolumn{2}{|c|}{\textbf{Zellen}} \\
						keine Zellatypien & Zellatypien \\
						differenziert & undifferenziert \\
						reif & unreif \\
						\multicolumn{2}{|c|}{\textbf{Metastasen}} \\
						keine Metastasen & bildet Metastasen \\
						\multicolumn{2}{|c|}{\textbf{Verlauf}} \\
						geringe Allgemeinstörung & starke Allgemeinstörung \\
						wenig Rezidive & oft Rezidive \\
						meist keine direkte Lebensgefahr & meist hohe Lebensgefahr\\
						\hline
					\end{tabular}
				\end{center}
		\end{itemize}
	\subsection{borderline lesions (=semimaligne Tumore)}
		\begin{itemize}
			\item \textbf{wachsen bösartig (lokal infiltrativ und destruierend)}
			\item \textbf{hochgradige Rezidivneigung}
			\item \textbf{metastasierend, jedoch sehr selten und sehr spät}
		\end{itemize}
	\subsection{Präkanzerosen}
		\begin{itemize}
			\item \textbf{Gewebsveränderungen mit erhöhtem Risiko der malignen Entartung}
			\item \textbf{fakultative Präkanzerose: Entartungsrisiko $<$ 20\%, Dauer $>$ 5 Jahre}
			\item \textbf{obligate Präkanzerose: Entartungsrisiko $>$ 20\%, Dauer $<$ 5 Jahre}
		\end{itemize}
	\subsection{Metastasen}
		\begin{itemize}
			\item \textbf{Absiedlungen (Tochtergeschwülste) vom Primärtumor(Muttergeschwulst) über}
				\begin{itemize}
					\item \textbf{den Lymphweg (lymphogen)}
						\begin{itemize}
							\item \textbf{regionale Lymphknoten - weitere Lymphknotengruppen - über den Ductus thoracicus in das Blutgefäßsystem}
						\end{itemize}
					\item \textbf{den Blutweg (hämatogen)}
						\begin{itemize}
							\item \textbf{arterieller Typ}
							\item \textbf{Holvenen Typ}
							\item \textbf{Pfortadertyp}
							\item \textbf{vertebraler Typ}
						\end{itemize}
					\item \textbf{innerhalb der Körperhölen (Absiedelung an Pleura, Peritoneum}
					\item \textbf{kanalikulär}
					\item \textbf{Knochenmetastasen (indifferent, osteoblastisch, osteoklastisch)}
				\end{itemize}
		\end{itemize}
	\subsection{Tumorrezidiv}
		\begin{itemize}
			\item \textbf{entsteht aus liegen gebliebenen Zellen eines unvollständig entfernten Primärtumors}
		\end{itemize}
	\subsection{5-Jahres-Heilungsrate / 5-Jahres-Überlebensrate}
		\begin{itemize}
			\item \textbf{fünf Jahre nach der Behandlung eines malignen Tumors weder ein Rezidiv noch Metastasen nachweisbar (= Behandlungserfolg) / überlebt}
		\end{itemize}
	\subsection{Tumorbeurteilung}
		\begin{itemize}
			\item \textbf{typing}
			\item \textbf{staging}
			\item \textbf{grading}
		\end{itemize}
	\subsection{typing (Tumornomenklatur)}
		\begin{itemize}
			\item \textbf{Benennung der Tumore: Endung  "-om"}
			\item \textbf{Bezeichnung nach der Bauart bzw. dem Muttergewebe}
				\begin{itemize}
					\item \textbf{Drüsengewebe}
					\item \textbf{Bindegewebe}
					\item \textbf{Fettgewebe}
					\item \textbf{Muskelgewebe}
					\item \textbf{Knorpelgewebe}
					\item \textbf{Knochengewebe}
					\item \textbf{Blutgefäße}
				\end{itemize}
			\item \textbf{Mischtumore: Tumoren mit sowohl epithelialen als auch mesenchymalen Anteilen (z.B. Fibroadenom)}
			\item \textbf{maligner epithelialer Tumor: Carcinom}
			\item \textbf{maligner mesenchymaler Tumor: Sarkom}
			\item \textbf{Tumore des lymphatischen Systems:}
				\begin{itemize}
					\item \textbf{maligne Lymphome}
				\end{itemize}
			\item \textbf{Tumore des blutbildenden Systems bzw.Knochenmark:}
				\begin{itemize}
					\item \textbf{maligne Leukosen bzw. Leukämie}
				\end{itemize}
			\item \textbf{Tumore des Nervensystems:}
				\begin{itemize}
					\item \textbf{Gehirnzwischensubstanz (Gliazellen)}
					\item \textbf{Hirnhaut (Meningen)}
					\item \textbf{periphere Nerven (Schwann'sche Zellen)}
				\end{itemize}
			\item \textbf{Tumore des pigmentbildenden Systems}
				\begin{itemize}
					\item \textbf{Nävus}
					\item \textbf{malignes Melanom}
				\end{itemize}
		\end{itemize}
	\subsection{staging: Tumorstadien - Klassifizierung nach dem TNM-System:}
		\begin{itemize}
			\item \textbf{Feststellung der Ausbreitung des Tumorgewebes}
						\begin{itemize}
							\item \textbf{am primären Entstehungsort (Primärtumor = T)}
							\item \textbf{Befall der Lymphknoten (Nodulus = N)}
							\item \textbf{entferntere Organe (Metastasen = M)}
						\end{itemize}
			\item \textbf{wichtig für Therapiewahl und Prognose!}
			\item \textbf{pTNM-Klassifizierung = postoperativ = aussagekräftiger!}
		\end{itemize}
	\subsection{grading: Beurteilung der Malignität:}
		\begin{itemize}
			\item \textbf{Grundlage für weitere Therapie und Prognose}
				\begin{itemize}
					\item \textbf{3 (4) Malignitätsgrade:}
						\begin{itemize}
							\item \textbf{G 1 niedrigste Malignität}
							\item \textbf{G 3 oder 4  höchste Malignität}
						\end{itemize}
					\item \textbf{Kriterien:}
						\begin{itemize}
							\item \textbf{gewebliche Entdifferenzierung}
							\item \textbf{Grad der Anaplasie}
							\item \textbf{Wachstumstendenz}
							\item \textbf{(Verhalten zum umliegenden Gewebe)}
						\end{itemize}
				\end{itemize}
		\end{itemize}
	\subsection{Tumorhäufigkeit:}
		\begin{itemize}
			\item \textbf{Frauen}
				\begin{itemize}
					\item \textbf{Inzidenz}
					\item \textbf{Mortalität}
				\end{itemize}
			\item \textbf{Männer}
				\begin{itemize}
					\item \textbf{Inzidenz}
					\item \textbf{Mortalität}
				\end{itemize}
		\end{itemize}
	\subsection{Folgen maligner Neoplasmen:}
		\begin{itemize}
			\item \textbf{lokal}
				\begin{itemize}
					\item \textbf{Organfunktionsstörungen}
					\item \textbf{Stenosen oder Verschluss von Hohlorganen}
					\item \textbf{Tumornekrosen}
				\end{itemize}
			\item \textbf{allgemein:}
				\begin{itemize}
					\item \textbf{Tumorkachexie}
					\item \textbf{Fieber}
					\item \textbf{Tumoranämie}
					\item \textbf{Infektanfälligkeit und herabgesetzte Immunabwehr}
					\item \textbf{endokrine Effekte bei endokrin-aktiven Tumoren}
					\item \textbf{paraneoplastische Syndrome}
				\end{itemize}
		\end{itemize}
	\subsection{Todesursachen bei malignen Tumoren}
		\begin{itemize}
			\item \textbf{Zerstörung lebenswichtiger Organe}
			\item \textbf{akute oder chronische Blutungen}
			\item \textbf{Verschluss wichtiger Hohlorgane}
			\item \textbf{Infektion}
			\item \textbf{Metastasierung in lebenswichtige Organe}
			\item \textbf{Herzversagen}
			\item \textbf{Tumorkachexie}
		\end{itemize}
	\subsection{Ätiologie maligner Neoplasmen}
		\begin{itemize}
			\item \textbf{Krebsentstehung: Zusammenwirken verschiedener krebserregender Faktoren}
			\item \textbf{endogene Ursachen}
				\begin{itemize}
					\item \textbf{genetische Faktoren}
						\begin{itemize}
							\item ~5\% 
							\item Bsp1.: familiäre Dickdarm-Adenomatose = fam. Polyposis
								\begin{itemize}
									\item Polypen: gutartige Tumore, aus denen mit d. Zeit bösartige entstehen können, treten im Alter einzeln auf, werden häufig Kontrolliert und ggf. entfernt
									\item bei Polyposis: hunderte mit hohem Entartungsrisiko, in kurzer Zeit, auch in jungen Jahren! Engmaschige Kontrollen, ggf. operative Entfernung der betroffenen Dickdarm-Teile (muss zu viel entfernt werden, kann nicht mehr ausreichend eingedickt werden $\rightarrow$ künstlicher Ausgang = Stoma) 
									\item Fehlen eines Tumor-Suppressor Gens (erstes entdecktes Tumor-Suppressor-Gen: p53Gen)
								\end{itemize}
							\item Bsp2.: Gendefekt-verursachtes Mamma-Carzinom (sehr selten): sehr hohes Risiko, Angebot der präventiven Brust-Amputation
						\end{itemize}
					\item  \textbf{hormonelle Faktoren} \\
						zB.: Prostata Carcinom (Details folgen)						
					\item	\textbf{chronische Gewebereizung} \\
						Chronisch gereiztes Gewebe hat höheres Karzinom Risiko\\
						zB.: chronische Entzündung, schlecht sitzende Implantate 
				\end{itemize}
			\item \textbf{exogene Ursachen}
				\begin{itemize}
					\item \textbf{chemische Faktoren}
						\begin{itemize}
							\item häufigste Ursache
							\item bei geringer Dosis kann es durchaus lange dauern bis Auftreten, aber: Dosisakkumulation!
							\item Beispiele für chem. Verbindungen:
								\begin{itemize}
									\item Benzidin, Anilin $\rightarrow$ Harnblasencarcinom
									\item Benzpyren, polyzyklische Wasserstoffe $\rightarrow$ Hautcarcinom
									\item versch. Substanzen $\rightarrow$ Lebercarcinom
										(zB Schimmelpilz im Getreide $\rightarrow$ Aflatoxin) 
									\item Arsen/Chrom Verbindungen
									\item Asbest, Nickel \& Holzstaub $\rightarrow$ Lungen und Nasennebenhöhlen
									\item Asbest $\rightarrow$ Pleuramesotheliom
									\item Nitrosamine, in gepökeltem/verbranntem Fleisch $\rightarrow$ Magen\\
										(daher in Tirol \& Vorarlberg höher wegen Speck, Japan durch geräucherten gepökeltem Fisch)
									\item Tabak $\rightarrow$ Mundhöhle, Lunge, Kehlkopf, Speiseröhre (meist Alkohol+Nikotin), Harnblase, Lippencarcinom (betrifft auch Zigarrenraucher - ohne Inhalation)
									\item Hormone:\\
										\mbox{}\quad ' Androgene: doping – Leber\\
										\mbox{}\quad ' Pille - geringe Erhöhung gutartiger Lebertumore, aber deutliche Senkung d. \\
										\mbox{}\quad\:  Ovarialencarzinome
								\end{itemize}
						\end{itemize}
					\item \textbf{physikalische Faktoren}
						\begin{itemize}
							\item Radioaktive Strahlung
								\begin{itemize}
									\item[$\rightarrow$] Plattenepidelkarz. an Händen durch ungeschützten, direkten Kontakt (z.B. erste Radiologie-Forscher, Hiroshima, Nagasaki, Tschernobyl: DNA-Schädigung $\rightarrow$ Leukämien, Schildrüsencarcinom)
								\end{itemize}
							\item UV-Strahlung: DNA-Schädigung
								\begin{itemize}
									\item[$\rightarrow$] Plattenepidelcarzinom, Melanom (maligner Hauttumor), Basaliom (Haut "merkt" sich Schädigung, muss nach UV-Einstrahlung Reparaturmaßnahmen durchführen. $\rightarrow$ bei zu viel UV-Einwirkung überfordert)
									\item Melanom: genetische Veranlagung, eventuell Viren u.a. unbekannte Einflüsse. Auch bei jungen Erwachsenen möglich
								\end{itemize}		  
						\end{itemize}	
					\item \textbf{infektiöse Faktoren}\\
						 onkogene Viren (selten Alleinauslösende Faktoren):
						\begin{itemize}
							\item Humanes Papillomavirus: Warzen an Haut u. Genitalien, deutlich erhöhtes 
								\begin{itemize}
									\item Cervixkarzinomrisiko (Impfung gegen die häufigsten Arten, kostspielig!)
									\item STD! durch oralen Verkehr: Larynxkarzinomrisiko $\dagger$
								\end{itemize}
							\item Herpes-Simplex-Virus (HSV) \textbf{Typ2}: genitaler Herpes $\rightarrow$ Cervixkarzinomrisiko $\dagger$
							\item Epstein-Barr-Virus: Preiffer'sches Drüsenfieber = Mononukleose
								\begin{itemize}
									\item engl. umgs. kissing disease
									\item (sichtbare) Schwellung der Hals-Lymphknoten
									\item meist komplikationslose Erkrankung i.d. Pubertät, aber: erhöhtes Risiko für maligne Lymphome
								\end{itemize}
						\end{itemize}
					\item \textbf{Ernährung}
						\begin{itemize}
							\item Nitrosamine, Ballaststoffe, tierische Fette? $\dots$
						\end{itemize}
				\end{itemize}
		\end{itemize}
	\subsection{Onkogenese (Karziogenese)}
		\begin{itemize}
			\item \textbf{immunologische Reaktion d. Wirtsorganismus}
				\begin{itemize}
					\item Immun-Überwachungs-Theorie: fehlende immunologische Reaktion d. Wirtsorganismus auf entartete Tumorzellen
				\end{itemize}
			\item \textbf{Tumorwachstum: Zellkommunikationsstörung}
				\begin{itemize}
					\item Zellkommunikationsstörung $\rightarrow$ Tumorwachstum
				\end{itemize}
			\item \textbf{Tumor-Angiogenese-Faktor: ausreichende Blutversorgung ist für das Tumorwachstum essentiell}
			\item \textbf{Invasion und Metastatsierung: verminderter interzellurärer Zusammenhalt}\\
				zB. Tarnung als Thrombus
		\end{itemize}
	\subsection{Diagnostik: Tumormarker}
		\begin{itemize}
			\item \textbf{im Blut messbare Substanzen, die mit malignem
				Tumorgewebe korrelieren \emph{können}}
			\item aber: \textbf{nicht tumorspezifisch, nicht organspezifisch}
			\item \textbf{Nachweis teilweise bis zu Grenzwert normal}
			\item \textbf{daher v.a. für postoperative Verlaufskontrolle}\\
				Vergleich mit pre-OP Wert
			\item \textbf{Beispiele:}
				\begin{itemize}
					\item \textbf{AFP} Alpha Feto Protein, \textbf{CEA} Carcino Enbryonales Antigen:
						\begin{itemize}
							\item Bei Embryos vorhanden, gehen m.d.Z. verloren, bilden sich bei Erkrankung neu
							\item Bsp: Dickdarmcarzinom
						\end{itemize}
					\item \textbf{HCG} Humanes Choriongonadotropin (von Tumorzellen erzeugte Hormone)
						\begin{itemize}
							\item wird auch an Beginn d. Schwangerschaft gebildet (Schwangerschaftstest!)
							\item gut verwertbar beim Mann $\rightarrow$ Hodentumor
						\end{itemize}
					\item \textbf{Calcitonin}: Kann mit Schilddrüsenkarzinom korrelieren
					\item Enzyme: \textbf{PSA} Enzyme: Prostata Spezifisches Antigen, \textbf{PAP} Prostatic Acid Phosphatase (Indikator erst ab physiologischem Schwellwert)
				\end{itemize}
		\end{itemize}
	\subsection{Behandlung}
		\begin{itemize}
			\item \textbf{Operation}
			\item \textbf{Radiotherapie}
				\begin{itemize}
					\item \textbf{Zelltod durch ionisierende Strahlung, präoperative oder/und postoperative Bestrahlung}
				\end{itemize}
			\item \textbf{Chemotherapie mit Zytostatika}
			\item \textbf{neuere Methoden}
				\begin{itemize}
					\item \textbf{monoklonale Antikörper}
					\item \textbf{dendritische Zelltherapie}
					\item \textbf{Hyperthermie}
					\item \textbf{Neutronenstrahlung}
					\item \textbf{$\dots$}
				\end{itemize}
		\end{itemize}
	\subsection{möglich unerwünschte NW}
		\begin{itemize}
			\item \textbf{Knochenmarkschädigung}
			\item \textbf{gastrointestinale NW}
			\item \textbf{Haarausfall (Alopezie)}
			\item \textbf{Hyperpigmentierung der Haut}
			\item \textbf{Fieber, Schüttelfrost, depressive Verstimmung}
			\item \textbf{Organschäden (Leber, Niere, Lunge, Herz, Muskulatur, Nerven)}
			\item \textbf{lokale Gewebstoxizität}
			\item \textbf{sekundäre Neoplasien}
		\end{itemize}
	\subsection{einzelne Tumorbeispiele}
		\begin{itemize}
			\item \textbf{Basaliom}
			\item \textbf{malignes Melanom}
			\item \textbf{Leukämien / maligne Lymphome}
			\item \textbf{Hodencarcinom}
			\item \textbf{Prostatacarcinom}
			\item \textbf{Mammacarcinom}
			\item \textbf{Cervixcarcinom}
			\item \textbf{Coloncarcinom}
		\end{itemize}