\section{Psychatrische Erkrankungen}
	\subsection{Übersicht}
		\begin{itemize}
			\item \textbf{Verwirrtheitszustände}
			\item \textbf{Psychosen}
			\item \textbf{Depression}
		\end{itemize}
	\subsection{Verwirrtheitszustände}
		\begin{itemize}
			\item \textbf{akute Verwirrtheit - Zeichen einer akuten Störung außerhalb des Gehirns, die den Gehirnstoffwechsel akut beeinflusst}
			\item \textbf{Blutdruck-oder Blutzuckerabfall (in den frühen Morgenstunden)}
			\item \textbf{Herz- und Kreislauf-Erkrankung (zB. Schlaganfall)}
			\item \textbf{Exsikkose, Störungen des Säure-Basen-Haushaltes}
			\item \textbf{akute fieberhafte Infekte}
			\item \textbf{Unverträglichkeit von Medikamenten, Narkose}
			\item \textbf{Mangelernährung (zB. Vit. B12, Folsäure)}
			\item \textbf{psychosoziale Ursachen}
			\item \textbf{Symptome}
				\begin{itemize}
					\item \textbf{Gedächtnisstörungen, Orientierungsstörungen}
					\item \textbf{Verlust von Vergangenheits- und Zukunftsbezug}
					\item \textbf{unklare Denkabläufe, planloses Handeln}
					\item \textbf{motorische Unruhe}
					\item \textbf{Erzählung meist zufälliger Gedanken (Konfabulationen)}
					\item \textbf{Bewußtseinsstörungen mit nachfolgender Erinnerungslücke}
				\end{itemize}
		\end{itemize}
	\subsection{Psychosen}
		\begin{itemize}
			\item \textbf{endogene Psychosen}
				\begin{itemize}
					\item \textbf{affektive Störungen: Depression, bipolar: MDK}
					\item \textbf{Wahnstörungen: Schizophrenie (M.Bleuler)}
				\end{itemize}
			\item \textbf{exogene, organische Psychosen = körperlich begründbare Psychosen}
				\begin{itemize}
					\item \textbf{psych. Störungen aufgrund Gehirnschädigung oder körperl. Erkrankung}
					\item \textbf{delirante Störungen}
					\item \textbf{chronisch organische Psychosen: Demenzen}
				\end{itemize}
			\item \textbf{Eßstörungen}
			\item \textbf{Zwangserkrankungen}
			\item \textbf{Angst-und Panikstörungen}
			\item \textbf{Persönlichkeitsstörungen und sexuelle	Störungen}
			\item \textbf{Mißbrauch und Abhängigkeit}
			\item \textbf{Suizid}
		\end{itemize}
	\subsection{Depression}
		\begin{itemize}
			\item \textbf{Definition}
				\begin{itemize}
					\item \textbf{Störung des Affekts, des Denkens und des Antriebs	aufgrund somatischer, psychogener, iatrogener Faktoren}
					\item \textbf{Auftreten erstmals im höheren Alter oder rezidivierende Phasen einer bereits länger dauernden Krankheitsgeschichte;}
					\item \textbf{möglich: Wechsel von depressiven mit manischen Phasen}
				\end{itemize}
			\item \textbf{Pathogenese (häufig multifaktoriell bedingt)}
				\begin{itemize}
					\item \textbf{genetische Prädisposition, zusätzlich Auslöser}
					\item \textbf{Folge schwerer Belastung}
					\item \textbf{Folge von Demenz}
					\item \textbf{Folge somatischer Erkrankungen}
					\item \textbf{Folge von Medikamenten}
				\end{itemize}
			\item \textbf{Symptome}
				\begin{itemize}		 
					\item \textbf{Antriebsstörung}
					\item \textbf{Denkstörung}
					\item \textbf{Affektstörung}
					\item \textbf{Begleitsymptome}
				\end{itemize}
			\item \textbf{Therapie: Antidepressiva}
				\begin{itemize} 
					\item \textbf{trizyklische AD}
					\item \textbf{Mao-Hemmer}
					\item \textbf{Serotonin-Wiederaufnahme-Hemmer}
				\end{itemize}
		\end{itemize}