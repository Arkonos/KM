\section{Psychatrische Erkrankungen}
	\subsection{Übersicht}
		\begin{itemize}
			\item \textbf{Verwirrtheitszustände}
			\item \textbf{Psychosen}
			\item \textbf{Depression}
		\end{itemize}
	\subsection{Verwirrtheitszustände}
		\begin{itemize}
			\item \textbf{akute Verwirrtheit – Zeichen einer akuten Störung außerhalb des Gehirns, die den Gehirnstoffwechsel akut beeinflusst}
			\item \textbf{Blutdruck-oder Blutzuckerabfall (in den frühen Morgenstunden)}
			\item \textbf{Herz- und Kreislauf-Erkrankung (zB. Schlaganfall)}
			\item \textbf{Exsikkose, Störungen des Säure-Basen-Haushaltes}
			\item \textbf{akute fieberhafte Infekte}
			\item \textbf{Unverträglichkeit von Medikamenten, Narkose}
			\item \textbf{Mangelernährung (zB. Vit. B 12, Folsäure)}
			\item \textbf{psychosoziale Ursachen}
			\item \textbf{Symptome}
				\begin{itemize}
					\item \textbf{Gedächtnisstörungen, Orientierungsstörungen}
					\item \textbf{Verlust von Vergangenheits- und Zukunftsbezug}
					\item \textbf{unklare Denkabläufe, planloses Handeln}
					\item \textbf{motorische Unruhe}
					\item \textbf{Erzählung meist zufälliger Gedanken (Konfabulationen)}
					\item \textbf{Bewußtseinsstörungen mit nachfolgender Erinnerungslücke}
				\end{itemize}
		\end{itemize}