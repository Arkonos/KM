\section{Erkrankungen des Atmungssystems}
\subsection{Atemwegserkrankungen}
	\subsubsection{Lungendiagnostik}
		\begin{itemize}
			\item \textbf{klinische Diagnostik}
				\begin{itemize}
					\item \textbf{Inspektion, Anamnese, klinische Untersuchung}
					\item \textbf{Perkussion} (Abklopfen, von Bildgebung verdrängt)
					\item \textbf{Auskulation} (Abhören mit Stethoskop)
				\end{itemize}
			\item \textbf{bildgebende Diagnostik}
				\begin{itemize}
					\item \textbf{Thorax-Röntgen, Durchleuchtung}
					\item \textbf{Sonographie} (Ultraschall)
					\item \textbf{CT, MRT}
					\item \textbf{nuklearmedizinische Untersuchungen (Szintigraphie)}
					\item \textbf{Kontrastmitteluntersuchungen} 
				\end{itemize}
			\item \textbf{Labor-Diagnostik}
				\begin{itemize}
					\item \textbf{Blutgasanalyse, pH-Wert}
					\item nicht funktionierende Lunge Arbeit kein pH ab
			\item \textbf{Lungenfunktionsuntersuchung}
				\begin{itemize}
					\item \textbf{Spirometrie} (Volumenuntersuchung)
					\item \textbf{Peak-flow-Meter}
					\item \textbf{Ganzkörperplethysmographie} (unüblich)
				\end{itemize}
			\item \textbf{endoskopische Untersuchungen}
				\begin{itemize}
					\item \textbf{Bronchoskopie}
					\item \textbf{Mediastinoskopie} (gesamter Thoraxbereich)
				\end{itemize}
			\item \textbf{Pleurapunktion} (bei Pleuraerguss)
		\end{itemize}
	\subsubsection{Therapie}
		\begin{itemize}
			\item \textbf{Ausschalten von schädigenden Einflüssen}
			\item \textbf{medikamentös}
				\begin{itemize}
					\item \textbf{Antibiotika bei bakteriell-infektiöse Erkrankungen}
					\item \textbf{Entzündungshemmung (Cortison-Inhalation)}
					\item \textbf{bronchialerweiternde Med. (Bronchodilatantien), Bronchospasmolytika}
					\item \textbf{schleimlösende Med. (Mukolytika}
					\item \textbf{hustenreizdämpfende Med. (Antitussiva)}
				\end{itemize}
			\item \textbf{Sauerstoffgabe bei Mangel}
			\item \textbf{ev. Entwässerung}
			\item \textbf{atemstimulierende Maßnahmen}
			\item \textbf{atemunterstützende Lagerungen}
			\item \textbf{Lockern, Lösen und Absaugen von Sekret}
			\item \textbf{Inhalationen}
		\end{itemize}
		
\subsection{Erkrankungen des Atmungssystems}
	\subsubsection{Übersicht}
		\begin{itemize}
			\item \textbf{Bronchitis}
				\begin{itemize}
					\item \textbf{akute Bronchitis}
				\end{itemize}
			\item \textbf{COPD}
				\begin{itemize}
					\item \textbf{chronische Bronchitis}
					\item \textbf{Lungenemphysem}
				\end{itemize}
			\item \textbf{Asthma bronchiale}
			\item \textbf{Pneumonie}
			\item \textbf{Lungenembolie}
			\item \textbf{Lungenödem}
		\end{itemize}
	\subsubsection{akute Bronchitis}
		\begin{itemize}
			\item \textbf{Definition}
				\begin{itemize}
					\item \textbf{akute Entzündung der Schleimhaut der Atemwege}
				\end{itemize}
			\item \textbf{Ätiologie}
				\begin{itemize}
					\item \textbf{meist viral}
				\end{itemize}
			\item \textbf{Symptome}
				\begin{itemize}
					\item \textbf{Husten, lokale und ev. allg. Entzündungszeichen}
				\end{itemize}
			\item \textbf{Komplikationen}
				\begin{itemize}
					\item \textbf{Pneumonie, Übergang in chron. Bronchitis}
				\end{itemize}
			\item \textbf{Diagnostik}
				\begin{itemize}
					\item \textbf{klinischer Verlauf; ev. Erregerdiagnostik, ev. Thorax-Röntgen}
				\end{itemize}
			\item \textbf{Therapie}
				\begin{itemize}
					\item \textbf{symptomatische Therapie; ev. Antibiotika}
				\end{itemize}
		\end{itemize}
	\subsubsection{COPD}
		\begin{itemize}
			\item \textbf{"chronic obstructive pulmonary disease"}
			\item \textbf{chronische Lungenerkrankung, die mit Einengung der Atemwege einhergeht (Obstruktion)}
				\begin{itemize}
					\item \textbf{chronische Bronchitis}
					\item \textbf{Lungenemphysem}
				\end{itemize}
			\item \textbf{chronische Bronchitis}
				\begin{itemize}
					\item \textbf{Definition}
						\begin{itemize}
							\item \textbf{Husten in 2 aufeinanderfolgenden Jahren mind. 3 Monate}
							\item \textbf{bei zusätzlicher Obstruktion = COPD}
						\end{itemize}
					\item \textbf{Ätiologie}
						\begin{itemize}
							\item \textbf{Rauchen}
							\item \textbf{andere inhalative Belastungen}
							\item \textbf{akute Bronchitis}
						\end{itemize}
		\pagebreak
					\item \textbf{Symptome}
						\begin{itemize}
							\item \textbf{Husten, ev. anfallsartig}
							\item \textbf{Auswurf (besonders morgens)}
							\item \textbf{vermehrte Schleimabsonderung}
							\item \textbf{Umwandlung des Flimmerepithels in Plattenepithel}
							\item \textbf{später wird die Bronchiolenwand dünner und erschlafft $\rightarrow$ bei verstärkter Ausatmung kommt es zum Kollaps des Bronchus $\rightarrow$ Lungenemphysem}
						\end{itemize}
				\end{itemize}
			\item \textbf{Lungenemphysem}
				\begin{itemize}
					\item \textbf{Vergrößerung / Erweiterung der Bronchiolen und Alveolen, Überblähung, Elastizitätsverlust $\rightarrow$ irreversibler Zerstörung der Alveolen}
					\item[$\rightarrow$] \textbf{Vergrößerung des Totraumes und Verkleinerung der Gasaustauschfläche}
					\item \textbf{Symptome:}
						\begin{itemize}
							\item \textbf{Dyspnoe, ev. Zyanose, Husten ohne Auswurf}
							\item \textbf{ev. Bronchospasmen mit erschwerter Exspiration (Atemgeräusche!)}
							\item \textbf{"Fassthorax"}
						\end{itemize}
				\end{itemize}
			\end{itemize}
			\item \textbf{Risikofaktoren}
				\begin{itemize}
					\item \textbf{Rauchen!}
					\item \textbf{inhalative Belastungen (beruflich, Luft, Ozon, Autoabgase!)}
					\item \textbf{rezidivierende Atemwegsinfekte}
					\item \textbf{genetische Disposition}
				\end{itemize}
			\item \textbf{Komplikationen}
				\begin{itemize}
					\item \textbf{zunehmende Ateminsuffizienz}
					\item \textbf{Druckerhöhung im Lungenkreislauf $\rightarrow$ Rechtsherzbelastung, Rechtsherzinsuffizienz („Cor pulmonale“)} (Cor, da vom Herzen kommend)
					\item \textbf{Pneumonien (resistente Problemkeime!)}
					\item \textbf{Pneumothorax (durch Platzen einer großen Emphysemblase)}
				\end{itemize}
		\end{itemize}
	\subsubsection{Asthma bronchiale}
		\begin{itemize}
			\item \textbf{Definition}
				\begin{itemize}
					\item \textbf{chronische, nicht erregerbedingte Entzündung der Atemwege mit Atemwegsobstruktion}
				\end{itemize}
			\item \textbf{Ätiologie}
				\begin{itemize}
					\item \textbf{allergisch}
					\item \textbf{nicht allergisch (Infekte, Luftverschmutzung, Kälte, Belastungen, Medikamente)}
				\end{itemize}
			\item \textbf{Symptome}
				\begin{itemize}
					\item \textbf{Atemnot (bes. Exspiration!) und Hustenattacken (bes. morgens) durch}
						\begin{itemize}
							\item \textbf{Bronchospasmus}
							\item \textbf{Ödem $\rightarrow$ Schwellung}
							\item \textbf{zähes Sekret}
						\end{itemize}
				\end{itemize}
			\item \textbf{Komplikationen}
				\begin{itemize}
					\item \textbf{Atemwegsinfekte, Pneumonien}
					\item \textbf{Lungenemphysem und COPD}
					\item \textbf{"Status asthmaticus" mit Atemstillstand und/oder Rechtsherzversagen}
					\item \textbf{Cor pulmonale}
				\end{itemize}
		\end{itemize}
	\subsubsection{Pneumonie}
		\begin{itemize}
			\item \textbf{Definition}
				\begin{itemize}
					\item \textbf{Entzündungen des Lungengewebes}
				\end{itemize}
			\item \textbf{Einteilung}
				\begin{itemize}
					\item \textbf{nach Verlauf bzw. Erreger in typische / atypische Pneumonie}
						\begin{itemize}
							\item \textbf{typisch: akuter Beginn, hohes Fieber, Tachykardie, Husten mit Auswurf, Schmerzen beim Atmen, Dyspnoe, ev. Zyanose}
							\item \textbf{atypisch: Symptomatik wenig ausgeprägt; oft bei zuvor gesunden, jüngeren Patienten, meist nach grippaler Vorerkrankung}
						\end{itemize}
					\item \textbf{nach Lokalisation in Lobärpneumonie / Bronchopneumonie} (Lappen / verstreut)
				\end{itemize}
			\item \textbf{Komplikatinoen}
				\begin{itemize}
					\item \textbf{respiratorische Insuffizienz}
					\item \textbf{Ausbreitung innerhalb der Lunge (Lungenabszess) und in den Pleuraspalt (Pleuritis)}
					\item \textbf{Sepsis, Schock mit Herz-Kreislauf-Versagen}
					\item \textbf{bei Bettruhe und Exsikkose: cave Thromboembolie!}
				\end{itemize}
			\item \textbf{Diagnostik}
				\begin{itemize}
					\item \textbf{Thoraxröntgen}
					\item \textbf{Blutbild}
						\begin{itemize}
							\item \textbf{Leukozytose mit Linksverschiebung (typisch bei bakterieller Pneumonie)}
							\item \textbf{erhöhtes CRP und erhöhte BSG}
							\item \textbf{BGA zur Einschätzung der Atemsituation}
						\end{itemize}
					\item \textbf{ev. Erregernachweis} aus dem Auswurf
				\end{itemize}
			\item \textbf{Therapie}
				\begin{itemize}
					\item \textbf{symptomatisch}
					\item \textbf{Erregerbekämpfung (Antibiotika, antiviral, antimykotisch)}
					\item \textbf{Inhalationen, Atemgymnastik}
					\item \textbf{ausreichende Flüssigkeitszufuhr}
				\end{itemize}
		\end{itemize}
	\subsubsection{Lungenembolie}
		\begin{itemize}
			\item \textbf{Definition}
				\begin{itemize}
					\item \textbf{Verschluss einer Lungenarterie durch venösen Thrombo-Embolus}
					\item \textbf{Folge: belüftetes, aber nicht durchblutetes Areal $\rightarrow$ Druckerhöhung $\rightarrow$ Rechtsherzbelastung}
				\end{itemize}
			\item \textbf{Ätiologie}
				\begin{itemize}
					\item \textbf{Thromben aus den tiefen Bein- und Beckenvenen}
					\item sehr \textbf{selten: anderes Emboliematerial (Fettembolie bei Polytrauma, Trümmerfrakturen; Luftembolie $\dots$)}
				\end{itemize}
			\item \textbf{Risikofaktoren (siehe Thrombose / Embolie)}
				\begin{itemize}
					\item \textbf{vorübergehende}
						\begin{itemize}
							\item \textbf{eingeschränkte Mobilität und Immobilität}
							\item \textbf{postoperativ (cave: Hüft-oder Bein-OP!), posttraumatisch}
							\item \textbf{Schwangerschaft, Wochenbett}
							\item \textbf{Rauchen}
							\item \textbf{Pille plus Rauchen}
						\end{itemize}
			\pagebreak
					\item \textbf{permanente Risikofaktoren}
						\begin{itemize}
							\item \textbf{Alter}
							\item \textbf{maligne Erkrankungen (paraneoplastische Syndrome)}
							\item \textbf{Übergewicht}
						\end{itemize}
				\end{itemize}
			\item \textbf{Symptome}
				\begin{itemize}
					\item \textbf{unspezifisch und abhängig vom Schweregrad}
						\begin{itemize}
							\item \textbf{von symptomlos (stumm) bis akutes Rechtsherzversagen (Cor pulmonale) mit akutem Herz-Kreislauf-Stillstand}
						\end{itemize}
					\item \textbf{Dyspnoe (Atemnot), Tachypnoe, Tachykardie}
					\item \textbf{Brustbeklemmung (Patient will aufrecht sitzen!), atemabhängiger Thoraxschmerz}
					\item \textbf{Bluthusten (Hämoptysen)}
					\item \textbf{Unruhe, Angst}
				\end{itemize}
			\item \textbf{Komplikationen}
				\begin{itemize}
					\item \textbf{akutes Cor pulmonale mit Abfall des HMV} wenn großer Ast verlegt
					\item \textbf{Schock}
					\item \textbf{Lungeninfarkt}
				\end{itemize}
			\item \textbf{Diagnostik}
				\begin{itemize}
					\item \textbf{EKG} (um Herzinfarkt auszuschließen, ähnliche Symptomatik)
					\item \textbf{Röntgen-Thorax}
					\item \textbf{CT}
					\item \textbf{Lungenszintigramm, Pulmonalisangiographie, Venensonographie}
				\end{itemize}
			\item \textbf{Therapie}
				\begin{itemize}
					\item \textbf{Lungenembolie ist ein akuter Notfall!}
					\item \textbf{Sofortmaßnahmen}
						\begin{itemize}
							\item \textbf{absolute Bettruhe, Oberkörper hochlagern, Atemfunktion sichern, Schmerztherapie}
						\end{itemize}
					\item \textbf{medikamentös}
						\begin{itemize}
							\item \textbf{Blutgerinnungshemmung ("Antikoagulation")}
							\item \textbf{Thrombus-Auflösung ("Lysetherapie")}
						\end{itemize}
					\item \textbf{operativ}
						\begin{itemize}
							\item \textbf{Entfernung des Thromboembolus ("Thrombektomie")}
							\item \textbf{IVC Filter}
						\end{itemize}
				\end{itemize}
		\end{itemize}
	\subsubsection{Lungenödem}
		\begin{itemize}
			\item \textbf{Definition}
					\begin{itemize}
						\item \textbf{durch starken Rückstau von Blut in den Lungenkreislauf tritt Flüssigkeit in die Alveolen über}
					\end{itemize}
			\item \textbf{Ursache}
					\begin{itemize}
						\item \textbf{Links-Herz-Insuffizienz ("Rückwärtsversagen")}
						\item \textbf{Folge: Behinderung des Gasaustausches}
					\end{itemize}
			\item \textbf{Symptome}
					\begin{itemize}
						\item \textbf{Dyspnoe, Zyanose, "Blubbern"}
						\item \textbf{Husten mit schaumig / blutigem Auswurf}
						\item \textbf{ev. Brustschmerz}
						\item \textbf{Tachykardie}
					\end{itemize}
			\item \textbf{Therapie}
					\begin{itemize}
						\item \textbf{Lagerung, O$_2$-Gabe, Schmerz- und Herz-Medikamente}
						\item \textbf{Entwässern} (Diuretika)
						\item \textbf{ev. Beatmen}
					\end{itemize}
		\end{itemize}